\documentclass[]{article}
\usepackage{hyperref}
\usepackage[a4paper, total={6in, 8in}]{geometry}
\usepackage{caption}
\usepackage{algpseudocodex}
\usepackage{algorithm}
\usepackage{amsmath}
\usepackage{listings}
\usepackage{graphicx}
\usepackage{setspace}
\usepackage{subfiles}
\usepackage{xcolor}
\usepackage{enumerate}
\usepackage{xepersian}
\settextfont{XB Niloofar}

\begin{document}
صحیح یا غلط بودن  گزاره‌های زیر را مشخص کنید و دلیل خود را نیز بیان کنید.
\vspace{0.2cm}
\begin{enumerate}[(A)]
    \item {ﭘﺸﺘﻪ ﮐﻪ ﯾﮑﯽ ﺍﺯ ﺑﺨﺶ ﻫﺎﯼ ﺣﺎﻓﻈﻪ ﻓﺮﺁﯾﻨﺪ ﺍﺳﺖ ﺩﺭ ﻭﺍﻗﻊ ﺣﺎﻓﻈﻪ ﺍﯼ ﺩﺍﺋﻤﯽ ﺍﺳﺖ
          ﮐﻪ ﺑﻪ ﻫﻨﮕﺎﻡ ﻓﺮﺍﺧﻮﺍﻧﯽ ﺗﻮﺍﺑﻊ ﻣﻮﺭﺩ ﺍﺳﺘﻔﺎﺩﻩ ﻗﺮﺍﺭ ﻣﯽ ﮔﯿﺮﺩ ﻭ ﺣﺎﻭﯼ ﺁﺩﺭﺱ ﺑﺎﺯﮔﺸﺖ ﺍﺳﺖ.}
          \\
          تنها بخشی از گزاره درست است. پشته بخشی از حافظه فرایند است که با فراخوانی تابع،
          یک رکورد شامل  پارامترهای تابع، متغیرهای محلی و آدرس بازگشت به آن اضافه می‌شود.
          با بازگشت تابع، رکورد از پشته حذف می‌شود. پس اندازه‌ی آن در حین اجرای فرایند می‌تواند تغییر کند.
          این پشته حافظه موقت محسوب می‌شود نه حافظه‌ی دائمی چرا که اطلاعات تابع تنها
          تا زمانی که تابع درحال اجرا باشد نگهداری میشوند.

    \item {ﺯﻣﺎﻥ ﺑﻨﺪ CPU ﻭﻇﯿﻔﮥ ﺍﻧﺘﺨﺎﺏ ﯾﮏ ﻓﺮﺁﯾﻨﺪ ﺍﺯ ﺑﯿﻦ ﻫﻤﻪ ﻓﺮﺁﯾﻨﺪﻫﺎﯼ ﻣﻮﺟﻮﺩ ﺩﺭ ﺳﯿﺴﺘﻢ ﻭ
          ﺗﺨﺼﯿﺺ ﻫﺴﺘﮥ ﭘﺮﺩﺍﺯﺷﯽ ﺑﻪ ﺁﻥ ﺭﺍ ﺑﺮ ﻋﻬﺪﻩ ﺩﺍﺭﺩ.}
          \\ گزاره غلط است.
          زمان بند CPU
          فرآیند را از میان فرآیند‌های آماده انتخاب می‌کند و نه همه‌ی فرآیند های سیستم.

    \item  ﯾﮏ ﻧﺦ ﻋﺎﺩﯼ ﺩﺭ ﻃﻮﻝ ﺣﯿﺎﺕ ﺧﻮﺩ ﻣﻤﮑﻦ ﺍﺳﺖ ﺩﺭ LWP ﻫﺎﯼ ﻣﺘﻔﺎﻭﺗﯽ، ﺑﺨﺶ ﻫﺎﯾﯽ ﺍﺯ ﺍﺟﺮﺍﯼ ﺧﻮﺩ ﺭﺍ ﺑﮕﺬﺭﺍﻧﺪ.
    \\
    درستی این گزاره به طراحی سیستم و کرنل بستگی دارد.
    LWP یک رابط بین نخ‌های سطح کاربرو نخ‌های سطح کرنل محسوب می‌شود که توسط کرنل سیستم‌ عامل مدیریت می‌شود.
    بدین صورت که از دید نخ کاربر LWP یک پردازندۀ مجازی است که توسعه دهنده به هنگام توسعه‌ی برنامه‌ی کاربری می تواند آن را برای اجرای نخ های سطح کاربر زمان بندی نماید. 
    از سوی دیگر، هر LWP متصل به یک نخ کرنل است و در واقع این نخ های کرنل هستند که توسط سیستم عامل برای اجرا روی پردازندۀ فیزیکی زمان بندی می شوند.
    در مدل چند به چند، نخ‌های سطح کاربر به طور مستقیم به یک نخ‌ سطح کرنل متصل نیستند و به تعداد کوچکتر یا مساوی نخ های سطح کرنل تعمیم می شوند.
    در مدل دو سطحی، یک نخ سطح کاربر می‌تواند به یک نخ سطح کرنل متصل شود و اگر به نخ خاصی محدود نباشد، می‌تواند در LWPهای متفاوتی اجرا شود.
    پس در سیستمی که از مدل چند به چند یا مدل دو سطحی استفاده می‌شود نخ سطح کاربر می‌تواند در LWPهای متفاوتی بخش‌هایی از اجرای خود را بگذراند.

    \item ﻓﺮﺁﯾﻨﺪ ﻓﺮﺯﻧﺪ ﻣﯽ ﺗﻮﺍﻧﺪ ﺑﺮﻧﺎﻣﻪ ﻭ ﺩﺍﺩﮤ ﯾﮑﺴﺎﻥ ﺑﺎ ﻓﺮﺁﯾﻨﺪ ﭘﺪﺭ ﺧﻮﺩ ﺩﺍﺷﺘﻪ ﺑﺎﺷﺪ
          \\
          این گزاره درست است. وقتی یک فرایند فرزند ایجاد می‌شود،
          متن برنامه‌اش را به صورت کامل از پدر خود گرفته‌است پس برنامه آن‌ها یکسان است.
          چون PCB فرزند
          با تغییرات کمی با PCB
          فرآیند پدر خود یکسان است پس می‌تواند داده‌های آن را نیز داشته باشد.
          البته علاوه بر این‌ها می‌تواند به منابع پدر خود که از سیستم عامل گرفته‌است
          نیز دسترسی داشته باشد.

    \item ﺑﺮﺍﯼ ﺑﺮﻗﺮﺍﺭﯼ ﯾﮏ ﭘﯿﻮﻧﺪ ﺑﯿﻦ ﻫﺮ ﺯﻭﺝ ﻓﺮﺁﯾﻨﺪ، ﻓﺮﺁﯾﻨﺪﻫﺎ ﻧﯿﺎﺯﯼ ﺑﻪ ﺍﻃﻼﻉ ﺍﺯ ﺷﻨﺎﺳﮥ ﯾﮑﺪﯾﮕﺮ ﻧﺪﺍﺭﻧﺪ.
          \\
          این گزاره درست است اما بستگی به نوع پیاده‌سازی روش تبادل پیام دارد. یکی از راه‌های
          برقراری ارتباط بین هر زوج فرایند، تبادل پیام است. در تبادل پیام غیرمستقیم،
          فرایندها با استفاده از صندوق پستی برای یکدیگر پیام می‌فرستند
          و چون این صندوق پستی را سیستم‌عامل مدیریت می‌کند تنها به شناسه‌ای مشترک برای این صندوق احتیاج دارند
          و نه PID یکدیگر.
          \\
          اما در تبادل پیام مستقیم لازمه یک پیوند بین هر زوج فرایند، اطلاع داشتن از شناسه‌ی یکدیگر است. برخلاف پیاده‌سازی صندوق پستی،
          این پیوند تنها مختص یک زوج فرایند است و نمی‌توان از آن برای برقراری ارتباط با فرایندهای دیگر استفاده کرد. با در نظر گرفتن این روش، گزاره نادرست است.


\end{enumerate}
\end{document}