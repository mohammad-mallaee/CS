\documentclass[]{article}
\usepackage{hyperref}
\usepackage[a4paper, total={6in, 8in}]{geometry}
\usepackage{caption}
\usepackage{algpseudocodex}
\usepackage{algorithm}
\usepackage{amsmath}
\usepackage{listings}
\usepackage{graphicx}
\usepackage{setspace}
\usepackage{subfiles}
\usepackage{xcolor}
\usepackage{enumerate}
\usepackage{xepersian}
\settextfont{XB Niloofar}

\begin{document}
به سوالات زیر پاسخ کامل دهید.
\vspace{0.2cm}
\begin{enumerate}[(A)]
    \item ﻣﺪﻝ ﺍﺷﺘﺮﺍﮎ ﮔﺬﺍﺭﯼ ﺣﺎﻓﻈﻪ ﻭ ﻣﺪﻝ ﺍﺭﺗﺒﺎﻃﯽ ﺗﺒﺎﺩﻝ ﭘﯿﺎﻡ ﺭﺍ ﺗﻌﺮﯾﻒ ﻭ ﺑﺎ ﯾﮑﺪﯾﮕﺮ ﻣﻘﺎﯾﺴﻪ ﮐﻨﯿﺪ.
          \\
          برای برقراری ارتباط درون فرایندی دو روش وجود دارد: مدل اشتراک گذاری حافظه و مدل ارتباطی تبادل پیام.
          \vspace{-0.25cm}
          \paragraph*{مدل اشتراک گذاری حافظه}
          در این مدل، فرایندها با استفاده از یک ناحیه اشتراکی حافظه با یکدیگر ارتباط برقرار می‌کنند.
          این ارتباط نیازمند وجود فرآیندهای جدیدی به نام فرآیندهای ارتباطی است که برای فراهم کردن یک ناحیه برای حافظۀ اشتراکی ایجاد شده اند.
          با یک فراخوان سیستمی، سیستم عامل حافظه اشتراکی را در فضای آدرس فرآیند ارتباطی ایجاد می‌کند.
          پس از ایجاد حافظه اشتراکی، فرآیندها ناحیه‌ی حافظه‌ی اشتراکی را به فضای آدرسشان اضافه می‌کنند.
          فرایندها می‌توانند به طور مستقیم به حافظه دسترسی داشته باشند و با عمل read یا write ، بدون دخالت کرنل، ارتباط مستقیم داشته باشند.
          در این روش احتمال تداخل عمل نوشتن و خواندن داده توسط فرایندها وجود دارد.
          \vspace{-0.25cm}
          \paragraph*{مدل تبادل پیام}
          در این مدل، فرآیندهای همکار فعالیت هایشان را بدون اشتراک گذاری فضای مشترک هماهنگ کرده و با رد و بدل پیام با یکدیگر ارتباط برقرار می کنند.
          یک فرایند فرستنده پیام و دیگری گیرنده پیام است. برای شکل گیری پیوند ارتباطی و ارسال پیام، فرایندها برای ارسال و دریافت پیام نیازمند فراخوان سیستمی و دخالت کرنل هستند.
          نوع پیوند ارتباطی می‌تواند مستقیم یا غیرمستقیم باشد.
          \\
          در ارتباط مستقیم، یک پیوند ارتباطی دقیقا برای دو فرآیند در نظر گرفته می‌شود و بین هر زوج فرآیند، دقیقا یک پیوند وجود دارد.
          فرایندها از شناسه‌ی یکدیگر مطلع هستند و نام دریافت‌کننده و ارسال‌کننده باید به صورت صریح مشخص باشد.
          \\
          در ارتباط غیرمستقیم، برای پیاده‌سازی این روش از صندوق پستی استفاده می‌شود و فرایندها پیام خود را در صندوق گذاشته یا پیام را حذف می‌کنند.
          یک پیوند می تواند به بیش از دو فرآیند اختصاص داشته باشد و بین هر زوج فرآیند می تواند بیش از یک پیوند وجود داشته باشد.
          هر پیوند منطبق با یک صندوق پستی است و هر صندوق پستی شناسه یکتا دارد.
          \vspace{-0.25cm}
          \paragraph*{تفاوت‌های دو روش}
          \begin{enumerate}[1.]
              \item مدل حافظه اشتراکی سرعت بیشتری نسبت به تبادل پیام دارد زیرا تنها با یک فراخوان سیستمی حافظه ایجاد می‌شود
                    اما در تبادل پیام، برای ارسال هر پیام به استفاده از فراخوان سیستمی نیاز دارند و این مورد باعث کند بودن ارتباط می‌شود.
              \item برای ایجاد هماهنگی بین فرایندها و حفاظت از داده، مدل حافظه اشتراکی نیازمند روش‌هایی مانند سمافور است.
                    اما در تبادل پیام، در نوع پیاده‌سازی هماهنگی در نظر گرفته شده
              \item پیاده سازی مدل تبادل پیام در سیستم های توزیع شده به مراتب ساده تر از مدل اشتراک حافظه است چرا که در سیستم های توزیع شده، فرآیندها بر روی سیستم های مختلفی مستقر هستند.
          \end{enumerate}

          \pagebreak

    \item ﺣﺎﻟﺖ ﻫﺎﯾﯽ ﺭﺍ ﺑﯿﺎﻥ ﮐﻨﯿﺪ ﮐﻪ ﺑﺎﻋﺚ ﭘﺎﯾﺎﻥ ﯾﺎﻓﺘﻦ ﻓﺮﺁﯾﻨﺪ ﻣﯽ ﺷﻮﺩ.
          \begin{enumerate}[-]
              \item اجرای فرایند به پایان رسیده باشد.
              \item اجرای فرایند بیشتر از زمان تعیین شده طول بکشد.
              \item فرایند به منابع مورد نیاز خود دسترسی ندارد و یا می‌خواهد به منابع غیرمجاز دسترسی داشته باشد.
              \item فرایند نیازمند حافظه‌ی بیشتر از حافظه موجود باشد.
              \item سیستم عامل به دلایلی مانند deadlock فرایند را متوقف کند.
              \item کاربر درخواست پایان یافتن فرآیند را بدهد.
              \item در برنامه خطایی رخ دهد که توسط برنامه مدیریت نشده باشد مانند خطای تقسیم بر صفر.
              \item اجرای آن توسط فرآیند پدر متوقف شده باشد.
          \end{enumerate}

    \item {ﺑﺮﻧﺎﻣﻪ ﻧﻮﯾﺴﯽ ﭼﻨﺪﻫﺴﺘﻪ ﺍﯼ ﻭ ﭼﻨﺪ ﻧﺨﯽ ﺭﺍ ﺗﻌﺮﯾﻒ ﻭ ﺑﺎ ﯾﮑﺪﯾﮕﺮ ﻣﻘﺎﯾﺴﻪ ﮐﻨﯿﺪ.
          ﺩﺭ ﭼﻪ ﺣﺎﻟﺖ ﻫﺎﯾﯽ ﺍﺳﺘﻔﺎﺩﻩ ﺍﺯ ﺳﯿﺴﺘﻢ ﻫﺎﯼ ﺗﮏ ﻫﺴﺘﻪ ﺍﯼ ﻣﻨﺎﺳﺐ ﺗﺮ ﺍﺯ ﺳﯿﺴﺘﻢ ﻫﺎﯼ ﭼﻨﺪ ﻫﺴﺘﻪ ﺍﯼ ﺍﺳﺖ؟}
          در برنامه نویسی فرآیندهای چند نخی، چند نخ درون یک فرایند ایجاد می‌شوند و هم روندی را ممکن می‌سازد.
          نخ های مربوط به فرآیند از منابع، کد و داده‌ی مشترک استفاده می‌کنند و در فضای آدرس یکسانی فعال هستند پس برقراری ارتباط بین آنها راحت‌تر است.
          با استفاده از نخ‌ها، اگر بخشی از فرآیند بلاک شود یا در حال انجام محاسبات طولانی باشد، بخش های دیگر اجرا شده و قابل استفاده هستند.
          در این راستا، مزایای این نوع برنامه نویسی را می توان در چهار ردۀ کلی زیر تقسیم بندی کرد:  پاسخگویی سریع، اشتراک گذاری منابع، صرفه اقتصادی، مقیاس پذیری.
          \\
          در برنامه نویسی چند هسته‌ای هدف تامین مکانیزمی است که با هم روندی و به کارگیری هم زمان چند هسته پردازشی، سبب استفاده‌ی بهتر از CPU و در نهایت تسریع محاسبات شود.
          \\
          یک طراح سیستم عامل می بایست الگوریتم های زمان بندی پیچیده‌تری را برای استفادۀ حداکثری از پردازنده طرح ریزی کند
          به طوری که ابتدا تشخیص دهد کدام وظایف می‌توانند به صورت موازی اجرا شوند، برنامه را به این وظایف تقسیم کند،
          از وجود تعادل بین وظایف اطمینان حاصل کند و داده‌ها را مدیریت کند.
          چالش های پیش روی برنامه نویسان در مواجهه با چنین سیستم‌هایی را می‌توان در پنج رده تقسیم بندی نمود: شناسایی وظایف، تعادل، جداسازی داده، وابستگی داده، تست و اشکال زدایی.
          \\
          تفاوت‌های برنامه‌ نویسی چند نخی و چند هسته‌ای:
          \begin{enumerate}[1.]
              \item اشتراک منابع: در برنامه چند نخی، برقراری ارتباط راحت‌تر از برنامه چندهسته‌ای است
                    زیرا آدرس حافظه، منابع و داده‌های مشترک دارند و نیازی به استفاده از روش‌های IPC ندارند.
              \item اجرای موازی وظایف: در برنامه چند هسته‌ای اجرای موازی از طریق اجرای وظایف بر چندین هسته پردازشی صورت می‌گیرد
                    اما در برنامه چند نخی، هم روندی با اجرای همزمان چند نخ درون یک فرایند امکان پذیر است.
              \item پیچیدگی: برنامه چند هسته‌ای به دلیل تقسیم کار و مدیریت ارتباط بین فرایندها پیچیدگی بیشتری دارد.
          \end{enumerate}
          در سیستم تک هسته‌ای نخ‌ها صرفاً در بین هم اجرا می شوند یعنی در یک زمان تنها یک نخ اجرا می شود.
          در سیستم چند هسته‌ای نخ های مجزا می توانند بر روی هسته های مختلف در یک زمان و به صورت موازی اجرا شوند.
          \\
          در سیستم‌هایی که مصرف انرژی و اندازه واحد پردازشی اهمیت بیشتری نسبت سرعت پردازشی دارد،
          استفاده از یک پردازنده تک هسته‌ای گزینه‌ای مناسب‌تر و معقول‌تر است. به علاوه در سیستم‌هایی که
          برنامه‌ها معمولا ساده هستند و تضمین کارکرد درست آنها اولویت بیشتری دارد، استفاده از پردازنده تک هسته‌ای
          می‌تواند گزینه‌ی بهتری باشد چرا که پیچیدگی‌های سیستم چند پردازنده‌ای را ندارد.
          \pagebreak
    \item ﺗﺸﺮﯾﺢ ﮐﻨﯿﺪ ﻫﻤﮑﺎﺭﯼ ﺑﯿﻦ ﻓﺮﺁﯾﻨﺪﻫﺎ ﭼﻪ ﺳﻮﺩﯼ ﺩﺍﺭﺩ ﻭ ﭼﻪ ﻧﻮﻉ ﭘﯿﭽﯿﺪﮔﯽ ﻫﺎﯾﯽ ﺭﺍ ﺩﺭ ﺳﯿﺴﺘﻢ ﺍﯾﺠﺎﺩ ﻣﯽ ﮐﻨﺪ.
          \vspace{-0.5cm}
          \paragraph*{سود همکاری بین فرایندها}
          \begin{enumerate}[-]
              \item \textbf{اشتراک گذاری اطلاعات:}
                    ممکن است در چندین برنامۀ کاربردی، نیاز به دسترسی همزمان به یک قطعه اطلاعاتی مشترک باشد.
              \item \textbf{تسریع محاسبات :}
                    اگر بخواهیم یک تسک را سریع تر اجرا کنیم، می بایست آن را به تسک‌های کوچکتر بشکنیم و هرکدام را به طور موازی با یکدیگر اجرا کنیم.
                    برای یکپارچگی و عملکرد صحیح سیستم در چنین شرایطی نیازمند محیط کاری مشترک برای کنترل و مدیریت زیرتسک‌ها و در نهایت تسک اصلی هستیم.
              \item \textbf{ماژولار یا پیمانه ای بودن:}
                    ممکن است در سیستمی تمایل داشته باشیم که توابع سیستم را در چندین دسته از فرآیندها یا نخ های مجزا تقسیم بندی کنیم.
                    در چنین حالتی برای داشتن عملکرد صحیح و یکپارچه، داشتن محیط همکاری مشترک ضروری است.
          \end{enumerate}
          \vspace{-0.5cm}
          \paragraph*{پیچیدگی‌های همکاری فرایندها}
          \begin{enumerate}[-]
              \item \textbf{همگام سازی:}
                    ممکن است در چندین برنامۀ کاربردی، نیاز به دسترسی همزمان به یک قطعه اطلاعاتی مشترک باشد.
              \item \textbf{احتمال رخ دادن deadlock یا گرسنگی:}
                    فرایندها منابعی که دارند را رها نمی‌کنند و منتظر دسترسی به منبع مورد نیاز می‌مانند یا یک فرایند پس از دسترسی به منابع، آن را نگه میدارد و اجازه دسترسی دیگر فرایندها به آن را نمی‌دهد.
          \end{enumerate}
    \item {ﺗﺸﺮﯾﺢ ﮐﻨﯿﺪ ﺩﺭ ﺗﻌﻮﯾﺾ ﻣﺘﻦ ﻧﺦ ﻫﺎ، ﭼﻪ ﻓﻌﺎﻟﯿﺖ ﻫﺎﯾﯽ ﺗﻮﺳﻂ ﻫﺴﺘﻪ ﺑﺎﯾﺪ ﺍﻧﺠﺎﻡ ﺷﻮﺩ ﻭ ﺑﺎ ﺗﻌﻮﯾﺾ ﻣﺘﻦ ﻓﺮﺁﯾﻨﺪﻫﺎ ﭼﻪ ﺗﻔﺎﻭﺕ ﻫﺎﯾﯽ ﺩﺍﺭﺩ.}
          در تعویض متن نخ‌ها، پردازنده متن نخ را ذخیره می‌کند و با بارگذاری متن نخ جدید، آن را اجرا می‌کند.
          از آنجایی که نخ‌ها در یک فرایند هستند، فضای آدرس یکسان دارند و از منابع و داده‌ی مشترک استفاده می‌کنند، تعویض متن سریع‌تر و راحت‌تر انجام می‌شود.
          پردازنده فقط باید حالت فعلی نخ (شمارنده برنامه و رجیسترها) را باید تغییر دهد.
          \vspace{-0.25cm}
          \paragraph*{تفاوت‌ها}
          \begin{enumerate}[1.]
              \item تعویض متن نخ وقتی رخ می‌دهد که CPU حالت کنونی نخ را ذخیره می‌کند و اجرای نخ دیگری از همان فرایند را شروع می‌کند.
                    تعویض متن فرایندها زمانی‌ است که زمان‌بند سیستم عامل حالت کنونی برنامه در حال اجرا را ذخیره می‌کند و سراغ برنامه‌ای دیگر می‌رود.
              \item تعویض متن نخ نیازی به تعویض آدرس حافظه ندارد.
                    تعویض متن فرایند نیاز به تعویض آدرس حافظه دارد. به همین دلیل تعویض متن نخ کاراتر است.
              \item تعویض متن نخ نسبتا سریع‌تر و ارزان‌تر است.
          \end{enumerate}
\end{enumerate}
\end{document}