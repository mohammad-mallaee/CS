\documentclass[]{article}
\usepackage{hyperref}
\usepackage[a4paper, total={6in, 8in}]{geometry}
\usepackage{caption}
\usepackage{algpseudocodex}
\usepackage{algorithm}
\usepackage{amsmath}
\usepackage{listings}
\usepackage{graphicx}
\usepackage{setspace}
\usepackage{subfiles}
\usepackage{xcolor}
\usepackage{enumerate}
\usepackage{xepersian}
\settextfont{XB Niloofar}

\begin{document}
صحیح یا غلط بودن  گزاره‌های زیر را مشخص کنید و دلیل خود را نیز بیان کنید.
\vspace{0.2cm}
\begin{enumerate}[(A)]
      \item {انتخاب فرآیند برای اجرا توسط زمان بند CPU انجام می شود.}
            \\ گزاره صحیح است. این مولفه از میان فرآیندهای آماده‌ی اجرا در حافظه، فرآیندی را انتخاب و CPU را به آن اختصاص می دهد.

      \item {طرح های غیرقبضه ای در حالت کلی سربار بیشتری را به سیستم تحمیل می کنند.}
            \\ گزاره غلط است. طرح‌های قبضه‌ای به طور کلی سربار بیشتری به سیستم تحمیل می‌کنند اما خدمات بهتری به کاربر ارائه می‌کنند.
            یکی از دلایل آن تعویض متن غیرداوطلبانه است. ممکن است در حین اجرای فرایند، به علت ورود فرایند با اولویت بالاتر یا تمام شدن کوانتوم زمانی،
            CPU از فرایند گرفته شود و به فرایند دیگری داده شود و تعویض متن داریم.
            درحالیکه در طرح غیر قبضه‌ای، تعویض متن درصورتی انجام می‌‌شود که اجرای فرایند به اتمام برسد یا وقفه‌ای رخ دهد.
            به طور کلی، در طرح غیرقبضه‌ای تعویض متن کمتر و در نتیجه سربار کمتری داریم.

      \item  زمان اجرای کامل یک فرآیند برابر حاصل جمع زمان پاسخ و زمان اجرا است.
            \\ به طور کلی، گزاره غلط است زیرا با تعریف اصلی زمان اجرای کامل همخوانی ندارد.
            زمان اجرای کامل (زمان بازگشت) فرایند، برابر است با مجموع زمان انتظار و زمان اجرای فرایند.
            زمان پاسخ یعنی فاصله زمانی ورود فرایند به صف تا اولین باری که CPU را دریافت می‌کند.
            \\
            در طرح زمان‌بندی غیرقبضه‌ای این گزاره صحیح است چرا که زمان پاسخ با زمان انتظار فرایند برابراست.
            اما به طور کلی آن را غلط محسوب می‌کنیم.

      \item در تابع انتخاب زمان بندی RR فرآیندی انتخاب می شود که از آخرین باری که به صف آماده اضافه شده است تا لحظه ی کنونی،
            زمان کمتری در صف آماده منتظر بوده است.
            \\ گزاره غلط است. در این تابع، پس از هر کوانتوم زمانی فرایندی که در ابتدای صف باشد انتخاب می‌شود.
            ممکن است در همان لحظه فرایندی به صف اضافه شود و طبق این گزاره، باید توسط الگوریتم انتخاب شود زیرا کمترین زمان را در صف سپری کرده.
            اما این اتفاق نمی‌افتد. فرایند به انتهای صف اضافه شده و الگوریتم فرایندی را که در ابتدای صف است انتخاب می‌کند.

      \item نحوه ی انتظار برای دریافت مجوز ورود به ناحیه ی بحرانی توسط یک فرآیند تنها با وقفه صورت می گیرد.
            \\ گزاره غلط است. دو حالت کلی برای دریافت مجوز و ورود به فرآیندها به ناحیۀ بحرانی وجود دارد:
            \\
            انتظار مشغول \lr{(busy waiting)} :
            فرآیند مکرراً شرطی را بررسی می کند و صبر می کند تا شرط برقرار شود و به ادامۀ اجرای خود بپردازد.
            \\
            مسدود کردن \lr{(blocking)} :
            اجرای فرآیند به طور کامل متوقف می شود.

\end{enumerate}
\end{document}