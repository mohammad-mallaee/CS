\documentclass[]{article}
\usepackage{hyperref}
\usepackage[a4paper, total={6in, 8in}]{geometry}
\usepackage{caption}
\usepackage{algpseudocodex}
\usepackage{algorithm}
\usepackage{amsmath}
\usepackage{listings}
\usepackage{graphicx}
\usepackage{setspace}
\usepackage{subfiles}
\usepackage{xcolor}
\usepackage{enumerate}
\usepackage{xepersian}
\settextfont{XB Niloofar}

\begin{document}
سیستمی را که شامل چهار فرآیند همزمان و دو منبع قابل استفاده مجدد است در نظر بگیرید. به شرط اینکه هر فرآیند حداکثر به دو منبع
نیاز داشته باشد، تعداد وضعیت های بن بست
\lr{(Deadlock States)}
در این سیستم حداکثر چند حالت است؟
\\
بن بست در یک گروه از فرآیندهای همکار بدین معنا است که هیچ کدام از آن‌ها از منبع اشتراکی
نمی‌توانند استفاده کنند چرا که در هر فرآیند منتظر فرآیند های دیگر می‌ماند.
البته برای این که بن‌بست رخ دهد این انتظار باید نامحدود باشد.
\\
با توجه به فرض بالا در صورتی که یک فرآیند به هیچ منبع اشتراکی احتیاج نداشته باشد، مستقل از
سایر فرآیند هاست و در اینجا مورد بررسی قرار نمی‌گیرد ولی اگر آن را در نظر بگیریم در واقع بن‌بستی نخواهیم داشت.
حال اگر فرض کنیم تعدادی از این فرآیندها به یک منبع و تعدادی دیگر به دو منبع اشتراکی برای اجرا نیاز داشته باشند،
شرایط بن‌بست را مورد بررسی قرار می‌دهیم.

فرض می‌کنیم در یک لحظه دو فرآیند که هر کدام به هر دو منبع اشتراکی نیاز دارند،
تنها یکی از آن‌ها را در اختیار داشته باشند و تا زمانی که منبع دیگر را در اختیار نگیرند این منبع را رها نکنند
(در حالت های دیگر بن بست رخ نخواهد داد و در بدترین حالت livelock خواهیم داشت).
تعداد حالت های رخ دادن بن‌بست با فرض این که تعداد فرآیندهایی که به هردو احتیاج دارند x و حداکثر چهار باشد،
$\binom{x}{2}$ خواهد بود.


\end{document}