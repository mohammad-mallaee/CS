\documentclass[]{article}
\usepackage{hyperref}
\usepackage[a4paper, total={6in, 8in}]{geometry}
\usepackage{caption}
\usepackage{algpseudocodex}
\usepackage{tabularray}
\usepackage{algorithm}
\usepackage{amsmath}
\usepackage{listings}
\usepackage{graphicx}
\usepackage{setspace}
\usepackage{subfiles}
\usepackage{xcolor}
\usepackage{enumerate}
\usepackage{spverbatim}
\usepackage{xepersian}
\settextfont{XB Niloofar}

\begin{document}
فرض کنید در کد زیر
\lr{p1} و \lr{p2}
همروند باشند و سمافورهای x و y دو سمافور دودویی و هر دو مقدار اولیه یک را داشته باشند. همچنین
فرض کنید دستور print در یک چرخه دستورالعمل اجرا شود.
\begin{latin}
    \begin{center}
        \begin{spverbatim}
            p2:                         p1:
                ----                        ----
                wait(y)                     wait(x)
                while true do               while true do
                    print "B"                   print "A"
                signal(y)                   signal(x)
                ----                        ----
        \end{spverbatim}
    \end{center}
\end{latin}

آیا هر تلفیقی از A ها و B ها ممکن است در خروجی ظاهر شوند؟ توضیح دهید.

با توجه به اینکه این دو فرآیند از سمافور‌های متفاوتی استفاده می‌کنند هیچ کدام منتظر دیگری نمی‌ماند.
با فرض اینکه این دو فرآیند همروند اجرا شوند، بنابراین در هر قسمتی از اجرا ممکن است اجرای آن‌ها متوقف شود
و CPU به فرآیند دیگر اختصاص داده شود که این بدان معناست که
خروجی این برنامه‌ها از الگوی خاصی پیروی نمی‌کند و هر حالتی ممکن است داشته باشد.
\end{document}