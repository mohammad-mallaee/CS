\documentclass[]{article}
\usepackage{hyperref}
\usepackage[a4paper, total={6in, 8in}]{geometry}
\usepackage{caption}
\usepackage{algpseudocodex}
\usepackage{tabularray}
\usepackage{algorithm}
\usepackage{amsmath}
\usepackage{listings}
\usepackage{graphicx}
\usepackage{setspace}
\usepackage{subfiles}
\usepackage{xcolor}
\usepackage{enumerate}
\usepackage{xepersian}
\settextfont{XB Niloofar}

\begin{document}
چرخۀ حیات یک فرآیند پدر و فرزندش که اجرای فرزند، غیر همروند با پدر و فضای حافظه ی آن نیز مستقل از والدش است را تشریح
کنید (در صورت لزوم مفروضاتی به مسئله اضافه کنید.) \\
وقتی فرآیند فرزند توسط والد تشکیل می‌شود، این فرزند یک کپی از فضای حافظه‌ای والد را دریافت میکند، که شامل متن، پشته، هرم و بقیه‌ی بخش‌های ممکن می‌باشد.
اگر فضا‌های حافظه‌ای فرآیند‌ها مجزا باشند، اجرا شدن یک فرآیند روی فضای حافظه‌ای فرآیند دیگر تأثیر ندارد و به اصطلاح فرآیند‌ها مستقل از هم اجرا می‌شوند.
اگر این دو فرآیند، حالت اجراییشان هم‌پوشانی نداشته باشد، آن‌گاه رقابت برای منابعی مانند CPU و RAM کاهش می‌یابد.
و چون فضای حافظه‌ای متفاوت است، فرآیند والد و فرزند، به طور مستقیم به متغیر‌ها و داده‌های یکدیگر دسترسی ندارند و باید از روش‌های ارتباطی غیرمستقیم (IPC) استفاده کنند.\\
\textbf{دو نکته مهم:}
\\
اگر یک فرآیند فرزند خاتمه یابد، آن‌گاه تبدیل به یک فرآیند زامبی می‌شود تا وقتی که فرآیند پدر از شرایط فرزندش (معمولاً توسط $wait()$) باخبر شود.
اگر فرآیند والد از دستور $wait()$ استفاده نکند، فرآیند فرزند زامبی خواهد ماند و یک بخش کوچکی از حافظه را آشغال خواهد کرد.
\\
اگر فرآیند پدر زودتر خاتمه یابد، فرآیند فرزند یتیم خواهد شد و تا وقتی که سیستم‌عامل خاتمه‌اش ندهد یتیم خواهد ماند.
% \begin{table}[H]
%     \def\arraystretch{1.75}
%     \begin{tabular}{|r|p{5cm}|r|}
%         \hline
%         ویژگی & مدلِ فضای حافظه‌ای اشتراکی و همروند & مدلِ فضای حافظه‌ای مجزا و غیرهمروند \\ \hline
%         دسترسیِ حافظه و همگام‌سازی & 
%         فرآیند والد و فرزند به متغیر‌های یکدیگر دسترسی مستقیم دارند.
%         یعنی نیازی به روش‌های نداریم.
%         & 21 \\ \hline
%         عملکرد & df & df \\ \hline
%         امنیت و یکپارچگی داده‌ه & df & df \\ \hline
%         استفاده & df & df \\ \hline
%     \end{tabular}
% \end{table}
% \begin{tabulary}{\linewidth}{LCL}
%     \hline
%     Short sentences      & \#  & Long sentences                                                 \\
%     \hline
%     This is short.       & 173 & This is much loooooooonger, because there are many more words.  \\
%     This is not shorter. & 317 & This is still loooooooonger, because there are many more words. \\
%     \hline
% \end{tabulary} 
\begin{table}[H]
    \begin{tblr}{|X[j,valign=m]|X[j,valign=m]|c|}
        \hline
        مدلِ فضای حافظه‌ای مجزا و غیرهمروند & مدلِ فضای حافظه‌ای اشتراکی و همروند  & ویژگی \\ \hline
        نیازی به همگام‌سازی در اجرای فرآیند‌ها نیست و انتقال داده در صورت لزوم به صورت غیر‌مستقیم صورت می‌گیرد.
        & فرآیند والد و فرزند به متغیر‌های یکدیگر دسترسی مستقیم دارند.
        یعنی نیاز به روش‌های ‌IPC نداریم. این ویژگی باعث افزایش سرعت ردوبدل شدن داده بین این دو فرآیند است. 
        البته از طرفی به دلیل دسترسی به سلول‌های حافظه‌ای یکسان، همگام‌سازی نخ اجرایی یک نقش کلیدی بازی خواهد کرد.
        & دسترسیِ حافظه و همگام‌سازی \\ \hline
        برای فرآیند‌های که اهداف نسبتاً متفاوت و غیر‌موازی دارند میتواند گزینه‌ی مناسبی باشد.
        نکته‌ی مثبت در این حالت این است که همگام‌سازیِ فرآیند‌ها، که در این مورد خاص به دلیل متفاوت بودن 
        اهداف فرآیند‌ها یک اضافه‌کاری (overhead) است، انجام نمیشود
        & اگر فرکانس انتقال بالا باشد، این روش بسیار می‌تواند مؤثر باشد، از این نظر که انتقال داده مستقیم است و
        همچنین عملیات تعویض متن $(context \: switch)$ در این روش سریع‌تر است.
        زیرا منابع به صورت اشتراکی مورد استفاده قرار می‌گیرد و CPU نیاز به دوباره بارگذاری‌کردن منابع ندارد.
        & عملکرد \\ \hline
    \end{tblr}
\end{table}
\vspace{-0.25cm}
\begin{table}[H]
    \begin{tblr}{|X[j,valign=m]|X[j,valign=m]|c|}
        \hline
        به دلیل مجزا بودن فضای حافظه‌ای، امکان بروز خطا‌های رخ داده در روش‌ دیگر وجود ندارد.
        & اختلال در فرآیند همگام‌سازی میتواند به مشکلاتی در این روش منجر شود. مثلاً از بین رفتن داده‌ها یا شرایط رقابتی که عمل‌کرد اجرای دستورات را پایین می‌آورد و حتی می‌تواند منجر به رقابت‌های بی‌پایان برای دسترسی به یک منبع خاص شود.
        & امنیت و یکپارچگی داده‌ها \\ \hline
        وقتی فرآیند‌های کار‌های متفاوتی را انجام می‌دهند و وظایف مستقلی دارند.
        مثال:
        \begin{LTR}
            database management systems \\
            system-level operations
        \end{LTR}
        & معمولاً در برنامه‌های چندرشته‌ای، که فرآیند‌های به هم وابسته‌اند و نیاز به انتقال بی‌درنگ و سریع داده‌ها است.
        مثال:
        \begin{LTR}
            web servers \\
            real time applications
        \end{LTR}
        & استفاده \\ \hline
    \end{tblr}
\end{table}

\end{document}