\documentclass[]{article}
\usepackage{hyperref}
\usepackage[a4paper, total={6in, 8in}]{geometry}
\usepackage{caption}
\usepackage{algpseudocodex}
\usepackage{algorithm}
\usepackage{amsmath}
\usepackage{listings}
\usepackage{graphicx}
\usepackage{setspace}
\usepackage{subfiles}
\usepackage{xcolor}
\usepackage{enumerate}
\usepackage{xepersian}
\settextfont{XB Niloofar}

\begin{document}
فرض کنید در یک سیستم کامپیوتری سازوکار وقفه وجود ندارد، در این صورت به سوالات زیر پاسخ کامل بدهید

\begin{enumerate}[(A)]
    \item چه چالش هایی برای سیستم به وجود خواهد آمد؟ \\
    در این سیستم کامپیوتری،‌
    CPU
    همواره همه‌ی دستگاه‌ها را بررسی می‌کند تا درصورت نیاز
    به آنها رسیدگی کند. در این سیستم وقت
    CPU
    برای بررسی دستگاه‌ها استفاده میشود که به معنای تلف شدن این زمان است و
    اگر تعداد دستگاه‌ها زیاد باشد یا اکثر مواقع بیکار باشند این موضوع بسیار بیشتر خواهد بود.
    این موضوع میتواند باعث تاخیر در اجرای فرآیند‌های مهم شود که احتیاج به اجرای آنی دارند و شاید
    این تاخیر باعث از دست رفتن فرآیند‌های شود که باید در زمان کوتاهی اجرا شوند و
    CPU
    نتواند آنها را اجرا کند.
    \item چنین سیستمی چه مزیت هایی خواهد داشت؟ \\
    این سیستم به نسبت سیستمی که از وقفه پشتیبانی میکند ساده‌تر است و پیاده‌سازی آن
    راحت‌تر و احتمالا کم‌هزینه‌تر است به علاوه به دلیل اینکه فاصله میان واکشی‌ها مشخص است پس 
    زمانبندی این سیستم دقیق‌تر و پیش‌بینی پذیر است که در بعضی از دستگاه‌ها مورد نیاز است.
    \item فرض کنید همان سیستم، مشغول یک عمل
    $I/O$
    است و همچنین فرآیندی بر روی آن در حال اجرا است که می دانیم عمل
    $I/O$
    پیش از این فرآیند به اتمام می رسد.
    اکنون برای مدیریت این سیستم سازوکاری ارائه دهید و اجرای آن روی فرآیند و عملیات موجود را گام به گام شرح دهید. \\
    در این سیستم به دلیل وجود نداشتن وقفه،
    CPU
    باید در زمان‌های مشخص عملیات واکشی از دستگاه
    I/O
    را انجام دهد. اگر کار این دستگاه تمام شده باشد، به آن رسیدگی و عملیات را خاتمه میدهد
    و در غیر اینصورت اجرای فرآیند را از سرمیگیرد.
    البته میتوانیم یک صف نیز برای عملیات‌ها
    I/O
    یا به صورت کلی وقفه‌ها نیز ایجاد کنیم که
    CPU
    به جای بررسی هریک از اجزا این صف را بررسی و دستورالعمل‌های موردنیاز را انجام دهد.
    اجرای این سیستم روی فرآیند و عملیات به صورت زیر خواهد بود:
    \begin{enumerate}[1-]
        \item اجرای فرآیند تا رسیدن به زمان واکشی وضعیت‌ها یا اتمام فرآیند
        \item اگر عملیات
        I/O
        به اتمام رسیده‌باشد، به آن خاتمه میدهد و منابع را آزاد میکند،
        در غیر اینصورت اجرای فرآیند را از سر می‌گیرد
    \end{enumerate}
\end{enumerate}

\end{document}