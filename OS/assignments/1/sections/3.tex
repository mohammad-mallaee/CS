\documentclass[]{article}
\usepackage{hyperref}
\usepackage[a4paper, total={6in, 8in}]{geometry}
\usepackage{caption}
\usepackage{algpseudocodex}
\usepackage{algorithm}
\usepackage{amsmath}
\usepackage{listings}
\usepackage{graphicx}
\usepackage{setspace}
\usepackage{subfiles}
\usepackage{xcolor}
\usepackage{enumerate}
\usepackage{xepersian}
\settextfont{XB Niloofar}

\begin{document}
با توجه به اینکه فرآیند های زامبی، منابع تخصیص یافته به خویش را رها کرده و عملا خاتمه یافته اند، آیا ممکن است داشتن فرآیند زامبی
به تعداد زیاد در سیستم اخلالی ایجاد کند؟ ( پاسخ خود را شرح دهید) \\
بله. تعداد زیاد فرآیند‌های زامبی در سیستم کامپیوتری می‌تواند مشکل ساز شود. اگرچه فرآیند‌های زامبی از منابعی مانند CPU و RAM استفاده نمی کنند.
اما مشکلات دیگری ممکن است رخ بدهد. برخی از این مشکلات عبارت‌اند از:
\vspace{-0.15cm}
\begin{itemize}
    \item \textbf{هدر رفت :PID}
    می‌دانیم که هر فرآیند زامبی یک سطر در بلاک کنترل فرآیند‌ها اشغال می‌کند. پس اگر تعداد این نوع فرآیندها زیاد شود،
    سطرهای بسیار زیادی را اشغال خواهد کرد. این امر موجب هدررفت PID های قابل دسترس می‌شود که درنهایت می‌تواند منجر به جلوگیری از ایجاد فرآیندهای جدید شود
    \item \textbf{عملکرد سیستمی:}
    سیستم‌عامل باید این فرآیند‌های غیرمفید را زیر نظر بگیرد. و همین اضافه کاری می‌تواند روی عمل‌کرد سیستم کامپیوتری تأثیر منفی داشته باشد.
    \item \textbf{نشت حافظه:}
    فرآیند‌های زامبی به طور مستقیم از حافظه‌ی اصلی استفاده نمی‌کنند، اما می‌تواند باعث بسته نشدن بعضی از فایل‌ها و بافر‌ها شود.
    این حافظه‌ی اضافی باعث هدررفت حافظه و در مواردی باعث نشت حافظه میشود.

\end{itemize}
\end{document}