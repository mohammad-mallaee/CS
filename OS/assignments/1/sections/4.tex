\documentclass[]{article}
\usepackage{hyperref}
\usepackage[a4paper, total={6in, 8in}]{geometry}
\usepackage{caption}
\usepackage{algpseudocodex}
\usepackage{algorithm}
\usepackage{amsmath}
\usepackage{listings}
\usepackage{graphicx}
\usepackage{setspace}
\usepackage{subfiles}
\usepackage{xcolor}
\usepackage{enumerate}
\usepackage{xepersian}
\settextfont{XB Niloofar}

\begin{document}
در ارتباط بین فرآیندی با استفاده از حافظه ی مشترک، نظر به محدود بودن حافظه در دنیای واقعی چه راهکارهایی برای نامحدود نگه
داشتن بافر به نظر شما می رسد؟ توضیح دهید.
\\
با توجه به محدود بودن فضای ذخیره سازی میتوانیم از
Buffer
حلقوی استفاده کنیم بدین صورت که مقدار ثابتی از حافظه را در نظر میگریم و تا پر شدن از آن
استفاده می‌کنیم و سپس از ابتدای آن شروع به نوشتن می‌کنیم انگار که ابتدا و انتهای آن به هم متصل اند.
\\
راه حل دیگر میتواند انتقال داده‌های کمتر استفاده شده
به حافظه جانبی در زمان پرشدنش باشد، هرچند که این روش به ما یک
Buffer
نامحدود نمیدهد اما میتواند بدون نیاز به حذف داده‌ها حجم زیادی از داده‌ها را مدیریت کند.
\end{document}