\documentclass[]{article}
\usepackage{hyperref}
\usepackage[a4paper, total={6in, 8in}]{geometry}
\usepackage{caption}
\usepackage{algpseudocodex}
\usepackage{algorithm}
\usepackage{amsmath}
\usepackage{listings}
\usepackage{graphicx}
\usepackage{setspace}
\usepackage{subfiles}
\usepackage{xcolor}
\usepackage{enumerate}
\usepackage{xepersian}
\settextfont{XB Niloofar}

\begin{document}
صحیح یا غلط بودن  گزاره‌های زیر را مشخص کنید و دلیل خود را نیز بیان کنید:
\vspace{0.2cm}
\begin{enumerate}[(A)]
    \item سیستم عامل تنها یک واسط بین کاربر و سیستم کامپیوتری است. \\
    این گزاره غلط است. سیستم‌های‌عامل تعاریف متعارفی ندارد و یک مفهوم کلی است. نسبت‌دادن یک نقش به چنین مفهوم جامعی نیز نمی‌تواند دقیق باشد. از لحاظ کاربردی هم سیستم‌های عامل نقش‌های زیادی را بازی می‌کند که یکی از آن‌ها می‌تواند بازی کردن نقش رابطه بین کاربر و سیستم کامپیوتری باشد.
    از نقش‌های دیگر آن می‌توان به موارد زیر اشاره کرد:
    \vspace{-0.15cm}
    \begin{itemize}
        \item مدیریت سیستم‌های کامپیوتری که یکی از ابعاد آن مدیریت‌ کردنِ بهنیه‌ی منابع سیستمی است.
        \item محیطی برای اجرای برنامه‌های کاربردیست
    \end{itemize}

    \item هسته ی اصلی سیستم عامل، کرنل است که کنترل روی برخی از اجزای سیستم را در حافظه بر عهده دارد. \\
    این گزاره نیز غلط است .با این فرض که سیستم عامل باید درستی سیستم را تضمین کند، این یک امر بدیهی است که  کرنل کنترل بر روی تمامیِ اجزای سیستمی را داشته باشد. فرض کنیم اینطور نباشد.
    آن‌گاه بخش‌هایی وجود دارند که زیر نظر سیستم‌عامل فعالیت نمی‌کنند و توسط آن مدیریت نمی شوند.
    در این حالت کرنل، به عنوان هسته‌ی سیستم‌عامل، چون کنترلی بر روی این اجزا ندارد، پس نمی‌تواند درستیِ کارکرد کل سیستم‌ کامپیوتری را نیز تضمین کند.

    \item دستورالعمل خواندن ساعت سیستم، فقط در مد کرنل اجرا می شود. \\
    این گزاره نیز غلط است. خواندن ساعت سیستم یک عملیات ممتاز نیست. در بیشتر سیستم‌های عامل، فرآیندها می‌توانند در حالتی به جز حالت کرنل(مثلاً حالت کاربر) به ساعت سیستم دسترسی داشته باشند.
    به اصطلاح این عملیات، یک عملیات غیرممتاز است. توجه شود که صرف خواندن ساعت سیستم، نباید با مدیریت و دستکاری در منبع ساعت سیستم(که امری حیاتی و حساس است) اشتباه گرفته شود.

    \item تنها تفاوت فرآیند با برنامه این است که فرآیند ماهیتی فعال دارد. \\
    این گزاره صحیح است. در تعریف فرآیند آمده است که: به برنامه‌ی در حال اجرا، فرآیند می‌گویند.
    از این تعریف اینگونه برداشت می‌شود که اساسی‌ترین تفاوت بین فرآیند‌ها و برنامه‌ها، ماهیتشان است. 
    برنامه‌ها ماهیتی انفعالی(passive) و حالتی ثابت(static) دارند. در حالی که فرآیند‌ها ماهیتی فعال(active) و حالتی منطعف(dynamic) دارند.
    این تفاوت اساسی شامل تفاوت‌های دیگر نیز بین فرآیند و برنامه می‌شود.
\end{enumerate}
\end{document}