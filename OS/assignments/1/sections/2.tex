\documentclass[]{article}
\usepackage{hyperref}
\usepackage[a4paper, total={6in, 8in}]{geometry}
\usepackage{caption}
\usepackage{algpseudocodex}
\usepackage{algorithm}
\usepackage{amsmath}
\usepackage{listings}
\usepackage{graphicx}
\usepackage{setspace}
\usepackage{subfiles}
\usepackage{xcolor}
\usepackage{enumerate}
\usepackage{xepersian}
\settextfont{XB Niloofar}

\begin{document}
به سوالات زیر پاسخ کامل دهید.
\vspace{0.2cm}
\begin{enumerate}[(A)]
    \item  تفاوت سیستم های چند پردازنده ای با سیستم های چندهسته ای چیست؟ \\
    اکثر سیستم‌های چندپردازنده‌ای از سیستم
    SMP
    استفاده می‌کنند بدین معنا که هر پردازنده
    register
    های خاص خودش و همینطور یک کش محلی دارد اما منابع سیستم را مانند
    BUS
   با دیگر پردازنده‌ها به اشتراک می‌گذارد.
   \\
   در سیستم‌های چند هسته‌ای،‌ تنها یک پردازنده وجود دارد که از چند هسته پردازشی تشکیل شده است.
   این سیستم‌ها می‌توانند از سیستم‌های چند پردازنده‌ای سریع‌تر باشند چراکه ارتباط درون تراشه‌ای
   سریع‌تر از ارتباط میان تراشه‌های مجزا است.
   از طرفی یک تراشه به مراتب انرژی کمتری مصرف میکند که موضوعی مهم برای دستگاه‌های موبایل و لپتاپ هااست.
   در این سیستم‌ها علاوه بر کش محلی یا
   $L_1$
   کش‌های بیشتر و در لایه‌های مختلفی وجود دارد که از لحاظ اندازه و سرعت نیز متفاوت اند.
   برای مثال در یک پردازنده‌ی دو هسته‌ای کش
   $L_2$
   یک حافظه‌ی مشترک بین این دو هسته خواهد بود. \\
   موضوع دیگر مقیاس پذیری‌است که در سیستم‌های چندپردازنده‌ای می‌تواند مشکل‌ساز باشد چراکه
   با افزودن پردازنده رقابت بر سر منابع و به طور خاص
   $BUS$ سیستم
   می‌تواند مشکلاتی را به وجود بیاورد و روی سرعت سیستم تاثیر گذار باشد که
   البته با تکنیک $NUMA$
   تا حد مناسبی قابل رفع است.

    
    \item PCB
    چیست و از چه قسمت هایی تشکیل شده است؟ \\
    بلاک کنترل فرآیند مهم‌ترین ساختمان داده در داخل هسته سیستم‌عامل است
    و شامل اطلاعات موردنیاز برای اجرای یک فرآیند است فرآیند‌های سیستم‌عامل به وسیله آن نشان داده میشوند.
    این بلاک شامل بخش‌های زیر است:
    \vspace{-0.1cm}
    \begin{itemize}
        \item \textbf{وضعیت فرآیند:}
        وضعیت فرآیند میتواند یکی از موارد جدید، آماده، درحال اجرا، انتظار . یا مسدود باشد.
        \item \textbf{شمارنده برنامه:}
        شمارنده نشان دهنده‌ی آدرس دستورالعمل بعدی از فرآیند است.
        \item \textbf{ثبات‌های :CPU}
        به همراه شمارنده برنامه، مقدار ثبات‌های
        CPU
        باید ذخیره شوند تا پس از رخدادن وقفه یا اتفاقات دیگر، فرآیند بتواند به درستی
        کار خود را ادامه دهد.
        \item \textbf{اطلاعات زمان‌بندی :CPU}
        اطلاعاتی از قبیل اولویت فرآیند، اشاره‌گر به صف زمان‌بندی و پارامتر‌های مربوطه دیگر.
        \item \textbf{اطلاعات مدیریت حافظه:}
        حاوی اطلاعاتی مانند مقدار ثبات‌های پایه، جداول صفحه و...
        \item \textbf{اطلاعات حسابرسی:}
        حاوی اطلاعات میزان استفاده از پردازنده، محدودیت‌های زمانی و شماره فرآیند و غیره.
        \item \textbf{اطلاعات وضعیت :I/O}
        حاوی اطلاعات دستگاه‌های ورودی/خروجی تخصیص داده شده به فرآیند و لیست فایل‌های باز و ...
    \end{itemize}

    \item عمل تعویض متن با چه هدفی و توسط چه بخشی انجام می شود؟ \\
    به هنگام ایجاد وقفه در یک سیستم کامپیوتری، سیستم عامل وظیفۀ جاری
    CPU
    را متوقف و یک روتین کرنل را اجرا می کند.
    چنین عملیاتی به طور مکرر در سیستم های همه منظوره اتفاق می افتد.
    هنگامی که یک وقفه در سیستم حادث می شود، سیستم عامل نیاز دارد تا محتوای فرآیند جاری در
    CPU
    را به نحوی ذخیره کند که در آینده بتواند برای از سرگیری اجرای فرآیند مذکور استفاده نماید.
    
    \item چرا مدیریت ورودی/خروجی در سیستم عامل پیچیده است؟ \\
    به علت نیازمندی‌های متفاوت و متنوع سیستم، دستگاه‌های I/O گستردگی زیاد دارند.
    علاوه بر این موضوع تاثیر مستقیم آنها بر قابلیت اطمینان و کارایی یک سیستم
    باعث پیچیده شدن مدیریت ورودی/خروجی در سیستم‌عامل میشود.

    \item به کارگیری ویژگی
    DMA
    در سیستم های کامپیوتری برای پاسخ به کدام چالش موجود است و چگونه کار می کند؟ \\
    اگر دستگاه‌های ورودی خروجی بخواهند داده‌ها را از طریق
    CPU
    انتقال دهند با محدودیت‌های ظرفیت
    CPU
    مواجه میشوند که باعث کند شدن روند انتقال و همینطور سرباری برای سیستم میشود
    به همین دلیل از تکنیک 
    DMA
    استفاده می‌کنیم که
    روشی برای انتقال داده از حافظه اصلی به دیگر بخش‌های سیستم کامپیوتری بدون درگیر کردن
    CPU است.
    عملکرد آن بدین صورت است که
    DC
    پس از انجام تنظيمات برای بافرها، اشاره‌گرها و شمارنده‌ها برای دستگاه
    I/O
    موردنظر، يک بلوک کامل از اطلاعات را از/به دستگاه و حافظه اصلی بدون دخالت
    CPU
    انتقال می دهد. بنابراين تنها يک وقفه برای کل بلوک توليد می شود.
\end{enumerate}
\end{document}