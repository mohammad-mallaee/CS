\documentclass[]{article}
\usepackage{hyperref}
\usepackage[a4paper, total={6in, 8in}]{geometry}
\usepackage{caption}
\usepackage{algpseudocodex}
\usepackage{algorithm}
\usepackage{amsmath}
\usepackage{listings}
\usepackage{xepersian}
\usepackage{graphicx}
\usepackage{setspace}
\usepackage{subfiles}
\usepackage{xcolor}
\settextfont{XB Niloofar}

\begin{document}
فرض کنید که ساختار یک شبکه تلفن را بتوان به شکل گراف $G = <V, E>$ تصور کرد 
که هر رأس آن، نشان یک مرکز هادی و هر یال آن، نشان یک خط ارتباطی موجود بین دو مرکز هادی باشد؛ و هر یال گراف با پهنای باند خط ارتباطی متناظر با آن
(که حداکثر سرعتی است برحسب بیت بر ثانیه، که دادهها را می‌توان در امتداد آن خط ارتباطی انتقال داد) برچسب خورده باشد.
پهنای باند هر مسیری در گراف $G$ را پهنای باند یالی از آن مسیر که کمترین پهنای باند را داشته باشد، تعریف میکنیم.

\paragraph*{الف}
الگوریتمی کارا برای این مسأله طراحی کنید که با آن بتوان از روی ساختار گرافی یک شبکه تلفن، حداکثر پهنای باند مسیرهای بین هر دو مرکز هادی را محاسبه کرد و به شکل یک ماتریس نمایش داد. 

\paragraph*{ب}
بر مبنای الگوریتم، برنامه‌ای برای حل رایانهای مسأله بنویسید. برای آنکه بتوانید درستی برنامه خود را به طور دستی بیازمایید،
سه گراف وزن‌دار خلوت 20 رأسی تولید کنید و برنامه خود را روی آن ورودیها اجرا کنید. نهایتاً آن سه گراف را به طور دستی نیز بکشید و سه ماتریس خروجی‌ برنامه را نیز در کنار آنها بگذارید. 
\end{document}