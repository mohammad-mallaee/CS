\documentclass[]{article}
\usepackage{hyperref}
\usepackage[a4paper, total={6in, 8in}]{geometry}
\usepackage{caption}
\usepackage{algpseudocodex}
\usepackage{algorithm}
\usepackage{amsmath}
\usepackage{listings}
\usepackage{xepersian}
\usepackage{graphicx}
\usepackage{setspace}
\usepackage{subfiles}
\usepackage{xcolor}
\settextfont{XB Niloofar}

\begin{document}
\paragraph[1.1]{الف)}
برنامهای برای پیادهسازی دو الگوریتم ساده‌اندیشانه (تعریف - مبنا) و الگوریتم کاراتسوبا برای محاسبه حاصل‌ضرب دو عدد صحیح  رقمی بنویسید.

\paragraph[1.2]{ب)}
الگوریتم کاراتسوبا از نظر مجانبی، کاراتر از الگوریتم ساده‌اندیشانه (تعریف - مبنا) است،
اما در عمل، طبق انتظار تا نقطه‌ای ( مقداری از $n$ ) اجرای الگوریتم ساده‌اندیشانه سریعتر از اجرای الگوریتم کاراتسوبا است و از آن نقطه به بعد،
اجرای الگوریتم کاراتسوبا سریعتر از اجرای الگوریتم ساده‌اندیشانه است.
با انجام آزمایشی، «نقطه عبور» از اجرای سریعتر الگوریتم سادهاندیشانه به اجرای سریعتر الگوریتم کاراتسوبا را تعیین کنید. 

\end{document}