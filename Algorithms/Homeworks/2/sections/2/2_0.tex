\documentclass[]{article}
\usepackage{hyperref}
\usepackage[a4paper, total={6in, 8in}]{geometry}
\usepackage{caption}
\usepackage{algpseudocodex}
\usepackage{algorithm}
\usepackage{amsmath}
\usepackage{listings}
\usepackage{xepersian}
\usepackage{graphicx}
\usepackage{setspace}
\usepackage{subfiles}
\usepackage{xcolor}
\settextfont{XB Niloofar}

\begin{document}
رستم (بازیکن 1) و اسفندیار (بازیکن 2) قرار است به یک بازی راهبردی دو نفره با این قوانین بپردازند:  عددی زوج است و  سکه با ارزشهای یکسان یا متفاوت در یک ردیف گذاشته شدهاند
هر دو بازیکن، به نوبت و هر بار یکی از سکهها را از یکی از دو انتهای (راست یا چپ) ردیف باقیمانده سکهها برمیدارند.
بازیکنی که مجموع ارزش سکههایی که برداشته باشد بیشتر باشد، در نهایت برنده بازی خواهد بود.
بازی را از چشمانداز بازیکن 1 ببینید و فرض کنید راهبرد هر دو بازیکن در انتخاب سکه بهینه باشد؛ 
بهینه به این معنا که رستم در هر حرکت خود، سکهای را انتخاب ‌می‌کند تا در نهایت بیشترین مبلغ ممکن از سکهها نصیب او شود؛
و اسفندیار در هر حرکت خود، سکهای را انتخاب ‌می‌کند تا در نهایت کمترین مبلغ ممکن از سکهها نصیب رستم شود.

\paragraph*{الف}
الگوریتمی برای تعیین راهبرد بهینه رستم (بازیکن 1) برای بازی ارائه کنید.

\paragraph*{ب}
اجرای گامبهگام الگوریتم را با در نظر گرفتن این ردیف سکهها نشان دهید:
$$2, 5, 8, 6, 9, 2, 10, 5, 7, 4$$
\end{document}