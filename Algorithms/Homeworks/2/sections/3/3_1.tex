\documentclass[]{article}
\usepackage{hyperref}
\usepackage[a4paper, total={6in, 8in}]{geometry}
\usepackage{caption}
\usepackage{algpseudocodex}
\usepackage{algorithm}
\usepackage{amsmath}
\usepackage{listings}
\usepackage{xepersian}
\usepackage{graphicx}
\usepackage{setspace}
\usepackage{subfiles}
\usepackage{xcolor}
\settextfont{XB Niloofar}

\begin{document}
برای حل این مسئله می‌توانیم از الگوریتم‌هایی استفاده کنیم که درخت پوشای کمینه متناظر
با گراف را که در اینجا یک گراف کامل است پیدا می‌کنند.
روند حل سوال بدین صورت است که یالی را که بیشترین وزن را دارد (با فرض اینکه وزن یال نشان دهنده فاصله بین دو راس است)
از درخت حذف می‌کنیم و جنگل به‌دست آمده را که متشکل از دو درخت است به عنوان افرازی از این نقاط
که شرایط مسئله را دارد معرفی می‌کنیم.
به تعبیری دیگری برای اینکه این n نقطه را به k مجموعه افراز کنیم،
کافیست در زمان اجرای الگوریتم $kruskal$ و زمانی که تعداد درخت‌های همبند به $k$ رسید،
اجرای الگوریتم را متوقف کنیم زیرا مولفه‌های همبندی ساخته شده توسط این الگوریتم شرط مسئله را دارند.
شبه‌کد الگوریتم بصورت زیر خواهد بود :

\begin{latin}
    \begin{algorithm}[H]
        \caption*{2-Partition($S = \lbrace s_0, s_1, \ldots, s_n \rbrace $)}
        \begin{algorithmic}
            \Require A non-emtpy set S of n points
            \Ensure 2 clusters of points in A
            \State Set points and their distances as graph G = $<V, E>$
            \State Create the minimum spanning tree of $G$ as T = $<V', E'>$ 
            \State Remove the most weighted edge of T
            \State Put the vertices of this two tree into sets A and B
            \State \Return A, B
        \end{algorithmic}
    \end{algorithm}
\end{latin}
اگر درخت پوشای کمینه را با الگوریتم $kruskal$ محاسبه کنیم،
زمان اجرای الگوریتم $O(|E|log(|E|))=O(n^2log(n^2))$ خواهد بود.
\end{document}