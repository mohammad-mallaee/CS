\documentclass[]{article}
\usepackage{hyperref}
\usepackage[a4paper, total={6in, 8in}]{geometry}
\usepackage{caption}
\usepackage{algpseudocodex}
\usepackage{algorithm}
\usepackage{amsmath}
\usepackage{listings}
\usepackage{xepersian}
\usepackage{graphicx}
\usepackage{setspace}
\usepackage{subfiles}
\usepackage{xcolor}
\settextfont{XB Niloofar}

\begin{document}
فرض کنید که $n$ رایانه در یک شبکه محلی سیمی، به شکل درخت ریشهدار $T$ چیده شده باشند؛
یعنی گره‌های درخت $T$ ، نشان رایانههای شبکه هستند و یالهای درخت زوج رایانههایی را مشخص میکنند که مستقیماً با کابل به هم وصل شدهاند.
ریشه درخت $T$ ، نشان رایانهای است که باید به اینترنت وصل باشد. مدیر شبکه دغدغه ارتباطات امن در شبکه را دارد و میخواهد با خرید تعدادی نرمافزار ناظر، پیوسته کیفیت ارتباطات در شبکه را بررسی کند.
اگر یک نرمافزار ناظر روی رایانه $x$ نصب شود، کاربر $x$ میتواند همه ارتباطات مستقیم خود با دیگر رایانهها را نظارت کند.
مدیر برای صرفهجویی بیشتر در هزینههای خرید و ارتقای نرم‌افزار، از شما خواسته است که کمترین تعداد نرم‌افزار لازم را برای آنکه حداقل یک نرمافزار ناظر بر هر خط ارتباطی در شبکه نظارت کند، تعیین کنید.
برای مثال، اگر هر رایانه در $T$ (جز ریشه) فرزند ریشه باشد، پس مدیر فقط به یک ناظر نیاز خواهد داشت که باید روی رایانه ریشه نصب شود.

\paragraph*{الف}
الگوریتمی کارا برای مسأله مدیر شبکه طراحی کنید که با آن بتوان از روی ساختار درختی یک شبکه رایانهای، کمترین تعداد نرم‌افزار لازم را محاسبه کرد و رایانههایی را که نرمافزار باید روی آنها نصب شود، تعیین کرد

\paragraph*{ب}
بر مبنای الگوریتم، برنامه‌ای برای حل رایانهای مسأله بنویسید. برای آنکه بتوانید درستی برنامه خود را به طور دستی بیازمایید،
سه درخت 20 گرهای تولید کنید و برنامه خود را روی آن ورودیها اجرا کنید. نهایتاً آن سه درخت را به طور دستی نیز بکشید و گرههای جواب را علامت بزنید.
\end{document}