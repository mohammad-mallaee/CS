\documentclass[]{article}
\usepackage{hyperref}
\usepackage[a4paper, total={6in, 8in}]{geometry}
\usepackage{caption}
\usepackage{algpseudocodex}
\usepackage{algorithm}
\usepackage{amsmath}
\usepackage{listings}
\usepackage{xepersian}
\usepackage{graphicx}
\usepackage{setspace}
\usepackage{subfiles}
\usepackage{xcolor}
\settextfont{XB Niloofar}

\begin{document}
مساله کوله پشتی را درنظر بگیرید :
n عنصر با وزن های معلوم
$w_1, w_2, w_3,... w_n$
و ارزش های معلوم $v_1, v_2, v_3,... v_n$
و یک کوله پشتی با ظرفیت ،W داده شده اند؛
با ارزش‌ترین زیرمجموعه‌ای از این عناصر را پیدا کنید که بتوان آن‌ها را درون کوله پشتی جا داد.
\paragraph[1.1]{الف)}
با تحلیلی ریاضی، کارایی زمانی الگوریتم جستجوی کامل برای مسأله کوله‌پشتی را با نماد مجانبی $\Theta$ بیان کنید.

\paragraph[1.2]{ب)}
برنامه‌ای برای پیاده‌سازی الگوریتم جستجوی کامل بنویسید. و بزرگترین مقداری از n را که به ازای آن، رایانه در کمتر از 1 دقیقه قادر به اجرای برنامه باشد، پیدا کنید. 

\end{document}