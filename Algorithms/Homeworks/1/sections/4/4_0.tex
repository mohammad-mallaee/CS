\documentclass[]{article}
\usepackage{hyperref}
\usepackage[a4paper, total={6in, 8in}]{geometry}
\usepackage{caption}
\usepackage{algpseudocodex}
\usepackage{algorithm}
\usepackage{amsmath}
\usepackage{listings}
\usepackage{xepersian}
\usepackage{graphicx}
\usepackage{setspace}
\usepackage{subfiles}
\usepackage{xcolor}
\settextfont{XB Niloofar}

\begin{document}
\paragraph[3.1]{الف)}
با تحلیل ریاضی، مشخص شده است که اگر اندازه یک جدول درهم‌سازی بسته n باشد و اگر تعداد کلیدها m باشد
و اگر $\alpha = \frac{n}{m}$ باشد،
آنگاه میانگین تعداد مقایسه‌های لازم برای جستجوهای ناموفق (یا درجها) در چنین جدولی با راهبرد کاوش خطی، تقریباً
$\displaystyle \frac{1}{2}(1 + \frac{1}{(1-\alpha)^2})$ خواهد بود.

\paragraph[3.2]{}
برنامه‌ای بنویسید که با آن بتوان یک جدول درهم‌سازی بسته با اندازه n ساخت و سپس با راهبرد کاوش خطی
$\frac{n}{2}$ عدد صحیح (کلید) تصادفی را در جدول درج کرد. سپس شمارنده‌ای (یا شمارنده‌هایی) را در برنامه درج کنید
و با آن، میانگین تعداد مقایسه‌هایی را که برای جستجوهای ناموفق (درج‌ها) در جدول لازم است محاسبه کنید.
آزمایش را با هر یک از مقادیر $n = 10, 10^2, 10^3, 10^4, 10^5, 10^6$ انجام دهید و نتایج تجربی را با نتایج نظری مقایسه کنید. 

\paragraph[3.1]{ب)}
این فرضیه مطرح شده است که تعداد مقایسه‌های لازم برای درج m کلید تصادفی با راهبرد کاوش خطی در یک جدول
درهم‌سازی بسته با اندازه m ، تقریباً برابر با
$\displaystyle \sqrt{\frac{\pi}{2}}(m^{\frac{3}{2}})$ است.
با نوشتن برنامه‌ای، درستی این ادعا را تحقیق کنید.
\end{document}