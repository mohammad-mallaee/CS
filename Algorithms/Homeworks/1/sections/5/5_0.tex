\documentclass[]{article}
\usepackage{hyperref}
\usepackage[a4paper, total={6in, 8in}]{geometry}
\usepackage{caption}
\usepackage{algpseudocodex}
\usepackage{algorithm}
\usepackage{amsmath}
\usepackage{listings}
\usepackage{xepersian}
\usepackage{graphicx}
\usepackage{setspace}
\usepackage{subfiles}
\usepackage{xcolor}
\settextfont{XB Niloofar}

\begin{document}
فرض کنید A مجموعه‌ای بزرگ از n عنصر باشد؛
آنقدر بزرگ که نتوان آن را به طور کامل در حافظه اصلی نگهداری کرد و لازم باشد که آن را روی دیسک نگهداری کرد.
یک راه برای مرتب کردن چنین مجموعه‌ای، طراحی گونه‌ای از الگوریتم مرتبسازی ادغامی است:
مجموعه‌ A را به k مجموعه کوچک‌تر $A_1, A_2,\dots, A_k$ تقسیم کنید،
هر یک از آن مجموعه‌ها را به طور بازگشتی مرتب کنید و سپس همه مجموعه‌های مرتب را با هم ادغام کنید تا مجموعه مرتب اصلی تشکیل شود.



\paragraph[3.2]{الف)}
فرض کنید اندازه هر صفحه دیسک b باشد و اندازه حافظه اصلی m باشد.
با یک مثال توضیح دهید که مقدار k
( برحسب $b$ و $m$ )
چند باشد تا بتوان با
$O(\frac{n}{b})$ عملیات انتقال صفحه
(خواندن یک صفحه از دیسک یا نوشتن یک صفحه روی دیسک) ، k مجموعه مرتب را با هم ادغام کرد.

\paragraph[3.1]{ب)}
با تشکیل و حل یک رابطه بازگشتی برای تعداد عملیات انتقال صفحه، ثابت کنید که الگوریتم مرتبسازی ادغامی مجموعه A را با
$\displaystyle O((\frac{n}{b})\frac{\log(\frac{n}{b})}{\log(\frac{m}{b})})$
عملیات انتقال صفحه مرتب میکند.
\end{document}