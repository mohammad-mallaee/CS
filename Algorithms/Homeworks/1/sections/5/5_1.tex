\documentclass[]{article}
\usepackage{hyperref}
\usepackage[a4paper, total={6in, 8in}]{geometry}
\usepackage{caption}
\usepackage{algpseudocodex}
\usepackage{algorithm}
\usepackage{amsmath}
\usepackage{listings}
\usepackage{xepersian}
\usepackage{graphicx}
\usepackage{setspace}
\usepackage{subfiles}
\usepackage{xcolor}
\settextfont{XB Niloofar}

\begin{document}
پس از مرتب کردن همه‌ی زیر آرایه‌های آرایه $A$
برای ادغام آن ها و ساخت زیرآرایه بزرگتر می‌توانیم از الگوریتم $k\_way\_merge$
استفاده کنیم در این الگوریتم هر صفحه اشغال شده توسط این آرایه را یکبار برای خواندن از
دیسک می‌گیریم و برای به اندازه‌ی تعداد صفحات اشغال شده نیز برای نوشتن به دیسک می‌فرستیم
بنابراین در هر مرحله ادغام $\displaystyle \frac{2n}{b}$ عملیات انتقال صفحه خواهیم داشت.
برای اینکه با $O(\frac{n}{b})$ عملیات انتقال صفحه این آرایه را مرتب کنیم، 
باید تعداد مراحل اندازه‌ای ثابت و یا در حالت ایده‌ال یک مرحله باشند.
با در نظر گرفتن این موضوع که اندازه هر زیرآرایه باید از اندازه حافظه اصلی کوچکتر باشد،
بنابراین $\displaystyle k > \frac{n}{m}$ خواهد بود.
از طرفی برای اینکه بتوانیم $k$ زیرآرایه را در یک مرحله ادغام کنیم، باید بتوانیم حداقل
یک صفحه از هر یک آز آن‌ها را در هر لحظه در حافظه اصلی داشته باشیم و این به این معناست که
$\displaystyle k < \frac{m}{b}$. می‌توانیم مقدار $k$ را $\displaystyle \frac{m}{b} - 1$
در نظر بگیریم تا حداکثر تعداد زیرآرایه را در هر مرحله باهم ادغام کنیم.

\end{document}