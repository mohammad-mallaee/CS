\documentclass[]{article}
\usepackage{hyperref}
\usepackage[a4paper, total={6in, 8in}]{geometry}
\usepackage{caption}
\usepackage{algpseudocodex}
\usepackage{algorithm}
\usepackage{amsmath}
\usepackage{listings}
\usepackage{xepersian}
\usepackage{graphicx}
\usepackage{setspace}
\usepackage{subfiles}
\usepackage{xcolor}
\settextfont{XB Niloofar}

\begin{document}
اگر تعداد صفحاتی که در هرلحظه می‌توانیم برای ادغام کردن در حافظه اصلی داشته باشیم،
$B$ باشد، از آنجایی که در هر مرحله مسئله به $B$ قسمت تقسیم می‌شود که
با $\displaystyle \frac{2n}{b}$ عملیات انتقال صفحه ادغام می‌شوند، رابطه بازگشتی به صورت زیر خواهد بود:
$$T(n) = BT(n/B) + 2n/b$$
بااستفاده از قضیه اصلی و درخت فراخوانی های بازگشتی این مسئله داریم:
\begin{equation*}
\begin{split}
    T(n) \in O((\frac{n}{b}) \log_B(\frac{n}{b})) \\
    \Rightarrow T(n) \in O(\displaystyle (\frac{n}{b}) \frac{\log(\frac{n}{b})}{\log(B)})
\end{split}
\end{equation*}
اگر بخواهیم کارایی الگوریتم حداکثر شود $B$ را همان $\displaystyle \frac{m}{b} - 1$
در نظر می‌گیریم و از آنجایی که $B \in O(m/b)$ داریم:
$$T(n) \in O(\frac{n}{b} \frac{\log(\frac{n}{b})}{\log(\frac{m}{b})})$$
\end{document}