\documentclass[]{article}
\usepackage{hyperref}
\usepackage[a4paper, total={6in, 8in}]{geometry}
\usepackage{caption}
\usepackage{algpseudocodex}
\usepackage{algorithm}
\usepackage{amsmath}
\usepackage{listings}
\usepackage{xepersian}
\usepackage{graphicx}
\usepackage{setspace}
\usepackage{subfiles}
\usepackage{xcolor}
\settextfont{XB Niloofar}

\begin{document}
با محاسبه $T(n)$ برای $n$ های کوچک، میتوانیم این رابطه بازگشتی را به صورت
$T(n) = (n+1)c + nd$ حدس بزنیم:
\begin{equation}
    T(0) = c
\end{equation}
\begin{equation}
\begin{split}
    T(1) = T(k) + T(k) + d \rightarrow k=0 \\
    T(1) = T(0) + T(0) + d = 2c + d
\end{split}
\end{equation} 
\begin{equation}
\begin{split}
    T(2) = T(k) + T(1 - k) + d \rightarrow k=0,1 \\
    k=0: T(2) = T(0) + T(1) + d = 3c + 2d \\
    k=1: T(2) = T(1) + T(0) + d = 3c + 2d
\end{split}
\end{equation} 
\begin{equation}
\begin{split}
    T(3) = T(k) + T(2 - k) + d \rightarrow k=0,1,2 \\
    k=0: T(3) = T(0) + T(2) + d = 4c + 3d \\
    k=1: T(3) = T(1) + T(1) + d = 4c + 3d \\
    k=1: T(3) = T(2) + T(0) + d = 4c + 3d
\end{split}
\end{equation}
از آنجایی که این رابطه برای $T(1)$ درست است آن‌را حالت پایه در نظر می‌گیریم
و فرض می‌کنیم این رابطه برای $n-1$ رابطه قبلی درست باشد آنگاه نشان می‌دهیم که برای $n$ نیز درست است:
\begin{align}
    T(n) = T(k) + T(n - 1 - k) + d ; \hspace{0.25cm} k=0,1, \dots, n-1 \\
    k=i < n: T(n) = T(i) + T(n - 1 - i) + d
\end{align}
از آنجا که طبق فرض این رابطه برای اعداد کوچکتر از $n$ درست است، داریم:
\begin{align}
    T(i) = (i+1)c + id \\
    T(n-1-i) = (n-1-i+1)c + (n-1-i)d \\
    T(i) + T(n-1-i) = ((n-1-i+1)+(i+1))c + ((n-1-i) + i)d \\
    T(i) + T(n-1-i) = (n+1)c + (n-1)d \\
    \Rightarrow T(n) = T(i) + T(n-1-i) + d = (n+1)c + nd
\end{align}

بنابراین حکم ثابت شد و از آنجایی که $c$ و $d$ اعداد ثابت اند، داریم:
$$T(n) = (n+1)c + nd = (c+d)n + c \Rightarrow T(n) \in \theta(n)$$

\end{document}