\documentclass[]{article}
\usepackage{hyperref}
\usepackage[a4paper, total={6in, 8in}]{geometry}
\usepackage{caption}
\usepackage{algpseudocodex}
\usepackage{algorithm}
\usepackage{amsmath}
\usepackage{listings}
\usepackage{xepersian}
\usepackage{graphicx}
\usepackage{setspace}
\usepackage{subfiles}
\usepackage{xcolor}
\settextfont{XB Niloofar}

\begin{document}
می‌توانیم با استفاده از درخت فراخوانی‌های بازگشتی، مرتبه رشد این تابع بازگشتی را مشخص کنیم.
در ابتدا این مسئله به دوقسمت نامساوی با اندازه‌های یک سوم و دو سوم مسئله اصلی تقسیم شده‌اند.
دو مرحله از این تابع بازگشتی به صورت زیر خواهد بود:
\begin{align}
T(n) = T(\frac{n}{3}) + T(\frac{2n}{3}) + n \\
= T(\frac{n}{9}) + T(\frac{2n}{9}) + \frac{n}{3} + T(\frac{2n}{9}) + T(\frac{4n}{9}) + \frac{n}{3} + n
\end{align}
همانطور که می‌بینیم در هر مرحله به اندازه تابع، $n$ واحد اضافه می‌شود
و این کار تا زمانی که تقسیم های ما به پایان برسد ادامه می‌یابد.
از آنجایی که در این مسئله، تقسیم به زیر مسئله‌ها به صورت نامتوازن است، طبیعتا درخت متناظر آن نیز همینطور خواهد بود.
اضافه شدن $n$ تا زمانی که به انتهای کوچکترین مسیر برسیم ادامه خواهد یافت
و این امر هنگامی که مسئله را به یک سوم مسئله اصلی تقسیم می‌کنیم اتفاق می‌افتد
و انجام این کار $log_3(n)$ بار اتفاق می‌افتد.
بنابراین اندازه تابع حداقل $n log_3(n)$ خواهد بود و پس از آن
مقادیری کمتر از آن به تابع اضافه می‌شوند بنابراین داریم:
$$T(n) \in \Omega(nlog_3(n))$$
\end{document}