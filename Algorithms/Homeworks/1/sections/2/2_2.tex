\documentclass[]{article}
\usepackage{hyperref}
\usepackage[a4paper, total={6in, 8in}]{geometry}
\usepackage{caption}
\usepackage{algpseudocodex}
\usepackage{algorithm}
\usepackage{amsmath}
\usepackage{listings}
\usepackage{xepersian}
\usepackage{graphicx}
\usepackage{setspace}
\usepackage{subfiles}
\usepackage{xcolor}
\settextfont{XB Niloofar}

\begin{document}
با محاسبه $T(n)$ برای $n$ های کوچک
در می‌یابیم که $T(n) = n + (n-1) + (n-2) + \dots + 2$.
برای اثبات این موضوع از استقرای ریاضی استفاده می‌کنیم:

$$T(2) = 2 + max \lbrace T(1) + T(1)\rbrace = 2$$

حالت پایه این تساوی برقرار است بنابراین فرض می‌کنیم این تساوی برای $k-1$
برقرار است آنگاه نشان می‌دهیم این تساوی برای $k$ نیز برقرار است:
\begin{equation}
\begin{split}
T(k) & = k + max \lbrace T(i) + T(j) : i + j = k, i,j > 0 \rbrace
\\ & = k + max \lbrace T(1) + T(k-1), T(2) + T(k-2), \dots, T(k-1) + T(1) \rbrace
\end{split}
\end{equation}

از آنجایی که $T(1) = 0$
کافی است نشان دهیم :
$$T(k-1) > T(i) + T(j) ; \hspace{0.15cm} i+j = k \And i,j < k-1$$
\begin{equation}
\begin{split}        
    T(k-1) = (k-1) + (k-2) + \dots + 2 \\
    T(i)= i + (i-1) + (i-2) + \dots + 2 \\
    T(j) = j + (j-1) + (j-2) + \dots + 2
\end{split}
\end{equation}

\begin{equation}
    T(i) + T(j) = 2 \times ( 2 + 3 + \dots + min(i, j)) + min(i,j) + 1 + \dots + max(i,j)
\end{equation}
\begin{equation}
    T(k-1) - \lbrack T(i) + T(j) \rbrack = (max(i,j) + 1) + \dots + (k - 1)- (2 + 3 + \dots + min(i, j))
\end{equation}
با توجه به اینکه $i + j = k$
بنابراین تعداد اعداد مثبت و منفی در این معادله برابر است و از آنجایی که هر عدد مثبت از هر
عدد منفی بزرگتر است حکم ثابت شد و برای هر $n$ بزرگتر از $1$
داریم:
\begin{equation}
    \begin{split}
        T(n) = n + T(n-1) = n + (n-1) + (n-2) + \dots + 2 = \frac{(n-1)(n+2)}{2}
        \\ \rightarrow T(n) \in \theta(n^2)
    \end{split}
\end{equation}
\end{document}