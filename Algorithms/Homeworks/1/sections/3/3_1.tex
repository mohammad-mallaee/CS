\documentclass[]{article}
\usepackage{hyperref}
\usepackage[a4paper, total={6in, 8in}]{geometry}
\usepackage{caption}
\usepackage{algpseudocodex}
\usepackage{algorithm}
\usepackage{amsmath}
\usepackage{listings}
\usepackage{xepersian}
\usepackage{graphicx}
\usepackage{setspace}
\usepackage{subfiles}
\usepackage{xcolor}
\settextfont{XB Niloofar}

\begin{document}
از آنجایی که اعضای $A$ در محدوده $1$ تا $2n$ هستند،
میتوان آرایه‌ای به انداز $2n$ ایجاد کرد
و اعداد را براساس اندازه در این آرایه دخیره کرد. بدین صورت که اگر عنصر $i$ ام
این آرایه $1$ باشد آنگاه این عدد در آرایه $A$ وجود دارد.
پس از ذخیره کردن $A$ در این آرایه با بررسی اعضای $B$
یکسان بودن یا نبودن این آرایه‌ها را بررسی می‌کنیم.

\begin{latin}
\begin{lstlisting}[language=python]
def are_equal(A: list[int], B: list[int]):
    n = len(A)
    helper_list = [0] * (2 * n + 1)
    for number in A:
        helper_list[number] = 1
    for number in B:
        if helper_list[number] == 0:
            return False
    return True
\end{lstlisting}
\end{latin}
در این تابع پس از ذخیره اطلاعات در \texttt{helper\_list}
اگر عددی در B وجود داشته باشد که مقدار آن در \texttt{helper\_list}
$1$ نباشد آنگاه این دو آرایه یکسان نیستند و اگر چنین عددی وجود نداشت باهم برابرند.

اگر عملیات پایه‌ای را انتصاب در نظر بگیریم، آنگاه تعداد عملیات های پایه‌ای تنها به اندازه $A$
وابسته خواهد بود و برابر اندازه $A$ خواهد بود.
اگر عملیات پایه‌ای را مقایسه در نظر بگیریم آنگاه تنها به اندازه $B$ وابسته خواهد بود و
در بدترین حالت برابر با اندازه $B$ خواهد بود اما اگر $A$ و $B$
به صورت تصادفی انتخاب شده باشند به صورت میانگین بین ۱ تا ۳ مقایسه برای نشان دادن اینکه
یکسان نیستند کافی‌است زیرا احتمال برابر بود آنها بسیار کم است و اگر عملیات پایه‌ای را
دسترسی به عنصری در هریک از آرایه‌ها در نظر بگیریم، آنگاه تابع نشان‌دهنده تعداد عملیات های پایه‌ای به صورت
$T(n) = n + k$
به طوری که k، تعداد مقایسه‌های ما خواهد بود که دسترسی به عناصر $B$ را نشان می‌دهد:
در زیر نمونه‌ای از خروجی برنامه برای مجموعه‌های تصادفی $A$ و $B$ آورده شده است:
\pagebreak
\begin{latin}
\begin{lstlisting}
    length of A and B is 10^1
    A and B are not equal
    a_access: 10, b_access: 2, assigns: 10
    helper_access: 2, comparisons: 2
    
    length of A and B is 10^2
    A and B are not equal
    a_access: 100, b_access: 1, assigns: 100
    helper_access: 1, comparisons: 1
    
    length of A and B is 10^3
    A and B are not equal
    a_access: 1000, b_access: 2, assigns: 1000
    helper_access: 2, comparisons: 2
    
    length of A and B is 10^4
    A and B are not equal
    a_access: 10000, b_access: 2, assigns: 10000
    helper_access: 2, comparisons: 2
    
    length of A and B is 10^5
    A and B are not equal
    a_access: 100000, b_access: 1, assigns: 100000
    helper_access: 1, comparisons: 1
    
    length of A and B is 10^6
    A and B are not equal
    a_access: 1000000, b_access: 1, assigns: 1000000
    helper_access: 1, comparisons: 1
\end{lstlisting}
\end{latin}
\end{document}