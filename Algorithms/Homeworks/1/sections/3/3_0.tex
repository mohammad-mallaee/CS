\documentclass[]{article}
\usepackage{hyperref}
\usepackage[a4paper, total={6in, 8in}]{geometry}
\usepackage{caption}
\usepackage{algpseudocodex}
\usepackage{algorithm}
\usepackage{amsmath}
\usepackage{listings}
\usepackage{xepersian}
\usepackage{graphicx}
\usepackage{setspace}
\usepackage{subfiles}
\usepackage{xcolor}
\settextfont{XB Niloofar}

\begin{document}
\paragraph[3.1]{الف)}
فرض کنید A و B ، دو مجموعه ای باشند هر یک متشکل از $n$ عدد صحیح که هر یک از آنها در محدوده $1$ تا $2n$ واقع باشد.
الگوریتمی را با کارایی زمانی O(n) توصیف کنید که که دو مجموعه A و B را به عنوان ورودی بگیرد و تعیین کند که آیا دو مجموعه A و B با هم برابر هستند یا خیر؛
یعنی آیا شامل عناصر کاملاً یکسانی (گرچه با ترتیب متفاوت) هستند یا خیر.

برنامه‌ای برای پیاده‌سازی الگوریتم بنویسید و با درج شمارنده‌ای (یا شمارنده‌هایی) در آن، تعداد عملیات پایه‌ای الگوریتم را به عنوان تابعی از اندازه دو مجموعه A و B محاسبه کنید. با هر یک از مقادیر 
$n=10, 10^2, 10^3, 10^4, 10^5, 10^6$
آزمایش کنید. 
\paragraph[3.2]{ب)}
فرض کنید A و B ، دو مجموعه‌ای باشند هر یک متشکل از
$n$ عدد صحیح  که هر یک از آنها در محدوده‌ی
$1$ تا $n^4$ واقع باشد.
الگوریتمی را با کارایی زمانی $O(n)$ توصیف کنید 
که دو مجموعه A و B و عدد صحیح $x$ را به عنوان ورودی بگیرد و تعیین کند که آیا عدد صحیح a در مجموعه A و عدد صحیح b در مجموعه B وجود دارند که
که رابطه $a+b=x$ برقرار باشد یا خیر.

برنامه‌ای برای پیاده‌سازی الگوریتم بنویسید و با درج شمارنده‌ای (یا شمارنده‌هایی) در آن، تعداد عملیات پایه‌ای الگوریتم را به عنوان تابعی از اندازه دو مجموعه A و B محاسبه کنید.
با هریک از مقادیر $n=10, 10^2, 10^3, 10^4, 10^5, 10^6$ آزمایش کنید.
\end{document}