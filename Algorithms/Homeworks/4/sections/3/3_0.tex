\documentclass[]{article}
\usepackage{hyperref}
\usepackage[a4paper, total={6in, 8in}]{geometry}
\usepackage{caption}
\usepackage{algpseudocodex}
\usepackage{algorithm}
\usepackage{amsmath}
\usepackage{listings}
\usepackage{xepersian}
\usepackage{graphicx}
\usepackage{setspace}
\usepackage{subfiles}
\usepackage{xcolor}
\settextfont{XB Niloofar}

\begin{document}
فروشگاهی بزرگ برای تحلیل رفتار مشتریان خود، آرایه‌ای دو بعدی $A$ را نگهداری میکند
که سطرهای آن، متناظر با مشتریان هستند و ستون‌های آن، متناظر با کالاهایی هستند که به مشتریان فروخته است.
خانه $A \lbrack i, j \rbrack$ نمایانگر این است که مشتری $i$ چند قلم از کالای $j$ را خریده است.

این مثالی است کوچک از چنین آرایه‌ای:

\begin{center}
    \begin{tabular}{ |c | c | c | c | c | }
    \hline
        & ماکارونی & کره & نوشابه & شیر \\ [0.05cm]
    \hline
        مشتری ۱ & 3 & 0 & 6 & 0 \\ [0.05cm]
        مشتری ۲ & 0 & 0 & 3 & 2 \\ [0.05cm]
        مشتری ۳ & 7 & 0 & 0 & 0 \\ [0.05cm]
    \hline
    \end{tabular}
\end{center}
یکی از کارهایی که فروشگاه میتواند با داده‌ها انجام دهد، پیدا کردن زیرمجموعه‌ای متنوع از مشتریان است:
میگوییم که یک زیرمجموعه $S$ از مشتریان، متنوع است اگر هیچ یک از دو مشتری عضو این مجموعه،
یک کالا را نخریده باشند (یعنی هر کالایی در فروشگاه را حداکثر یکی از مشتریان در مجموعه $S$ خریده باشد). 
مجموعه‌های متنوع مشتریان میتوانند مفید باشند؛ مثلاً می‌توان از آنها به عنوان منبعی برای تحقیقات بازاریابی استفاده کرد.

مسأله فروشگاه را می‌توان به این شکل بیان کرد: 
آرایه $A$ با ابعاد $m \times n$ (به شکلی که توصیف شد)
و عدد صحیح $k$ $(k \leq m)$ مشخص شده است؛
آیا زیرمجموعه‌ای از حداقل $k$ مشتری وجود دارد که متنوع باشد؟

\paragraph*{الف}
نشان دهید که این مسأله، در رده $NP$ است.
\paragraph*{ب}
نشان دهید که این مسأله، در رده $NPC$ است.

\end{document}