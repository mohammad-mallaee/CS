\documentclass[]{article}
\usepackage{hyperref}
\usepackage[a4paper, total={6in, 8in}]{geometry}
\usepackage{caption}
\usepackage{algpseudocodex}
\usepackage{algorithm}
\usepackage{amsmath}
\usepackage{listings}
\usepackage{xepersian}
\usepackage{graphicx}
\usepackage{setspace}
\usepackage{subfiles}
\usepackage{xcolor}
\settextfont{XB Niloofar}

\begin{document}
ابتدا مساله را بصورت دقیق‌تر توصیف می‌کنیم:
اگر مجموعه همه‌ی کالاها را $P$ بنامیم، داریم $P = \lbrace p_1, p_2, \dots, p_n \rbrace$.
کالاهایی که هر مشتری خریده‌است را با $C_i$ نشان می‌دهیم به طوری که $C_i \subseteq P$.
بنابراین مجموعه کالاهای خریداری شده توسط مشتری‌ها را $C = \lbrace C_1, \dots C_n \rbrace$
تعریف می‌کنیم. جواب مسئله مجموعه $S$ خواهد بود و داریم:
$$ S = \lbrace S_i \; | \; S_i \in C \rbrace ; \; \forall S_j, S_k: S_j \cap S_k = \emptyset $$

برای اینکه نشان دهیم این مسئله در رده $NPC$ است،
می‌توانیم یکی از مسائل این رده را به این مسئله تبدیل کنیم.
برای این کار مسئله مجموعه مستقل را انتخاب می‌کنیم بدین صورت که:
گراف $G = <V, E>$ را در نظر می‌گیریم، جواب مسئله بزرگترین زیر مجموعه
$V' \subseteq V$
از رئوس گراف است به طوری که هیچ یک از دو راس این مجموعه با یالی به هم وصل نباشند.

با پیمایش گراف $G$، به ازای هر راس یک مجموعه از یال‌های متصل به آن می‌سازیم
بدین معنی که هر راس یک مشتری و هر یال نشان دهنده یک نوع کالا و اتصال دو راس با یک یال
نشان دهنده‌ی خرید یک کالا توسط دو مشتری خواهد بود.
با این توضیحات، مجموعه ساخته شده همان مجموعه $C$ خواهد بود و
حل آن، جواب مسئله مجموعه مستقل را به ما می‌دهد
یعنی مجموعه $S$.

بنابراین این مسئله نیز در رده $NPC$ است.
\end{document}