\documentclass[]{article}
\usepackage{hyperref}
\usepackage[a4paper, total={6in, 8in}]{geometry}
\usepackage{caption}
\usepackage{algpseudocodex}
\usepackage{algorithm}
\usepackage{amsmath}
\usepackage{listings}
\usepackage{xepersian}
\usepackage{graphicx}
\usepackage{setspace}
\usepackage{subfiles}
\usepackage{xcolor}
\settextfont{XB Niloofar}

\begin{document}
برای حل این مسئله از الگوریتم جریان بیشینه استفاده می‌کنیم بدین شرح که 
مسئله را به گرافی سه سطحی تبدیل می‌کنیم که در  سطح اول، راس‌ها نشان‌دهنده دکترها و
یال‌هایی که از مبدا به آنها وصل شده است، دارای حداکثر جریانی هستند که با $c_i$
که حداکثر روزهایی است که دکتر $i$ ام می‌تواند در تعطیلات به بیمارستان بیاید برابر است.
در سطح دوم راس‌های سطح اول به دوره‌هایی تعطلاتی که می‌توانند در آن بازه در بیمارستان حاضر شوند
متصل است و یال‌های بین این دو حداکثر، جریان ۱ را می‌توانند انتقال دهند و
سطح سوم روزهایی از این دوره‌های تعطیلات است که دکتر $i$ ام می‌تواند در آنها حاضر شود.
حداکثر جریان گذرنده از یال‌های بین این راس‌ها نیز یک خواهد بود
و در انتها همه‌ی این راس‌ها به راس مقصد متصل هستند و به طبع یال‌های متناظر دارای ظرفیت ۱ خواهند بود.
اگر جریان ارسالی از مبدا برابر با تعداد همه‌ی روز‌‌های تعطیل باشد و با جریان دریافتی 
در مقصد برابر باشد، آنگاه این مسئله جواب دارد و در غیر اینصورت جواب ندارد.

اگر مجموعه دکترها به شکل $A = \lbrace a_1, \dots, a_n \rbrace$ و 
تعداد همه‌ی روزهای تعطیل را $D$ بنامیم و
مجموعه تعطیلاتی که دکتر $i$ ام می‌تواند در آنها حاظر شود را $D^i$ بنامیم داریم:
$$D^i = \lbrace D_1^i, \dots, D_m^i \rbrace; \; D_j^i = \lbrace d_{j1}^i, \dots d_{jp}^i \rbrace$$

\pagebreak
گراف مورد بحث به شکل زیر در خواهد آمد:

\vspace{0.3cm}
\includegraphics*[scale=0.35]{graph.jpg}
\end{document}