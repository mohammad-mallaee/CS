\documentclass[]{article}
\usepackage{hyperref}
\usepackage[a4paper, total={6in, 8in}]{geometry}
\usepackage{caption}
\usepackage{algpseudocodex}
\usepackage{algorithm}
\usepackage{amsmath}
\usepackage{listings}
\usepackage{xepersian}
\usepackage{graphicx}
\usepackage{setspace}
\usepackage{subfiles}
\usepackage{xcolor}
\settextfont{XB Niloofar}

\begin{document}
این مسأله را در نظر بگیرید:
با گرفتن الفبای $\Sigma$ و مجموعه $S$ از رشته‌های ممنوعه و عدد $n$ ، 
رشته‌ای را به‌ طول $n$ با الفبای $\Sigma$ بسازید که هیچ یک از عناصر مجموعه $S$ ، زیررشته آن نباشد.

برای مثال، اگر $\Sigma = \lbrace 0, 1 \rbrace$ و $S = \lbrace 01, 10 \rbrace$ و $n = 4$ باشند،
آنگاه دو جواب مقبول مسأله عبارتند از $0000$ و $1111$ ؛ 
اما اگر $\Sigma = \lbrace 0, 1 \rbrace$ و $S = \lbrace 0, 11 \rbrace$ و $n = 4$ باشند،
رشته مطلوب وجود نخواهد داشت. 

\paragraph*{الف}
یک الگوریتم عقبگرد برای این مسأله طراحی کنید و با دو مثال مذکور، نحوه اجرای آن را با رسم درخت فضای حالت توضیح دهید.

\paragraph*{ب}
برنامه‌ای برای پیاده‌سازی الگوریتم بنویسید.
درستی و کارایی برنامه خود را با ورودی‌های مختلف (الفباهای مختلف و مجموعه رشته‌های ممنوعه مختلف و طول‌های رشته‌های مطلوب مختلف)
آزمایش کنید. هم خروجی برنامه و هم زمان اجرای برنامه در هر مورد را در جواب خود ذکر کنید. 
\end{document}