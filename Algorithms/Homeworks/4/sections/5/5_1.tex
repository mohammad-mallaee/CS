\documentclass[]{article}
\usepackage{hyperref}
\usepackage[a4paper, total={6in, 8in}]{geometry}
\usepackage{caption}
\usepackage{algpseudocodex}
\usepackage{algorithm}
\usepackage{amsmath}
\usepackage{listings}
\usepackage{xepersian}
\usepackage{graphicx}
\usepackage{setspace}
\usepackage{subfiles}
\usepackage[linguistics]{forest}
\usepackage{xcolor}
\settextfont{XB Niloofar}

\begin{document}
الگوریتم عقبگرد بدین صورت خواهد بود که هر گره می‌تواند به رشته قبلی
هریک از اعضای $\Sigma$ را اضافه کند و این بدین معنی است که هر گره از درخت
به تعداد اعضای $\Sigma$ فرزند دارد.
الگوریتم در هر مرحله شرایط مسئله را بررسی می‌کند تا زیردرخت‌های کمتری تولید کند.

در مثال اول درخت فضای حالت به شکل زیر خواهد بود :
\begin{latin}
\begin{align}    
\begin{forest}
        [ 
            [0 [0 [0 [0] [1 \\ $\times$]] [1 \\ $\times$]] [1 \\ $\times$]]
            [1 [0 \\ $\times$] [1 [0 \\ $\times$] [1 [0 \\ $\times$] [1]]]]
        ]
\end{forest}
\end{align}
\end{latin}
و درخت فضای حالت مثال دوم :
\begin{latin}
\begin{align}    
\begin{forest}
        [ 
            [0 \\ $\times$]
            [1 [0 \\ $\times$] [1 \\ $\times$]]
        ]
\end{forest}
\end{align}
\end{latin}
\end{document}