\documentclass[]{article}
\usepackage{hyperref}
\usepackage[a4paper, total={6in, 8in}]{geometry}
\usepackage{caption}
\usepackage{algpseudocodex}
\usepackage{algorithm}
\usepackage{amsmath}
\usepackage{listings}
\usepackage{xepersian}
\usepackage{graphicx}
\usepackage{setspace}
\usepackage{subfiles}
\usepackage{xcolor}
\settextfont{XB Niloofar}

\begin{document}
فرض میکنیم $A$ یک ماتریس $d \times n$ است و $Ax = 1$.
دستگاه معادلات بصورت زیر خواهد بود:
\begin{align*}
  a_{11}x_1+a_{12}x_2 + \dots + a_{1n}x_n &= \,1 \\
  a_{21}x_1+a_{22}x_2 + \dots + a_{2n}x_n &= \,1 \\
  & \vdots \\
  a_{d1}x_1+a_{d2}x_2 + \dots + a_{dn}x_n &= \,1 \\
\end{align*}
با افزودن همه‌ی معادلات به یکدیگر به معادله زیر دست می‌یابیم:
$$s_1x_1+s_2x_2 + \dots + s_nx_n = d$$
که در آن $s_j = \sum_{i=1}^d a_{ij}$ است.
از آنجایی که بردار $x$ یک بردار صفر-یک است،
بدین معنی است که پس از یافتن جواب اگر $x_i$ یک باشد،
آنگاه $s_i$ در این مجموع محاسبه شده است و اگر صفر باشد،
محاسبه نشده است.
این تعبیر در واقع همان زیر مجموعه‌ای از مجموعه $S = \lbrace s_1, s_2, s_n \rbrace$
است که مجموع عناصر آن برابر $d$ است.
\end{document}