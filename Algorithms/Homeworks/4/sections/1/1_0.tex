\documentclass[]{article}
\usepackage{hyperref}
\usepackage[a4paper, total={6in, 8in}]{geometry}
\usepackage{caption}
\usepackage{algpseudocodex}
\usepackage{algorithm}
\usepackage{amsmath}
\usepackage{listings}
\usepackage{xepersian}
\usepackage{graphicx}
\usepackage{setspace}
\usepackage{subfiles}
\usepackage{xcolor}
\settextfont{XB Niloofar}

\begin{document}
این سه مسأله را (که هر سه در رده NPC هستند) در نظر بگیرید:
\begin{itemize}
    \item \textbf{مسأله 1}:
    فرض کنید $A$ و $B$ و $C$ 
    سه مجموعه مجزای $n$ عنصری
    و $T \subseteq A \times B \times C$
    مجموعه‌ای از سه‌تایی‌های مرتب باشد.
    زیرمجموعه‌ای از $n$ سه‌تایی را
    (در صورت وجود) در $T$ بیابید
    که هر عنصر در $A \cup B \cup C$ ،
    در  دقیقاً یکی از آن $n$ سه‌تایی قرار داشته باشد.

    \item \textbf{مسأله 2}:
    با این فرض که ماتریس $A$
    یک ماتریس صفر-یک $m \times n$ و
    $
    1 = 
    \left[
    \begin{matrix}
        1 \\
        \vdots \\
        1
    \end{matrix}  
    \right]
    $
    بردار $m$ بعدی تمام 1 باشد،
    بردار مجهول صفر-یک
    $
    x = 
    \left[
    \begin{matrix}
        x_1 \\
        \vdots \\
        x_n
    \end{matrix}  
    \right]
    $
    (در صورت وجود)
    بیابید که معادله ماتریسی $Ax = 1$ برقرار شود.

    \item \textbf{مسأله 3}:
    مجموعه $S$ از $n$ عدد صحیح،
    و عدد صحیح $d$ داده شده است؛
    زیرمجموعه‌ای از مجموعه $S$ را
    (در صورت وجود) بیابید که مجموع اعداد آن برابر با $d$ باشد.
\end{itemize}
\paragraph*{الف}
نشان دهید که مسأله 1 را میتوان به مسأله 2 تبدیل کرد.
\paragraph*{ب}
نشان دهید که مسأله 2 را میتوان به مسأله 3 تبدیل کرد.
\end{document}