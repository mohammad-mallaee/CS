\documentclass[]{article}
\usepackage{hyperref}
\usepackage[a4paper, total={6in, 8in}]{geometry}
\usepackage{caption}
\usepackage{algpseudocodex}
\usepackage{algorithm}
\usepackage{amsmath}
\usepackage{listings}
\usepackage{xepersian}
\usepackage{graphicx}
\usepackage{setspace}
\usepackage{subfiles}
\usepackage{xcolor}
\settextfont{XB Niloofar}

\begin{document}
بله، این مسئله از رده NPC است.
ابتدا این مساله را به شکل یک مسئله تصمیم گیری در می‌آوریم:

آیا مجموعه‌ای از وبگاه‌ها با شرایط مسئله وجود دارد که اندازه آن از $k$
کمتر باشد ؟

برای نشان دادن اینکه این مسئله در رده $NP$ است،
فرض می‌کنیم مجموعه $S$ جواب پیشنهادی مسئله است.
درستی این جواب را بدین صورت بررسی می‌کنیم که هر یک از کامپیوتر‌های آلوده حداقل
از یکی از وبسایت‌های این مجموعه بازدید کرده است
در غیر اینصورت جواب پیشنهادی، جواب مسئله نخواهد بود.

حال برای نشان دادن اینکه این مسئله در رده $NPC$ است،
باید یکی از مسائل شناخته شده در این رده را به این مسئله تبدیل کنیم.
برای این کار مسئله پوشش راسی را تبدیل خواهیم کرد.
گراف $G = <V, E>$ را در نظر می‌گیریم.
مجموعه راس‌های گراف را مجموعه‌ی همه‌ی وبسایت‌هایی که کاربران بازدید کرده‌اند در نظر می‌گیریم
و آن را $W = \lbrace w_1, w_2, \dots, w_n \rbrace$ می‌نامیم.
یال گراف را کامپیوترهای آلوده‌ای درنظر می‌گیریم که تنها از راس‌های دوسر خود بازدید کرده‌اند
بدینصورت که $C = \lbrace \lbrace c_{i1}, c_{i2} \rbrace \, | (c_{i1}, c_{i2}) \in E \rbrace$.
اگر $V'$ جواب مسئله پوشش راسی باشد،
آنگاه حداقل یکی از راس‌های $u$ یا $v$ واقع بر هر یال $(u, v) \in E$
در مجموعه $V'$ وجود خواهد داشت
که همان جوابی است که از این مسئله می گیریم بنابراین این مسئله نیز در رده $NPC$ قرار دارد. 
\end{document}