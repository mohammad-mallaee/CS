\documentclass[]{article}
\usepackage{hyperref}
\usepackage[a4paper, total={6.25in, 8.3in}]{geometry}
\usepackage{caption}
\usepackage{algpseudocodex}
\usepackage{algorithm}
\usepackage{amsmath}

\usepackage{listings}
\usepackage{xepersian}
\usepackage{graphicx}
\usepackage{setspace}
\usepackage{subfiles}
\usepackage{xcolor}
\usepackage{mathtools}
\usepackage[linguistics]{forest}

\settextfont{XB Niloofar}
\algrenewcommand{\algorithmicrequire}{\textbf{Input:}}
\algrenewcommand{\algorithmicensure}{\textbf{Output:}}

\definecolor{codegreen}{rgb}{0,0.6,0}
\definecolor{codegray}{rgb}{0.5,0.5,0.5}
\definecolor{codepurple}{rgb}{0.58,0,0.82}
\definecolor{backcolour}{rgb}{0.95,0.95,0.92}

\lstdefinestyle{mystyle}{
    backgroundcolor=\color{backcolour},   
    commentstyle=\color{codegreen},
    keywordstyle=\color{magenta},
    numberstyle=\tiny\color{codegray},
    stringstyle=\color{codepurple},
    basicstyle=\ttfamily\footnotesize,
    breakatwhitespace=false,         
    breaklines=true,                 
    captionpos=b,                    
    keepspaces=true,                 
    numbers=left,                    
    numbersep=5pt,                  
    showspaces=false,                
    showstringspaces=false,
    showtabs=false,                  
    tabsize=2
}

\lstset{style=mystyle}

\begin{document}

% ------------------------------------------------------
%  Course Project Report Information
% -------------------------------------------------------
\newcommand{\ProjectReportType}
{ گزارش ‍‍‍‍‍پروژه درس} 

\newcommand{\StudentDegree}
{کارشناسی}  % کارشناسی / کارشناسی ارشد / دکتری

\newcommand{\CourseMajor}
{علوم کامپیوتر}  % مهندسی کامپیوتر

\newcommand{\CourseReportTitle}
{تمرینات ۱}

\newcommand{\GroupeMembers}
{
    محمد ملائی
}

\newcommand{\CourseName}
{طراحی و تحلیل و الگوریتم ها}

\newcommand{\courseSemester}
{نیم‌سال اول ۱۴۰۲-۱۴۰۳}

\newcommand{\CourseProfessor}
{جعفر الماسی زاده}
\newcommand{\Department}
{دانشکده علوم ریاضی و کامپیوتر}
\newcommand{\University}
{دانشگاه اصفهان}
\newcommand{\EnglishCourseTitle}
{Design and Analysis of Algorithms}

% -------------------------------------------------------
%  English Information
% -------------------------------------------------------


\begin{center}

    \includegraphics[scale=0.2]{./UILogo.png}
    
    \University \\
    \Department\\
    
    
    \begin{large}
    \vspace{0.5cm}
    
    \end{large}
    
    \vspace{1cm}
    \begin{latin}
        {\Large\textbf\EnglishCourseTitle}
    \end{latin}
    \begin{center}
        \CourseName
    \end{center}

    \Large\textbf{محمد ملائی}
    
    \vspace{1cm}
    {عنوان:}\\[0.5em]
    {\LARGE\textbf{\CourseReportTitle}}\\ 
    
    \vspace{1.25cm}
    
    {\large\textbf{\courseSemester}}

    \vspace{1cm}
    {نام استاد درس}\\[0.5em]
    {\large\textbf{\CourseProfessor}}
    
    \vspace{1.2cm}

    \pagebreak
    
    \end{center}
\section*{تمرین ۱}
\subfile{sections/1/1_0.tex}
\subsection*{\color{blue}{جواب}}
\subsubsection*{الف}
\subfile{sections/1/1_1.tex}
\subsubsection*{ب}
\subfile{sections/1/1_2.tex}

% \subsection*{\color{red}{مراجع}}
% \begin{latin}
% \href{https://docs.python.org/3/library/itertools.html}{python documentations -- itertools}
% \end{latin}

\pagebreak
\section*{تمرین ۲}
\subfile{sections/2/2_0.tex}
\subsection*{\color{blue}{جواب}}
\subsection*{الف}
\subfile{sections/2/2_1.tex}
\subsection*{ب}
\subfile{sections/2/2_2.tex}
% \begin{latin}
% \href{https://cs.stackexchange.com/questions/136466/how-to-solve-recursion-tn-tn-3-t2n-3-n}{solving recursion - statckexchange}
% \end{latin}

\pagebreak
\section*{تمرین ۳}
\subfile{sections/3/3_0.tex}
\subsection*{\color{blue}{جواب}}
\subsection*{الف}
\subfile{sections/3/3_1.tex}
\subsection*{ب}
\subfile{sections/3/3_2.tex}
% \subsection*{\color{red}{مراجع}}
\begin{latin}
\href{https://cs.stackexchange.com/questions/85400/show-that-maximum-disjoint-set-problem-is-np-complete}{-CS.StackExchange-}
\end{latin}


\pagebreak
\section*{تمرین ۴}
\subfile{sections/4/4_0.tex}
\subsection*{\color{blue}{جواب}}
\subfile{sections/4/4_1.tex}
% \begin{latin}
%     \href{https://www.youtube.com/watch?v=0tjpC0MCwY8l}{-YouTube-}
% \end{latin}

\pagebreak
\section*{تمرین ۵}
\subfile{sections/5/5_0.tex}
\subsection*{\color{blue}{جواب}}
\subsection*{الف}
\subfile{sections/5/5_1.tex}
\subsection*{ب}
\subfile{sections/5/5_2.tex}
% \subsection*{\color{red}{مراجع}}
% \begin{latin}
%     \href{https://github.com/anxiaonong/Maxflow-Algorithms/blob/master/Ford-Fulkerson%20Algorithm.py}{Ford-Fulkerson github}
% \end{latin}

\end{document}