\documentclass[]{article}
\usepackage{hyperref}
\usepackage[a4paper, total={6in, 8in}]{geometry}
\usepackage{caption}
\usepackage{algpseudocodex}
\usepackage{algorithm}
\usepackage{amsmath}
\usepackage{listings}
\usepackage{xepersian}
\usepackage{graphicx}
\usepackage{setspace}
\usepackage{subfiles}
\usepackage{xcolor}
\settextfont{XB Niloofar}

\begin{document}
ابتدا با این تعریف از ناپایداری نشان می‌دهیم که خروجی الگوریتم پایدار خواهد بود.
با اولین قسمت از تعرق شروع می‌کنیم و با استفاده از برهان خلف فرض می‌کنیم این دو زوج وجود دارند یعنی
(تابع $P_x(y)$ را بدین صورت تعریف می‌کنیم که اندیس $y$ در لیست ترجیحات $x$ را برمی‌گرداند) :
\begin{equation}
    \exists (m, w), (m', w') \in S : P_m(w') < P_m(w) \land P_{w'}(m) < P_{w'}(m')
\end{equation}
و از آنجایی که $P_m(w') < P_m(w)$
می‌توانیم این نتیجه را بگیریم که طبق الگوریتم $m$ از $w'$ زودتر از $w$ خواستگاری می‌کند
اما طبق فرض می‌دانیم که $(m, w') \not \in S$  بنابراین یا $w'$ آزاد نبوده است و پارتنر خود را
به $m$ ترجیح داده است یا آزاد بوده است اما با پیدا کردن پارتنر بهتر $m$ را آزاد کرده است.
در هر صورت می‌توانیم نتیجه بگیریم که $w'$ مرد $m'$ را به $m$ ترجیح می‌دهد و این تناقض است
که نشان می‌دهد چنین اتفاقی رخ نخواهد داد.

در بخش دوم این تعریف نیز با برهان خلف فرض می‌کنیم که چنین مردی وجود دارد، یعنی:
\begin{align}
    (m, w) \in S \\
    \exists m' \in M \; \: \forall w' \in W: (m', w') \not \in S \\
    P_w(m') < P_w(m)
\end{align}
طبق الگوریتم اگر $m'$ آزاد است یعنی $m'$ به همه‌ی زن‌هایی که در لیست ترجیحاتش
هستند پیشنهاد داده است و طبق فرض ۳ می‌دانیم که $(m', w) \not \in S$،
مانند قسمت قبل و از آنجایی که می‌دانیم حتما $m'$ به $w$ پیشنهاد داده است،
می‌توانیم نتیجه بگیریم که $w$ یا درخواست $m'$ را رد کرده است یا او را آزاد کرده‌است
که هردو به این معنا هستند که $P_w(m) < P_w(m')$ که تناقض است و نشان می‌دهد که این اتفاق نیز رخ نخواهد داد.

بخش سوم نیز به سادگی رد می‌شود چرا که اگر $m$، $w'$ را ترجیح می‌داد به او زودتر پیشنهاد می‌داد
و $w'$ چون آزاد است، پیشنهاد او را قبول می‌کرد.

قسمت آخر نیز به سادگی رد می‌شود زیرا الگوریتم تا جایی ادامه پیدا می‌کند که هر مرد یا پارتنر خود را پیدا کرده باشد
یا به تمام زن‌هایی که در لیست ترجیحاتش هستند پیشنهاد داده باشد و این بدین معنی است که اگر
در بدترین حالت $w$ آخرین نفر این لیست باشد، به دلیل آزاد بودن پیشنهاد $m$ را خواهد پذیرفت و تنها نخواهد ماند.


\end{document}