\documentclass[]{article}
\usepackage{hyperref}
\usepackage[a4paper, total={6in, 8in}]{geometry}
\usepackage{caption}
\usepackage{algpseudocodex}
\usepackage{algorithm}
\usepackage{amsmath}
\usepackage{listings}
\usepackage{xepersian}
\usepackage{graphicx}
\usepackage{setspace}
\usepackage{subfiles}
\usepackage{xcolor}
\settextfont{XB Niloofar}

\begin{document}
مسأله ازدواج پایدار را به این شکل تعمیم میدهیم که تشکیل زوج‌های خاصی از مردها - زنها صریحاً ممنوع باشد.
(در مورد تطابق کارفرماها و کارجوها، میتوانیم این گونه تصور کنیم که بعضی از کارجوها فاقد صلاحیت‌ها یا گواهی‌های لازم باشند و بنابراین،
با وجود آنکه موجه به نظر میرسند، نتوانند در شرکتهای خاصی استخدام شوند.)
پس ما یک مجموعه $M$ شامل $n$ مرد داریم و یک مجموعه $W$ شامل $n$ زن.
و یک مجموعه $F \subseteq M \times W$ شامل زوج‌هایی که مجاز به ازدواج با یکدیگر نیستند.
هر مرد $m$ ، تمام زنهای $w$ را با شرط $(m, w) \not \in F$ رتبه‌بندی میکند و هر زن $w'$ ، تمام مردهای $m'$ را با شرط $(m', w') \not \in F$ رتبه‌بندی میکند.

در این قالب کلیتر از مسأله ازدواج پایدار، ما میگوییم که یک تطابق ازدواج $S$ پایدار است، اگر هیچ یک از این نوع ناپایداریها را نداشته باشد:
\begin{itemize}
    \item دو زوج $(m, w)$ و $(m', w')$ در $S$ وجود داشته باشند و با شرط $(m', w') \not \in F$ ،
    مرد $m$ ترجیح دهد زن $w'$ را به $w$ ، و زن $w'$ ترجیح دهد مرد $m$ را به $m'$ . (این حالت، همان نوعِ عادی ناپایداری است.)

    \item زوج $(m, w) \in S$ باشد، اما یک مرد $m'$ وجود داشته باشد که در هیچ زوجی از تطابق قرار نگرفته باشد، و با شرط $(m', w) \not \in F$ ،
    $w$ ترجیح دهد $m'$ را به $m$ .
    (در این حالت، زنی با مردی زوج شده است، ولی مردی مجرد را که ازدواج با او ممنوع نیست، به آن مرد ترجیح میدهد.)

    \item زوج $(m, w) \in S$ باشد، اما یک زن $w'$ وجود داشته باشد که در هیچ زوجی از تطابق قرار نگرفته باشد،
    و با شرط $(m, w') \not \in F$ ، $m$ ترجیح دهد $w'$ را به $w$ .
    (در این حالت، مردی با زنی زوج شده است، اما زنی مجرد را که ازدواج با او ممنوع نیست، به آن زن ترجیح میدهد.)

    \item یک مرد $m$ و یک زن $w$ وجود داشته باشند که با شرط $(m, w) \not \in F$ ،
    هیچ یک از آن دو در هیچ زوجی از تطابق قرار نگرفته باشند.
    (در این حالت، یک مرد مجرد و یک زن مجرد وجود دارند که مانعی برای ازدواج آنها با یکدیگر وجود ندارد.)
\end{itemize}

\paragraph*{الف}
ثابت کنید که با این تعریف از ناپایداری یک تطابق ازدواج، میتوان با همان الگوریتمی که مسأله پایه‌ای ازدواج پایدار را حل میکند، این مسأله را نیز حل کرد.
الگوریتم باید همیشه برای هر مجموعه‌ای از لیستهای ترجیحات مردان و زنان و هر مجموعه‌ای از زوجهای ممنوع، یک تطابق ازدواج پایدار تولید کند.

\paragraph*{ب}
برنامه‌ای برای پیاده‌سازی الگوریتم بنویسید به نحوی که رده کارایی زمانی آن $O(n^2)$ باشد.
با اجرای برنامه روی چند نمونه ورودی مختلف، درستی آن را تحقیق کنید.

\end{document}