\documentclass[]{article}
\usepackage{hyperref}
\usepackage[a4paper, total={6in, 8in}]{geometry}
\usepackage{caption}
\usepackage{algpseudocodex}
\usepackage{algorithm}
\usepackage{amsmath}
\usepackage{listings}
\usepackage{xepersian}
\usepackage{graphicx}
\usepackage{setspace}
\usepackage{subfiles}
\usepackage{xcolor}
\settextfont{XB Niloofar}

\begin{document}
در گونه‌ای از مسائل جریان شبکه، مقدار جریانی که باید از مبدأ به مقصد بفرستیم، از قبل مشخص شده است
اما موضوع این است که ارسال جریان از هر رأس به رأسی دیگر هزینهای خواهد داشت. 
بنابراین، ما به دنبال «هدایت بهینه» جریانی با مقدار معلوم هستیم؛
به این معنا که هزینه ارسال آن از مبدأ به مقصد، حداقل مقدار ممکن باشد.

در اینجا نیز می‌توان شبکه انتقال مورد نظر را با گراف $G = \: <V,E>$ که گرافی جهتدار، همبند و وزندار است، نمایش داد.
این گراف، $n$ رأس دارد و رئوس آن از 1 تا $n$ شمارهگذاری شده‌اند؛
دقیقاً یک رأسِ بدون یال ورودی دارد؛این رأس، مبدأ نامیده میشود و شماره آن 1 است؛
دقیقاً یک رأس بدون یال خروجی دارد؛ این رأس، مقصد نامیده می‌شود و شماره آن $n$ است.
\\ \\
هر یال جهتدار $(i, j)$ گراف دو برچسب دارد:
\begin{itemize}
    \item برچسب $u_{ij}$ ، که یک عدد صحیح مثبت است و ظرفیت یال را مشخص میکند.
    \item برچسب $c_{ij}$ ، که یک عدد حقیقی مثبت است و هزینه ارسال یک واحد جریان را از طریق آن یال (خط) مشخص می‌کند.
\end{itemize}

راه‌حلی الگوریتمی برای این مسأله بیان کنید که با آن بتوان در یک شبکه انتقال که ساختار آن و ظرفیت هر یک از خطوط آن و هزینه ارسال جریان از هر یک از خطوط آن معلوم باشد،
جریانی با مقدار مشخص $f$ را (در صورت امکان) با کمترین هزینه ممکن، از مبدأ به مقصد فرستاد. 

\end{document}