\documentclass[]{article}
\usepackage{hyperref}
\usepackage[a4paper, total={6in, 8in}]{geometry}
\usepackage{caption}
\usepackage{algpseudocodex}
\usepackage{algorithm}
\usepackage{amsmath}
\usepackage{listings}
\usepackage{xepersian}
\usepackage{graphicx}
\usepackage{setspace}
\usepackage{subfiles}
\usepackage{xcolor}
\settextfont{XB Niloofar}

\begin{document}
فرض کنید میخواهیم در شبکه‌ای رایانه‌ای، از رایانه‌ای خاص به عنوان رایانه مبدأ،
جریانی پیوسته از داده‌ها را با حداکثر سرعت ممکن به رایانه‌ای دیگر به عنوان رایانه مقصد منتقل کنیم.
برای آنکه بتوانیم میزان بیشتری از داده‌ها را در واحد زمان ارسال کنیم،
میتوانیم داده‌ها را به بسته‌هایی تقسیم کنیم و آن بسته‌ها را از طریق مسیرهای مختلف از مبدأ به مقصد بفرستیم.
از طرف دیگر، برای ارسال داده‌ها هم با این قید مواجهیم که نمیتوان داده‌ها را با سرعتی بیشتر از یک مقدار مشخص از هر خط ارتباطی بین دو رایانه عبور داد
(هر مسیریاب) عبور داد. 
این شبکه رایانه‌ای را میتوان با یک گراف‌جهتدارِ وزندار نمایش داد:
هر رایانه عضو شبکه را میتوان رأسی از رئوس گراف در نظر گرفت و هر خط ارتباطیِ یک طرفه بین دو رایانه را میتوان یالی از یالهای جهتدار گراف دانست.
پهنای باند هر خط ارتباطی (که حداکثر تعداد بایتهایی است که میتوان در یک ثانیه از آن خط عبور داد) در شبکه،
وزن یک یال جهتدار در گراف را مشخص میکند و پهنای باند هر رایانه در شبکه، وزن یک رأس در گراف را مشخص میکند.

\paragraph*{الف}
برنامه‌ای برای حل حالت کلی این مسأله بنویسید.
این برنامه باید گراف جهتدارِ وزندار $G=<V, E>$ و دو رأس مبدأ $s$ و مقصد $t$ را بگیرد و حداکثر میزان داده‌هایی را که میتوان در واحد زمان از رایانه $s$ به رایانه $t$ منتقل کرد،
و همچنین میزان بسته‌های ارسالی از هر یک از خطوط ارتباطی و از هر یک از رایانه‌های میانی شبکه را تعیین کند.

\paragraph*{ب}
برای آنکه بتوانید درستی برنامه خود را به طور دستی بیازمایید، سه شبکه (گراف‌جهتدارِ وزندار) تولید کنید
برنامه خود را روی آن ورودی‌ها اجرا کنید.
نهایتاً آن سه شبکه را به طور دستی نیز بکشید و میزان بسته‌های ارسالی از هر یک از خطوط ارتباطی و از هر یک از رایانه‌های میانی شبکه را نیز روی آنها مشخص کنید. 
\end{document}