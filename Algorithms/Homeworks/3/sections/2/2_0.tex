\documentclass[]{article}
\usepackage{hyperref}
\usepackage[a4paper, total={6in, 8in}]{geometry}
\usepackage{caption}
\usepackage{algpseudocodex}
\usepackage{algorithm}
\usepackage{amsmath}
\usepackage{listings}
\usepackage{xepersian}
\usepackage{graphicx}
\usepackage{setspace}
\usepackage{subfiles}
\usepackage{xcolor}
\settextfont{XB Niloofar}

\begin{document}
جاده‌ای طولانی و مستقیم را در منطقه‌ای سرسبز تصور کنید که از نزدیکی شهری بزرگ میگذرد و در کناره‌های آن، خانه‌های مسکونی و فروشگاه‌هایی، دور از هم یا نزدیک به هم، قرار گرفته‌اند.
موضوع این است که افراد بومی ساکن در خانه‌ها و شاغل در فروشگاه‌ها، جمعیت نسبتاً زیادی را تشکیل میدهند و روزانه و شبانه مسافران زیادی نیز از جاده گذر میکنند،
ولی سطل زباله‌ای در کناره‌های جاده وجود ندارد و افراد بومی و مسافران حواشی جاده را پر از زباله کرده‌اند!
شهرداری برای رفع این معضل، تصمیم گرفته است که تعدادی سطل زباله بزرگ را با فاصله‌های مناسب در کناره‌های جاده نصب کند.
از آنجا که رهگذران میتوانند زباله‌های خود را در هر سطل زباله‌ای خالی کنند، شهرداری برای تشویق افراد بومی به تمیز نگه داشتن محل زندگی خود،
میخواهد سطلهای زباله را در نقاطی نصب کند که افراد ساکن در هر ساختمان کناره جاده،
حداکثر 1 کیلومتر راه را برای رسیدن به نزدیک‌ترین سطل زباله به ساختمان خود بپیمایند.
از طرف دیگر، شهرداری بودجهای آنچنانی برای خرید و نصب سطل‌های زباله ندارد و میخواهد از کمترین تعداد سطل زباله ممکن استفاده کند.


\paragraph*{الف}
الگوریتمی را توصیف کنید که به عنوان ورودی، مجموعه نقاطی را که مکان‌های همه ساختمان‌های مستقر در کناره‌های جاده را مشخص میکنند،
بگیرد و به عنوان خروجی، کمترین تعداد سطل‌های زباله لازم و نقطه نصب هر یک از آنها را بدهد.
برای حل مسأله در حالت کلی، تعداد ساختمان‌ها را $n$ بگیرید.
درستی الگوریتم خود را ثابت کنید.

\paragraph*{ب}
برنامه‌ای برای پیاده‌سازی الگوریتم خود بنویسید و با اجرای آن روی چند نمونه ورودی مختلف، درستی آن را تحقیق کنید.
\end{document}