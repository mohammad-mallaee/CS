\documentclass[]{article}
\usepackage{hyperref}
\usepackage[a4paper, total={6in, 8in}]{geometry}
\usepackage{caption}
\usepackage{algpseudocodex}
\usepackage{algorithm}
\usepackage{amsmath}
\usepackage{listings}
\usepackage{xepersian}
\usepackage{graphicx}
\usepackage{setspace}
\usepackage{subfiles}
\usepackage{xcolor}
\settextfont{XB Niloofar}

\begin{document}
با فرض اینکه نقاط نشان‌دهنده ساختمان‌ها روی یک خط قرار دارند و می‌توان مختصات آنها را بصورت یک عدد حقیقی
نشان داد، الگوریتم حریصانه‌ای را برای حل این مسئله به شکل زیر توصیف می‌کنیم:

ابتدا نقاط را مرتب می‌کنیم و سپس با شروع از اولین نقطه، نقاطی را که فاصله آنها تا این نقطه حداکثر ۲ است را
یک مجموعه در نظر می‌گیریم که نشان دهنده این است که تنها یک سطل زباله می‌تواند شرط‌های مسئله را
برای این مجموعه نقاط تامین کنید. البته با ذکر این نکته که مختصات آن دقیقا در وسط این مجموعه باشد.
سپس مجموعه‌ای جدید در نظر می‌گیریم و همین روند برای برای نقاط باقیمانده ادامه می‌دهیم تا همه‌ی ساختمان‌ها
به یک سطل زباله دسترسی داشته باشند.

\begin{latin}
\begin{lstlisting}[language=python]
def min_trash_bins(points: list):
    points.sort()
    n = len(points)
    i = 0
    bins = []
    while i < n:
        p = points[i]
        j = 1
        while i + j < n and abs(p - points[i + j]) <= 2:
            j += 1
        bins.append((p + points[i + j - 1]) / 2)
        i = i + j
    return bins
\end{lstlisting}
\end{latin}

برای اثبات درستی الگوریتم از استقرای ریاضی استفاده می‌کنیم. اگر فقط یک ساختمان داشته باشیم،
الگوریتم مختصات همان ساختمان را به عنوان جایی که باید سطل زباله را نصب کنیم برمی‌گرداند که درست است.
بنابر استقرای ریاضی فرض می‌کنیم الگوریتم برای $n$ نقطه اول جواب درستی می‌دهد،
نشان می‌دهیم با در نظر گرفتن نقطه $n + 1$ نیز خروجی الگوریتم همچنان درست است.

اگر برای $n$ نقطه اول $k$ سطل زباله نصب کنیم که از 1 شماره گذاری شده‌اند، $P_{k0}$ را اولین نقطه در مجموعه
$k$ ام تعریف می‌کنیم و $P_{t}$ را مختصات سطل زباله‌ای که به این مجموعه تعلق می‌گیرد تعریف می‌کنیم.

اگر $P_{n+1} - P_{k0} \le 2$، الگوریتم این نقطه را به این مجموعه اضافه می‌کند و تغییری در تعداد سطل‌های زباله ایجاد نمی‌کند
چرا که:
$$max \lbrace P_t - P_{k0} \rbrace = 1, \; max \lbrace P_{n+1} - P_{t} \rbrace = 1 \rightarrow
max \lbrace P_{n+1} - P_{k0} \rbrace = 2$$
بنابراین در بدترین حالت که فاصله نقطه جدید و اولین نقطه ۲ است، کافی است سطل زباله را وسط این دو نقطه
قرار دهیم. این کار روی سایر نقاط مجموعه تاثیری ندارد زیرا به دلیل مرتب بودن آنها فاصله‌شان از نقطه ابتدایی از نقطه جدید کمتر است
و چون سطل زباله نقاط ابتدایی و انتهایی مجموعه را پوشش می‌دهد، آنهارا نیز پوشش خواهد داد.

اگر $P_{n+1} - P_{k0} > 2$، الگوریتم مجموعه جدیدی ایجاد خواهد کرد و این نقطه اولین نقطه آن خواهد بود.
نمی‌توانیم این نقطه را به مجموعه قبلی اضافه کنیم زیرا نمی‌توانیم جایی برای سطل زباله بیابیم که شرط‌های مسئله را داشته باشد.
نتیجه گیری بالا برای این قسمت نیز صادق است و احتیاجی به اثبات دوباره نیست چرا که اگر
$max \lbrace P_{n+1} - P_{k0} \rbrace = 2$ با شرط این قسمت در تضاد است.
بنابراین ثابت شد که الگوریتم جواب بهینه را به ما می‌دهد زیرا در صورت امکان نقاط را به مجموعه‌های
قبلی اضافه می‌کند و تنها زمانی که ممکن نیست، مجموعه‌ای جدید ایجاد می‌کند.

\end{document}