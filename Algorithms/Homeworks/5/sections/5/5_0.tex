\documentclass[]{article}
\usepackage{hyperref}
\usepackage[a4paper, total={6in, 8in}]{geometry}
\usepackage{caption}
\usepackage{algpseudocodex}
\usepackage{algorithm}
\usepackage{amsmath}
\usepackage{listings}
\usepackage{xepersian}
\usepackage{graphicx}
\usepackage{setspace}
\usepackage{subfiles}
\usepackage{xcolor}
\settextfont{XB Niloofar}

\begin{document}
\paragraph*{الف}
برنامه‌ای برای پیاده‌سازی الگوریتم «نزدیک‌ترین همسایه» برای حل مسأله فروشنده دوره‌گرد بنویسید.
حداقل 10 نمونه‌ با اندازه‌های مختلف از مسأله فروشنده دوره‌گرد را تولید کنید و برنامه خود را روی آن ورودی‌ها اجرا کنید
و زمان‌های اجرای برنامه و خروجی‌های برنامه (مقدار تقریبی گشت فروشنده‌گرد) را در یک جدول ثبت کنید.

\paragraph*{ب}
برنامه‌ای برای پیاده‌سازی الگوریتم $2OPT$ برای حل مسأله فروشنده دوره‌گرد بنویسید.
به عنوان گشت اولیه، از خروجی برنامه‌ پیاده‌ساز الگوریتم «نزدیک‌ترین همسایه» استفاده کنید.
برنامه خود را روی همان داده‌های ورودی‌ای که در قسمت (الف) تولید کرده‌اید اجرا کنید.
(پیشاپیش مشخص کنید که برنامه خود را روی هر ورودی به چه مدت زمانی می‌خواهید اجرا کنید.)
زمان‌های اجرا و خروجی‌های این برنامه و برنامه اول را در یک جدول ثبت کنید تا کارایی زمانی و میزان دقت خروجی‌های دو برنامه را با هم مقایسه کنید.
\end{document}