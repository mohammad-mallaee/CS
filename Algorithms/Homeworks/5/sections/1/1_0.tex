\documentclass[]{article}
\usepackage{hyperref}
\usepackage[a4paper, total={6in, 8in}]{geometry}
\usepackage{caption}
\usepackage{algpseudocodex}
\usepackage{algorithm}
\usepackage{amsmath}
\usepackage{listings}
\usepackage{xepersian}
\usepackage{graphicx}
\usepackage{setspace}
\usepackage{subfiles}
\usepackage{xcolor}
\settextfont{XB Niloofar}

\begin{document}
این دو مسأله را در نظر بگیرید:
\begin{itemize}
    \item \textbf{مساله ۱}:
    فرض کنید $G = <V, E>$ یک گراف جهتدار بی‌دور باشد. طولانی‌ترین مسیر را در گراف $G$ بیابید.
    \item \textbf{مساله ۲}:
    فرض کنید $G=<V, E>$ یک گراف جهتدار وزندار باشد.
    (وزن یالها ممکن است عددی مثبت یا عددی منفی باشد.)
    دوری را (در صورت وجود) در گراف $G$ بیابید که مجموع وزن یالهای آن صفر باشد. 
\end{itemize}

\paragraph*{الف}
یا با ارائه الگوریتمی کارا ثابت کنید که مسأله 1 در رده P است؛
یا با تبدیل مسأله‌ای در رده NPC به آن، ثابت کنید که در رده NPC است.

\paragraph*{ب}
یا با ارائه الگوریتمی کارا ثابت کنید که مسأله 2 در رده P است؛
یا با تبدیل مسأله‌ای در رده NPC به آن، ثابت کنید که در رده NPC است.
\end{document}