\documentclass[]{article}
\usepackage{hyperref}
\usepackage[a4paper, total={6in, 8in}]{geometry}
\usepackage{caption}
\usepackage{algpseudocodex}
\usepackage{algorithm}
\usepackage{amsmath, mathtools}
\usepackage{listings}
\usepackage{xepersian}
\usepackage{graphicx}
\usepackage{setspace}
\usepackage{subfiles}
\usepackage{xcolor}
\settextfont{XB Niloofar}

\begin{document}
این مسئله در رده P است.
برای حل این مسئله ابتدا اگر گراف وزن دار باشد، وزن همه‌ی یال‌ها را قرینه می‌کنیم
و اگر بدون وزن باشند همه را $-1$ در نظر می‌گیریم،
سپس کافیست به ازای هر راس، کوتاه‌ترین مسیر گراف جدید را پیدا کنیم. کوتاه‌ترین مسیر
این مجموعه، همان جواب مسئله‌است چون مقدار آن از بقیه بیشتر است.

برای پیدا کردن کوتاه ترین مسیر از الگوریتم $dijkstra$ نمی‌توانیم استفاده کنیم
اما می‌توانیم ابتدا آنها را با الگوریتم $toplogical$ $sort$ مرتب کنیم
سپس با استفاده از الگوریتمی تعمیم یافته از پیمایش سطحی، هربار با رسیدن به یک راس، چون کوتاه‌ترین
مسیر تا خود راس را داریم با بررسی راس های متصل، اگر مسیری که از راس کنونی می‌گذرد
کوتاه‌تر باشد، آن را در نظر می‌گیریم. پس از اتمام الگوریتم
کوتاه ترین مسیر را بین راس ابتدایی و هر راس دیگر داریم. بنابراین کافیست
الگوریتم را یکبار برای هر راس اجرا کنیم تا طولانی‌تری مسیر گراف $G$ را بیابیم

از آنجایی که می‌توانیم الگوریتم $toplogical$ $sort$
را در زمان $O(V + E)$ اجرا کنیم
و الگوریتم تعریف شده نیز کارایی زمانیش $O(|V + E|)$
است که آن را به تعداد راس‌ها اجرا می‌کنیم پس کارایی الگوریتم بصورت زیر خواهد بود:
$$O((V+E|) + V(V+E)) = O((V+1)(V+E)) \in O(V^3)$$
\end{document}