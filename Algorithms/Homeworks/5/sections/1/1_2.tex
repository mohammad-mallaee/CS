\documentclass[]{article}
\usepackage{hyperref}
\usepackage[a4paper, total={6in, 8in}]{geometry}
\usepackage{caption}
\usepackage{algpseudocodex}
\usepackage{algorithm}
\usepackage{amsmath}
\usepackage{listings}
\usepackage{xepersian}
\usepackage{graphicx}
\usepackage{setspace}
\usepackage{subfiles}
\usepackage{xcolor}
\settextfont{XB Niloofar}

\begin{document}
مساله ۲ در رده $NPC$ است.
برای اثبات این موضوع، مسئله مسیر همیلتونی را که یک مسئله سخت است
به این مسئله تبدیل می‌کنیم.
فرض می‌کنیم گراف $G$ و دو راس $t$ و $s$ را داشته باشیم
می‌خواهیم یک مسیر همیلتونی از راس $s$ به راس $t$ بیابیم.
حالا از روی گراف $G$ گراف $G'$ را می‌سازیم
و وزن همه‌ی یال‌ها را ۱ قرار می‌دهیم سپس یال جدیدی از $t$ به $s$ اضافه می‌کنیم
و وزن آن‌را $1-n$ قرار می‌دهیم.
واضح است که یک مسیر همیلتونی وجود دارد اگر و تنها اگر یک دور با مجموع صفر در
گراف $G'$ وجود داشته باشد.
\end{document}