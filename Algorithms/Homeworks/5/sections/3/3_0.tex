\documentclass[]{article}
\usepackage{hyperref}
\usepackage[a4paper, total={6in, 8in}]{geometry}
\usepackage{caption}
\usepackage{algpseudocodex}
\usepackage{algorithm}
\usepackage{amsmath}
\usepackage{listings}
\usepackage{xepersian}
\usepackage{graphicx}
\usepackage{setspace}
\usepackage{subfiles}
\usepackage{xcolor}
\settextfont{XB Niloofar}

\begin{document}
تصور کنید که شما برای یک شرکت باربری بزرگ کار میکنید و یکی از وظایف شما این است که با استفاده از تعدادی کامیون،
مجموعه‌ای از $n$ جعبه را از یک شهر بندری به شهرهای دور از آن منتقل کنید.
شما میدانید که بار این کامیون‌ها در نقاط مختلفی در طول مسیر، وزن خواهند شد
و در صورتی که هر یک از کامیون‌ها بیش از حد مجاز بار شده باشد،
شرکت باربری باید جریمه‌ای بپردازد.
بنابراین، شما میخواهید که جعبه‌ها را به گونه‌ای در کامیون‌‌ها بگذارید
که «وزن بارِ پر بارترین کامیون» به حداقل برسد.

\paragraph*{الف}
با این فرض که تعداد کامیون‌ها و وزن هر یک از  جعبه (که اعدادی صحیح هستند) معلوم باشند،
یک الگوریتم حریصانه تقریبی برای تخصیص جعبه‌ها به کامیون‌ها طراحی کنید
که نسبت تقریب آن 2 باشد.

\paragraph*{ب}
ثابت کنید که الگوریتم حریصانه‌ای که برای مسأله «کمینه‌سازی وزن بار پر بارترین کامیون»
طراحی کرده‌اید، یک الگوریتم 2 – تقریبی است. کارایی زمانی الگوریتم‌تان را هم اندازه بگیرید.
\end{document}