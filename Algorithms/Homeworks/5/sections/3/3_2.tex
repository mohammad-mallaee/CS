\documentclass[]{article}
\usepackage{hyperref}
\usepackage[a4paper, total={6in, 8in}]{geometry}
\usepackage{caption}
\usepackage{algpseudocodex}
\usepackage{algorithm}
\usepackage{amsmath}
\usepackage{listings}
\usepackage{xepersian}
\usepackage{graphicx}
\usepackage{setspace}
\usepackage{subfiles}
\usepackage{xcolor}
\settextfont{XB Niloofar}

\begin{document}
فرض می‌کنیم $S$ مجموع وزن همه‌ی جعبه‌ها باشد و
$s^*$ جواب بهینه باشد بطوری‌که وزن آن $W(s^*)$ باشد.
داریم :
\begin{equation}
    W(s^*) \ge max(b_i); b_i \in Boxes
\end{equation}
این موضوع واضح است چرا که بزرگترین حتی اگر بزرگترین جعبه را به تنهایی در یک
کامیون بگذاریم، یک حد پایین برای جواب بهینه خواهد بود.

از طرفی به سادگی می‌توانیم نتیجه بگیریم که:
\begin{equation}
    W(s_a) \le W(s^*) + max(b_i) ; bi \in Boxes
\end{equation}
چرا که اگر جز این بود یعنی یکی از کامیون‌ها حداقل دو جعبه بیشتر از دیگر کامیون‌ها داشت
که با روند الگوریتم در تضاد بود چون حداقل یکی از جعبه‌ها را به کامیون دیگری می‌داد که از 
$W(s^*)$ کمتر بود.
و از (۱) نتیجه می‌گیریم که :
\begin{equation}
    W(s_a) \le W(s^*) + W(s^*) = 2 W(s^*)
\end{equation}
بنایراین ثابت شد که الگوریتم، یک الگوریتم ۲-تقریبی است.
\end{document}