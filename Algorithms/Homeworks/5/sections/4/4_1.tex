\documentclass[]{article}
\usepackage{hyperref}
\usepackage[a4paper, total={6in, 8in}]{geometry}
\usepackage{caption}
\usepackage{algpseudocodex}
\usepackage{algorithm}
\usepackage{amsmath}
\usepackage{listings}
\usepackage{xepersian}
\usepackage{graphicx}
\usepackage{setspace}
\usepackage{subfiles}
\usepackage{xcolor}
\settextfont{XB Niloofar}

\begin{document}
برای حل این مسئله  کافی‌است آن را به مسئله پیدا کردن ریشه تبدیل کنیم
سپس با الگوریتم‌هایی مانند دوبخشی یا نابجایی حل کنیم که هر دو کارا هستند.
تابع $g$ را تعریف می‌کنیم به طوری که :
$$g(x) = f(x) - y$$
پس از تبدیل مسئله خواهیم داشت:
\begin{equation}
    g(a) = f(a) - y, \; g(b) = f(b) - y
\end{equation}

اگر $f$ در بازه مورد نظر شرط زیر را داشته باشده آنگاه طبق قضیه میانی مسئله اول دارای جواب است:
\begin{align}
    f(a) < y < f(b)
\end{align}
پس داریم:
\begin{align}
    \rightarrow f(a) - y < y - y < f(b) - y \\
    (1) \longrightarrow g(a) < 0 < g(b)
\end{align}
بنابراین مسئله دوم نیز جواب خواهد داشت و جواب آن با مسئله اول برابر خواهد بود.
\end{document}