\documentclass[]{article}
\usepackage{hyperref}
\usepackage[a4paper, total={6in, 8in}]{geometry}
\usepackage{caption}
\usepackage{algpseudocodex}
\usepackage{algorithm}
\usepackage{amsmath}
\usepackage{listings}
\usepackage{xepersian}
\usepackage{graphicx}
\usepackage{setspace}
\usepackage{subfiles}
\usepackage{xcolor}
\settextfont{XB Niloofar}

\begin{document}
ابتدا تابع محاسبه ریشه را پیاده‌سازی می‌کنیم:
\begin{latin}
\begin{lstlisting}[language=Python]
def bisection_root(fn, eps, period, max_iters=math.inf):
    a, b = period
    iterations = 0
    while iterations <= max_iters:
        iterations += 1
        x = (a + b) / 2
        if abs(x - a) <= eps:
            return x
        if (fn(x) * fn(a)) < 0:
            b = x
        else:
            a = x
\end{lstlisting}
\end{latin}

\begin{latin}
\begin{lstlisting}[language=Python] 
def false_position_root(fn, eps, period, max_iters=math.inf):
    a, b = period
    iterations = 0
    while iterations <= max_iters:
        iterations += 1
        x = (a * fn(b) - b * fn(a)) / (fn(b) - fn(a))
        if abs(fn(x)) <= eps:
            return x
        if fn(x) * fn(a) < 0:
            b = x
        else:
            a = x
    return x
\end{lstlisting}
\end{latin}
هر دو تابع از روشی مشابه استفاده می‌کنند که در هر مرحله بازه را کوچکتر می‌کنند
تا به دقت مورد نظر برسند.
نحوه کار الگوریتم در کلاس توضیح داده شده است پس  به سراغ پیدا کردن جواب مسئله اصلی می‌رویم.
برای این کار کافی است تنها تابع را تغییر دهیم و به توابع بالا بدهیم بدین‌صورت که :
\begin{latin}
\begin{lstlisting}[language=Python]
def find_point_bisection(fn, y, eps, period):
    return bisection_root(lambda x: fn(x) - y, eps, period)

def find_point_false_p(fn, y, eps, period, max_iters=math.inf):
    return false_position_root(lambda x: fn(x) - y, eps, period, max_iters)
\end{lstlisting}
\end{latin}

حالا برنامه را با دقت $10^{-14}$ روی چند تابع پیوسته که در بازه مورد نظر صعودی باشند بررسی می‌کنیم:
\begin{align}
    x^2 + \ln(x) \\
    e^{x + 2} + \sin(x) \\
    x^3 + 2x^2 + 2x + 4 \\
    x^3 - x - 1
\end{align}
خروجی برنامه بصورت زیر خواهد بود:
\begin{latin}
\begin{lstlisting}
    0 -- y: 3 , period: (0.4, 4)
    bisection: 1.5921429370580948
    false position: 1.5921429370580926
    answer : 1.5921429370581
    
    1 -- y: 2 , period: (-3, -0.5)
    bisection: -0.9629509544247856
    false position: -0.9629509544247973
    answer : -0.9629509545352
    
    2 -- y: 7 , period: (-1, 3)
    bisection: 0.7429592021663112
    false position: 0.7429592021663136
    answer : 0.7429592021663
    
    3 -- y: 1 , period: (-1, 2)
    bisection: 1.5213797068045647
    false position: 1.5213797068045662
    answer : 1.5213797068046  
\end{lstlisting}
\end{latin}
\end{document}