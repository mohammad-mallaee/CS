\documentclass[]{article}
\usepackage{hyperref}
\usepackage[a4paper, total={6in, 8in}]{geometry}
\usepackage{caption}
\usepackage{algpseudocodex}
\usepackage{algorithm}
\usepackage{amsmath}
\usepackage{listings}
\usepackage{xepersian}
\usepackage{graphicx}
\usepackage{setspace}
\usepackage{subfiles}
\usepackage{xcolor}
\settextfont{XB Niloofar}

\begin{document}
این مسأله را در نظر بگیرید: تابع پیوسته صعودی $f$ و مقدار $y$ و بازه باز $(a, b)$ مشخص شده‌اند. 
مقدار $x$ ای را در بازه باز $(a, b)$ بیابید که $f(x) = y$ باشد.
(تابعی مانند $f(x) = x^3 + x - 100$ در کل دامنه‌اش یعنی در کل بازه $(-\infty, \infty)$ صعودی است.
اما اگر تابعی در کل دامنه‌اش صعودی نباشد، کافی است که در بازه مورد نظر صعودی باشد.
مثلاً تابع $f(x) = sin(x)$ در بازه  صعودی است و در بازه $(0, \frac{\pi}{2})$ نزولی است.)

\paragraph*{الف}
الگوریتمی کارا را برای حل این مسأله توصیف کنید.

\paragraph*{ب}
برنامه‌ای برای حل تقریبی مسأله بنویسید. 
آستانه‌ای برای حداکثر میزان خطا تعیین کنید و درستی برنامه‌تان را با چند تابع پیوسته 
(که در بازه‌های مورد نظر صعودی باشند) بیازمایید.
\end{document}