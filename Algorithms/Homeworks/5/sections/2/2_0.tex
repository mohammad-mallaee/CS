\documentclass[]{article}
\usepackage{hyperref}
\usepackage[a4paper, total={6in, 8in}]{geometry}
\usepackage{caption}
\usepackage{algpseudocodex}
\usepackage{algorithm}
\usepackage{amsmath}
\usepackage{listings}
\usepackage{xepersian}
\usepackage{graphicx}
\usepackage{setspace}
\usepackage{subfiles}
\usepackage{xcolor}
\settextfont{XB Niloofar}

\begin{document}
مدیر یکی از انجمن‌های دانشجویی بزرگِ دانشگاه، با مسأله‌ای به سراغ شما آمده است.
او مسئول نظارت بر کار گروهی $n$ نفره از دانشجویان است
که هر یک از آنها طبق یک زمانبندی، باید یک نوبت در هفته، در انجمن کار کند.
کارهای مربوط به نوبت‌های کاری دانشجویان متفاوت است
(مانند حضور در پشت میز، کمک در تحویل بسته‌هایی و غیره) ،
اما می‌توانیم هر نوبت را به شکل یک بازه زمانی پیوسته ببینیم.
ممکن است چند نوبت کاری در یک زمان باشند.

مدیر می‌خواهد تعدادی از  دانشجوی انجمن را انتخاب کند تا با آنها یک هیأت ناظر تشکیل دهد
و با آنها جلسه‌های هفتگی داشته باشد. 
از نظر او، چنین هیأتی وقتی کامل خواهد بود که نوبت کاری هر دانشجویی که در هیأت نیست،
با نوبت کاری یکی از دانشجویانی که در هیأت است، تداخل (گرچه جزئی) داشته باشد.
بدین طریق، کارایی هر دانشجویی توسط حداقل یکی از افرادی که در هیأت حضور دارند، قابل مشاهده خواهد بود.

\paragraph*{الف}
الگوریتم کارایی را با شبه‌کد توصیف کنید که زمانبندی $n$ نوبت کاری دانشجویان را بگیرد 
و یک هیأت ناظرِ کامل تشکیل دهد که شامل کمترین تعداد دانشجو باشد.

مثلاً اگر  باشد و نوبتهای کاری دانشجویان :
\begin{itemize}
    \item دوشنبه 4 بعد از ظهر تا 8 بعد از ظهر
    \item و دوشنبه 6 بعد از ظهر تا 10 بعد از ظهر
    \item و دوشنبه 9 بعد از ظهر تا 11 بعد از ظهر
\end{itemize}
باشند، از آنجا که زمان نوبت کاری دوم، با زمان نوبتهای کاری اول و سوم، تداخل دارد،
کوچکترین هیأت ناظرِ کامل، شامل تنها دومین دانشجو خواهد بود.

\paragraph*{ب}
کارایی زمانی الگوریتم خود را اندازه بگیرید و درستی آن را نیز ثابت کنید؛
یعنی ثابت کنید که الگوریتم همیشه جواب بهینه مسأله را برمیگرداند.
\end{document}