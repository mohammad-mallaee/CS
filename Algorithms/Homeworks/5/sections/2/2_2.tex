\documentclass[]{article}
\usepackage{hyperref}
\usepackage[a4paper, total={6in, 8in}]{geometry}
\usepackage{caption}
\usepackage{algpseudocodex}
\usepackage{algorithm}
\usepackage{amsmath}
\usepackage{listings}
\usepackage{xepersian}
\usepackage{graphicx}
\usepackage{setspace}
\usepackage{subfiles}
\usepackage{xcolor}
\settextfont{XB Niloofar}

\begin{document}
کارایی زمانی الگوریتم در بدترین حالت که همه‌ی شیفت‌ها
باهم تداخل داشته باشند $O(n^3)$ خواهد بود.

فرض می‌کنیم الگوریتم جواب بهینه را به ما ندهد بنابراین
در مجوعه $S$ عضوی وجود دارد که عضو دیگری می‌تواند
مجموعه بازه‌هایی که این عضو پوشش می دهد را پوشش دهد که این با روند الگوریتم
تناقض دارد چرا که این عضو در الگوریتم حذف می‌شود پس به سادگی اثبات شد که
الگوریتم جواب بهینه را با می‌دهد.
\end{document}