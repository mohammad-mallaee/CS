\documentclass[]{article}
\usepackage{hyperref}
\usepackage[a4paper, total={6in, 8in}]{geometry}
\usepackage{caption}
\usepackage{algpseudocodex}
\usepackage{algorithm}
\usepackage{amsmath}
\usepackage{listings}
\usepackage{xepersian}
\usepackage{graphicx}
\usepackage{setspace}
\usepackage{subfiles}
\usepackage{xcolor}
\settextfont{XB Niloofar}

\begin{document}

فرض کنید یک بازی رایانه‌­ای به نام
\textbf{شکرستان}
تولید شده است که بازیکن آن در دنیایی سه بعدی حرکت می‌­کند.
مکان­‌های بازیکن در آن دنیای سه بعدی را خانه­‌های آرایه سه‌بعدی
$C$
با ابعاد
$n \times n \times n$
مشخص می­‌کنند و علاوه بر آن، مقدار هر خانه
$C[i, j, k]$
،تعداد امتیازاتی را که 
بازیکن شکرستان با بودن در مکان
$(i, j, k)$
به دست می­آورد، مشخص می­کند. در شروع بازی، تنها
$O(n)$
خانه از خانه‌­های آرایه­
$C$ 
غیر صفر است؛ همه
$O(n^3)$
خانه دیگر آن صفر هستند. 
اگر بازیکن در طول بازی، به مکان
$(i, j, k)$
برود و
$C[i, j, k]$
غیرصفر باشد، آنگاه به مقدار
$C[i, j, k]$
،
امتیاز به عنوان جایزه به او داده می­شود و سپس مقدار
$C[i, j, k]$
صفر می‌شود. در ادامه، رایانه مکان دیگری مثل
$(i, j, k)$
را تصادفاً انتخاب می­‌کند
و 100 امتیاز را به مقدار
$C[i, j, k]$
اضافه می‌­کند.

مسأله این است که بازی شکرستان، برای اجرا روی رایانه­‌ای بزرگ طراحی شده است
و چون حالا قرار است که آن را برای اجرا روی تلفن هوشمند 
(که حافظه بسیار کمتری دارد)
سازگار کنیم، شما نمی­توانید مانند نسخه رایانه­‌ای بازی، از
$O(n^3)$
خانه حافظه برای نمایش آرایه
$C$
در حافظه تلفن استفاده کنید.

توضیح دهید که چگونه می‌توانیم محتویات آرایه
$C$
را با استفاده از تنها
$O(n)$
واحد حافظه نمایش دهیم و اینکه چگونه می‌توانیم با الگوریتم‌هایی با کارایی
$O(\log n)$
\begin{itemize}
    \item مقدار هر خانه
    $C[i, j, k]$
    را تعیین کنیم؛
    \item مقدار هر خانه غیر صفر
    $C[i, j, k]$
    را صفر کنیم؛
    \item و 100 امتیاز را به مقدار هر خانه
    $C[i, j, k]$
    اضافه کنیم.
\end{itemize}
  
\end{document}