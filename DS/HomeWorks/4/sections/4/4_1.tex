\documentclass[]{article}
\usepackage{hyperref}
\usepackage[a4paper, total={6in, 8in}]{geometry}
\usepackage{caption}
\usepackage{algpseudocodex}
\usepackage{algorithm}
\usepackage{amsmath}
\usepackage{listings}
\usepackage{xepersian}
\usepackage{graphicx}
\usepackage{setspace}
\usepackage{subfiles}
\usepackage{xcolor}
\settextfont{XB Niloofar}

\begin{document}
از آنجایی که در هر لحظه $O(n)$ خانه در این بازی پر شده اند،
تنها خانه هایی که مقدار آن ها غیر صفر است را ذخیره می کنیم و با این کار می توانیم
دنیای بازی را در $O(n)$ واحد حافظه ذخیره کنیم.
با توجه به اینکه برای عملیات های این بازی به ساختار و الگوریتم های کارا نیازمندیم ، می توانیم از
درخت های خود متوازن استفاده کنیم که در این سوال، استفاده از درخت $AVL$
با توجه به شرایط مسئله مناسب است.
در این ساختار با توجه به متوازن بودن آن، کارایی الگوریتم های جستجو، حذف و اضافه کردن،
$O(logn)$
خواهد بود.

برای ذخیره سازی این دنیای سه بعدی که از مولفه های $i, j, k$ تشکیل شده است،
کافی است به هر یک از این خانه ها یک عدد یکتا نسبت دهیم تا بتوانیم با استفاده از آن، درخت $AVL$
مورد بحث را تشکیل دهیم.
اگر فرایند اندیس گذاری را از خانه $(1,1,1)$ شروع کنیم و در راستای محور $x$ حرکت کنیم،
خواهیم داشت :
$$
index(1,1,1) = 1, index(2, 1, 1) = 2, ..., index(n, 1, 1) = n
$$
سپس یک واحد به مولفه $j$ اضافه کرده و این روند را تکرار می کنیم.
برای خانه هایی که مولفه $j$ آنها $2$ است داریم:
$$
index(1,2,1) = 1 + n, index(2, 2, 2) = 2 + n, \ldots, index(n, 2, 2) = n + n
$$
با تکرار این روند و پوشش کامل خانه هایی که مولفه $k$ آنها $1$ و تعداد آنها $n^2$ است،
$k$ را یک واحد افزایش داده و این روند را تا انتها تکرار می کنیم.
بنابراین اندیس هر خانه را می توانیم به صورت تابعی از مولفه ها بنویسیم :
$$index(i, j, k) = i + n(j-1) + n^2(k-1)$$
بدین ترتیب با مجموعه اندیس و مقدار خانه های غیر صفر درخت دودویی مورد نظر را می سازیم که عملیات های آن
به صورت زیر تعریف شده اند:
\begin{itemize}
    \item {
        تعیین مقدار هر خانه: این کار را با جستجو در درخت انجام می دهیم و درصورت
        ناموفق بودن جستجو مقدار آن خانه را صفر برمی گردانیم
    }
    \item {
        صفر کردن مقدار خانه:
        با حذف کردن گره متناظر در درخت، مقدار خانه را صفر می کنیم.
    }
    \item {
        اضافه کردن امتیاز:
        ابتدا اندیس خانه را جستجو کرده و در صورت موفق بودن، مقدار آن را $100$
        واحد افزایش می دهیم در غیر این صورت یک گره به درخت اضافه می کنیم.
    }
\end{itemize}
\end{document}