\documentclass[]{article}
\usepackage{hyperref}
\usepackage[a4paper, total={6in, 8in}]{geometry}
\usepackage{caption}
\usepackage{algpseudocodex}
\usepackage{algorithm}
\usepackage{amsmath}
\usepackage{listings}
\usepackage{xepersian}
\usepackage{graphicx}
\usepackage{setspace}
\usepackage{subfiles}
\usepackage{xcolor}
\settextfont{XB Niloofar}

\begin{document}
برای پیدا کردن تعداد گره های پر درخت می توانیم به صورت بازگشتی تعداد گره های فرزندان
چپ و راست گره را پیدا کرده و در صورت پر بودن گره ریشه آن را یک واحد اضافه می کنیم تا
تعداد همه ی گره های پر را به دست آوریم. (طبق تعریف گره پر یا دو فرزند دارد یا فرزند ندارد)
\begin{latin}
    \begin{algorithm}[H]
        \caption*{CountFullNodes($P$)}
        \begin{algorithmic}
            \Require Pointer p to tree T
            \Ensure Count of full nodes in the tree
            \State get tree T pointed by P
            \State count = 0
            \If{T is empty}
                \State \Return 1
            \Else
                \State $countLeft = CountFullNodes(T_{left})$
                \State $countRight = CountFullNodes(T_{right})$
                \If{T has left and right child}
                \State count += 1
                \EndIf
                \State count = count + countLeft + countRight
                \State \Return count
            \EndIf
        \end{algorithmic}
    \end{algorithm}
\end{latin}
\begin{latin}
\begin{lstlisting}[language=Python, caption=Python Implementation]
def count_full_nodes(root):
    count = 0
    
    if not root.left and not root.right:
        return 1

    if root.left:
        count += count_full_nodes(root.left)
    if root.right:
        count += count_full_nodes(root.right)

    if root.right and root.right:
        count += 1

    return count
\end{lstlisting}
\end{latin}
الگوریتم، مسئله را به دو مسئله کوچکتر تبدیل می کند و پس از حل آنها با انجام یک مقایسه و جمع
آنها را باهم ترکیب می کند بنابراین اگر $T(n)$ نشان دهنده زمان اجرای الگوریتم باشد داریم :

$$T(n) = 2\;T(n/2) + \theta(1)$$

طبق قضیه اصلی، زمان اجرای الگوریتم $\theta(n)$ خواهد بود.
\end{document}