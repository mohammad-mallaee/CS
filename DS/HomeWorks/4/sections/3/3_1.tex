\documentclass[]{article}
\usepackage{hyperref}
\usepackage[a4paper, total={6in, 8in}]{geometry}
\usepackage{caption}
\usepackage{algpseudocodex}
\usepackage{algorithm}
\usepackage{amsmath}
\usepackage{listings}
\usepackage{xepersian}
\usepackage{graphicx}
\usepackage{setspace}
\usepackage{subfiles}
\usepackage{xcolor}
\settextfont{XB Niloofar}

\begin{document}
حالت پایه الگوریتم، آرایه ای به طول ۲ است که به درستی مرتب می شود. بنابر استقرا ریاضی
فرض می کنیم که الگوریتم آرایه ای به طول $n - 1$ را به درستی مرتب می کند.
نشان می دهیم که الگوریتم آرایه به طول $n$ را نیز مرتب خواهد کرد.

آرایه ای به طول n مانند $A = [a_0, a_1, \ldots, a_{n-1}]$
را در نظر می گیریم و با استفاده از الگوریتم دو سوم ابتدایی آرایه را مرتب می کنیم و داریم:
$$
m = \lfloor \frac{n}{3} \rfloor \rightarrow A = [a_0, \ldots, a_{n-m}, \ldots, a_n] ; a_0 < \ldots < a_m < \ldots < a_{n-m}
$$

در مرحله ی بعدی دو سوم پایانی آرایه مرحله قبل را مرتب می کنیم و داریم :
$$
m = \lfloor \frac{n}{3} \rfloor \rightarrow A = [a_0, \ldots, a_{m}, a'_{m+1}, \ldots, a'_n] ; a'_{m+1} < \ldots < a'_{n-m} < \ldots < a'_n
$$

در این مرحله اطمینان داریم که یک سوم پایانی این آرایه در جای درست خود قرار گرفته.
برای اثبات این گزاره از برهان خلف استفاده می کنیم و فرض می کنیم که در یک سوم ابتدایی عددی وجود دارد که
حداقل از یکی از اعداد یک سوم پایانی بزرگ تر است
(از اعداد ۱ و ۳ برای نشان دادن یک سوم ابتدایی و پایانی آرایه به دست امده در مرحله دوم استفاده شده است):
$$
\exists x \in 1 \; \; \exists y \in 3 : x > y \xrightarrow{x \le a_m} \exists y \in 3: a_m > y
\xrightarrow{a_m < a_{m+1}} \exists y \in 3 : y < a_{m+1} < ... < a_{n-m} \rightarrow y \not \in 3
$$
نتیجه می گیریم $y$ از یک سوم میانی آرایه پس از مرحله اول کوچکتر است
و پس از مرحله دوم باید قبل از همه آن ها قرار بگیرد پس $y$ نمی تواند بعد از مرحله دوم
در یک سوم پایانی باشد
که این تناقض است بنابر این حکم درست است و اعداد در یک سوم پایانی به درستی در جای خود قرار گرفته اند.

در مرحله سوم دوباره دو سوم ابتدایی آرایه را مرتب می کنیم و این باعث مرتب شدن تمام آرایه خواهد شد. بنابر استقرا
درستی این الگوریتم برای آرایه های به طول $n$ که $n$
عددی صحیح و مثبت است اثبات شد.
\end{document}