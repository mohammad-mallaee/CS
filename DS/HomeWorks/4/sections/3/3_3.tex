\documentclass[]{article}
\usepackage{hyperref}
\usepackage[a4paper, total={6in, 8in}]{geometry}
\usepackage{caption}
\usepackage{algpseudocodex}
\usepackage{algorithm}
\usepackage{amsmath}
\usepackage{listings}
\usepackage{xepersian}
\usepackage{graphicx}
\usepackage{setspace}
\usepackage{subfiles}
\usepackage{xcolor}
\usepackage{mathtools}
\settextfont{XB Niloofar}

\begin{document}
بد ترین حالت در این الگوریتم زمانی اتفاق می افتد که آرایه نزولی باشد.
در مرحله دوم همین اتفاق برای عناصر یک سوم میانی و یک سوم پایانی رخ خواهد داد و در آخر همین اتفاق
برای یک سوم ابتدایی و میانی می افتد. بدین ترتیب هر دو عنصر ای آرایه دقیقا یکبار
باهم جابجا می شوند که داریم :
$$\binom{n}{2} = \frac{n(n-2)}{2}$$
بنابراین حداکثر تعداد جابجایی های الگوریتم $\frac{n(n-1)}{2}$ خواهد بود.
\end{document}