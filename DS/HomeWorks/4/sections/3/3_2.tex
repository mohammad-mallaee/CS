\documentclass[]{article}
\usepackage{hyperref}
\usepackage[a4paper, total={6in, 8in}]{geometry}
\usepackage{caption}
\usepackage{algpseudocodex}
\usepackage{algorithm}
\usepackage{amsmath}
\usepackage{listings}
\usepackage{xepersian}
\usepackage{graphicx}
\usepackage{setspace}
\usepackage{subfiles}
\usepackage{xcolor}
\usepackage{mathtools}
\settextfont{XB Niloofar}

\begin{document}
برای حل، این مسئله به سه مسئله کوچکتر با اندازه های $2/3$ مسئله اصلی تبدیل می شود
که پس از حل شدن آن ها مسئله اصلی بدون نیاز به ترکیب جواب ها حل می شود.
بنابراین اگر $T(n)$ را زمان اجرای الگوریتم در نظر بگیریم، خواهیم داشت :
$$T(n) = 3T(2/3n) + \theta(1)$$
طبق قضیه اصلی کارایی الگوریتم $\theta(n^{log_{1.5}{3}}) = \theta(n^{2.7})$ است.
\end{document}