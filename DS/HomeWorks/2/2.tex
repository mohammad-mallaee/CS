\documentclass{article}

\usepackage{algpseudocodex}
\usepackage{algorithm}
\usepackage{amsmath}
\usepackage{listings}
\usepackage{xepersian}

\setlength{\textwidth}{5.5in}


\definecolor{codegreen}{rgb}{0,0.6,0}
\definecolor{codegray}{rgb}{0.5,0.5,0.5}
\definecolor{codepurple}{rgb}{0.58,0,0.82}
\definecolor{backcolour}{rgb}{0.95,0.95,0.92}

\lstdefinestyle{mystyle}{
    backgroundcolor=\color{backcolour},   
    commentstyle=\color{codegreen},
    keywordstyle=\color{magenta},
    numberstyle=\tiny\color{codegray},
    stringstyle=\color{codepurple},
    basicstyle=\ttfamily\footnotesize,
    breakatwhitespace=false,         
    breaklines=true,                 
    captionpos=b,                    
    keepspaces=true,                 
    numbers=left,                    
    numbersep=5pt,                  
    showspaces=false,                
    showstringspaces=false,
    showtabs=false,                  
    tabsize=2
}

\lstset{style=mystyle}

\algrenewcommand{\algorithmicrequire}{\textbf{Input:}}
\algrenewcommand{\algorithmicensure}{\textbf{Output:}}

\settextfont{Vazirmatn}
\date{\today}
\title{تمرینات ۲}
\author{محمد ملائی، محسن محمودآبادی، داوود نصرتی امیرآبادی}
\begin{document}
\begin{center}
    {محمد ملائی،  محسن محمودآبادی،  داوود نصرتی امیرآبادی}
\end{center}
\pagebreak
\section{تمرین}
\subsection{الف}
sometimes
الگوریتمی که در صورت سوال به عنوان نمونه و ورودی داده شده است تنها به یک تغییر جزئی در 
مقدار خروجی خود دارد تا به جواب بخش الف تمرین دست پیدا کند. در واقع چون الگوریتم اندیس نویسه ی اول رشته را 
برمی گرداند، با داشتن طول نویسه می توان به آخرین نویسه ی رشته دست پیدا کرد و اندیس آن را 
به عنوان خروجی بازگرداند.
\begin{latin}
    \begin{algorithm}
        \caption{Last-Index}
        \begin{algorithmic}
            \Require An array T[0...n-1] of n characters representing a text
            and an array $P[0...m-1]$ representing a pattern
            \Ensure The index of last character in text that ends a matching substring or -1 if the search is unsuccessful
            \For{$ i = 0 \; to \; n-m$}
                \State j = 0
                \While {$j < m \; and \; P[j] = T[i+j]$}
                    \State j = j + 1
                \EndWhile
                \If{$j=m$}
                    \State \Return $ i + m - 1$
                \EndIf
            \EndFor
            \State \Return -1
        \end{algorithmic}
    \end{algorithm}
\end{latin}
\subsection{ب}
با توجه به خواسته ی مسئله،
ابتدا به اندازه ی طول الگو به سمت راست حرکت کرده و سپس یکی یکی به سمت چپ
حرکت می کنیم.  در پایان در صورت عدم موفقیت آمیز بودن جستجو، از نویسه ی بعدی پیمایش را آغاز می کنیم.
\begin{latin}
    \begin{algorithm}
        \caption{RTL-Check}
        \begin{algorithmic}
            \Require An array T[0...n-1] of n characters representing a text
            and an array $P[0...m-1]$ representing a pattern
            \Ensure The index of last character in text that ends a matching substring or -1 if the search is unsuccessful
            \For{$ i = 0 \; to \; n-m$}
                \State j = m - 1
                \While {$j > 0\; and \; P[j] = T[i+j]$}
                    \State j = j - 1
                \EndWhile
                \If{$j=0$}
                    \State \Return $ i $
                \EndIf
            \EndFor
            \State \Return -1
        \end{algorithmic}
    \end{algorithm}
\end{latin}

\pagebreak
\subsection{پ}
در اینجا متغیری به نام 
$count$
تعریف می کنیم و بار هر بار شناسایی کامل الگو، یک واحد آن را افزایش می دهیم.
در ضمن، پس از یافتن الگو به طور کامل، به اندازه ی طول الگو به سمت جلو حرکت می کنیم و 
پیمایش را از آنجا آغاز می نماییم تا از یافتن الگوهای دارای اشتراک جلوگیری کرد.
\begin{latin}
    \begin{algorithm}
        \caption{Count}
        \begin{algorithmic}
            \Require An array T[0...n-1] of n characters representing a text
            and an array $P[0...m-1]$ representing a pattern
            \Ensure Number of pattern occurrences in text
            \State i = 0
            \State count = 0
            \While{$ i < n-m$}
                \State j = m - 1
                \While {$j > 0\; and \; P[j] = T[i+j]$}
                    \State j = j - 1
                \EndWhile
                \If{$j=0$}
                    \State $ i = i + m$
                    \State $ count = count + 1$
                \EndIf
            \EndWhile
            \State \Return count
        \end{algorithmic}
    \end{algorithm}
\end{latin}

\subsection{ت‌}
با توجه به اینکه کارایی زمانی الگوریتم باید
O(n)
باشد ، باید نهایتا از یک حلقه استفاده کنیم که تمام کاراکتر های متن را فقط یک بار بررسی می کند.
در این الگوریتم، متن و الگو همزمان و با یک متغیر ( i )
بررسی می شوند و با هر بار مقایسه j افزایش می یابد
که نشانگر تطابق آن بخش از رشته و الگوست که رسیدن آن به اندازه الگو به معنی یافتن زیر رشت های یکسان با الگو است.
 پس از یافتن تطابق مکان آخرین کاراکتر این زیر رشته به عنوان خروجی الگوریتم برگردانده می شود.
\begin{latin}
    \begin{algorithm}
        \caption[short]{Linear-Index}
        \begin{algorithmic}
            \Require An array T[0...n-1] of n characters representing a text
            and an array $P[0...m-1]$ representing a pattern
            \Ensure The index of last character in text that ends a matching substring or -1 if the search is unsuccessful
            \State j = 0
            \For{$ i = 0 \; to \; n$}
                \If{T[i] = P[j]}
                    \State j = j + 1
                \Else
                    \State j = 0
                \EndIf

                \If {j = m}
                    \State \Return i
                \EndIf
            \EndFor
            \State \Return -1
        \end{algorithmic}
    \end{algorithm}
\end{latin}

\pagebreak

\section{تمرین}
\subsection{الف}
از آنجایی که تعداد اعمال انتساب و تقسیم با هم برابر است تعداد تکرار عمل انتساب را در نظر می گیریم
برای محاسبات تعداد مراحل انجام شده و از آنجا که 
$m \,\, mod \,\, n$
می تواند حداکثر نصف 
$n$
باشد و در بدترین حالت کارایی زمانی برابر است با :
$$\gcd(m, n) = \gcd(n, \frac{n}{2}) = \gcd(\frac{n}{2}, \frac{n}{4}) =... = \gcd(\frac{n}{n^{n-2}}, \frac{n}{n^{n-1}})$$
که 
$\frac{n}{n^{n-1}} = 1$
و حلقه ی
while
به پایان می رسد.
پس داریم:
$$n-1 = \log_2^n \Rightarrow n = \log_2^n + 1 \Rightarrow \lim_{n \rightarrow \infty} \frac{\log_2^n + 1}{\log_2^n} = 1 \Rightarrow \log_2^n + 1 \in O(\log_2^n)$$

\subsection{ب}
برای ثابت کردن این رابطه بازگشتی، ما می‌توانیم از اصل استقرا استفاده کنیم. برای این کار، 
ابتدا باید ثابت کنیم که این رابطه بازگشتی برای دو عدد صحیح مثبت درست است. سپس، فرض کنید که
این رابطه بازگشتی برای 
$n-1$
عدد صحیح مثبت درست است. بنابراین،
\begin{align}
    \gcd(a_1, a_2, \ldots, a_n) &= \gcd(\gcd(a_1, \ldots, a_{n-1}), gcd(a_n)) \\
                            &= \gcd(\gcd(\gcd(a_1, \ldots, a_{n-2}), a_{n-1}), a_n)\\
                            &=\ldots \\
                            &= \gcd(\gcd(...(\gcd(a_1, a_2), a_3), a_3), \ldots, a_{n})
\end{align}
که این رابطه بازگشتی بدون توجه به ترتیب اعداد ورودی، همیشه درست است.

الگوریتم پایین به محاسبه 
$\gcd$
دو عدد می پردازد.

\subsection{پ}
\begin{latin}
    \begin{algorithm}[H]
        \caption{gcd(a, b)}
        \begin{algorithmic}
            \Require a, b are two integers to find gcd
            \Ensure Greatest Common Devisor of a and b
                \While{y != 0}
                    \State x, y = y, (x $mod$ y)
                \EndWhile
                \State \Return x
        \end{algorithmic}
    \end{algorithm}
\end{latin}


الگوریتم پایین با کمک الگوریتمی که در بالا معرفی شد،
در هر مرحله در صورت وجود لیستی با طول بیش از ۲،
به محاسبه ی 
$\gcd$
\underline{عضو اول لیست}
و
\underline{باقی اعضای لیست}
می پردازد و مقدار حاصل را بر می گرداند.
\begin{latin}
    \begin{algorithm}[H]
        \caption{super\_gcd($A[a_0, a_1, \ldots, a_{n-1}$])}
        \begin{algorithmic}
            \Require A: A list of n integers
            \Ensure gcd of all the elements of A
                \If{len(a) == 2}
                    \State \Return gcd(a[0], a[1])
                \Else 
                    \State \Return $gcd(a[0], super\_gcd(A[a_1, a_2, \ldots, a_{n-1}]))$
                \EndIf
        \end{algorithmic}
    \end{algorithm}
\end{latin}

با توجه به اینکه کارایی الگوریتم 
$\gcd$
برابر با 
$\log_2(n)$
است و هربار یک عدد از لیست ما کاسته می شود، 
و این کار 
$n$
بار انجام می شود،
پس کارایی زمانی الگوریتم ما از مرتبه ی 
$n\log_2(n)$
می باشد.
یا به عبارتی:
    \begin{align}
        M(2) = \log_2(n)\\
        M(n) = \\
        &= \log_2(n) + M(n - 1) \\
        &= 2\log_2(n) + M(n - 2) \\
        &= i \log_2(n) + M(n - i) \\
        &= (n-1)log_2(n) \\
        \rightarrow M(n) \in \Theta(n.log_2(n))
    \end{align}

\section{تمرین}
\subsection{الف}
برای محاسبه جواب این مسئله به روش ساده اندیشانه، ابتدا تمام زیر مجموعه های مجموعه S را ساخته و سپس
بزرگترین زیرمجموعه ای که در شرط مسئله صدق کند را به عنوان جواب مسئله انتخاب می کنیم یعنی :

$$Space = \{A \in S| \not\exists 1 \le j \le n : A_j \le A_i \le F_j \}$$

بزرگترین عضو مجموعه Space
از نظر تعداد عضو، جواب مسئله خواهد بود.

\begin{latin}
    \begin{algorithm}
        \caption{GreatestSubset}
        \begin{algorithmic}
            \Require Sets S = $\{s_1, s_2, ..., s_n\}$ and F = $\{ f_1, f_2, .., f_n\}$
            \Ensure Set A = $\{a_1, a_2, ..., a_m\}$ which is a subset of S
            \State A = $\{\}$
            \For{$X = \{x_1, x_2, .., x_q\}$ in Subsets of \{1, 2, ..., n\}}
                \If {q > m}
                    \State iEqual = True
                    \For{$i = 1$ to $q$}
                        \State jEqual = True
                        \For{$j = 1$ to $q$}
                            \If {$x_j \not = x_i \; And \; s_{x_j} \le s_{x_i} \leq f_{x_j} $}
                                \State jEqual = False
                                \State break
                            \EndIf
                        \EndFor
                        \If{$jEqual = False$}
                            \State iEqual = False
                            \State break
                        \EndIf
                    \EndFor
                    \If{iEqual = True}
                        \State A = X
                    \EndIf
                \EndIf
            \EndFor
            \Return A
        \end{algorithmic}
    \end{algorithm}
\end{latin}
\pagebreak
\subsection{ب}
برای پیدا کردن جواب این مسئله به $2^n$
زیر مجموعه از S
نیازمندیم که برای بررسی شرط مسئله برای هرکدام از آنها، دو حلقه For
که هر یک سه مقایسه انجام می دهند. بنابراین با فرض اینکه هر کارایی زمانی ساخت هر زیرمجموعه
$\Theta(1)$
باشد داریم :

\begin{align}
    M(n) &= \sum_{i=1}^{n} {\binom{n}{i}} \sum_{j=1}^{i} \sum_{k=1}^{n} 3 \\
    &= \sum_{i=1}^{n} {\binom{n}{i}} \sum_{j=1}^{i} 3n \\
    &= \sum_{i=1}^{n} {\binom{n}{i}} 3(i)(n) \\
    &= {3n}\sum_{i=1}^{n} { (i) \binom{n}{i}} \\
    &= 3n^2 * 2^{n-1} \\
    &\Rightarrow M(n) \in O(n^2*2^n)
\end{align}

\pagebreak

\section{تمرین}
\subsection{الف - ساختار داده}
با اجرای الگوریتم و دادن دنباله به عنوان ورودی تابع get\_data\_structe
، خروجی یک ماتریس $n \times n$
خواهد بود که درایه $ij$ آن کوچکترین مقدار محدوده بین i و j خواهد بود.

\begin{latin}
\begin{lstlisting}[language=Python, caption=Python Implementation]
def get_min_of_slice(list, start, end):
    minimum = list[start]
    for k in range(start, end + 1):
        minimum = list[k] if list[k] < minimum else minimum
    return minimum

def get_data_structure(list):
    list_size = len(list)
    Matrix = [[0] * list_size for _ in range(list_size)]
    for i in range(0, len(list)):
        for j in range(i + 1):
            min_ij = get_min(list, j, i)
            Matrix[j][i] = min_ij
            Matrix[i][j] = min_ij
    return Matrix
\end{lstlisting}
\end{latin}

\subsection{ب - تحلیل کارایی}
M(n) را تعداد مقایسه های الگوریتم در نظر می گیریم
و داریم :
\begin{align}
    M(n) &= \sum_{i=0}^{n}\sum_{j=0}^{i}\sum_{k = j}^{i} 1 \\
    &= \sum_{i=0}^{n}\sum_{j=0}^{i} i-j+1 \\
    &= \sum_{i=0}^{n} \left( \sum_{j=0}^{i}i - \sum_{j=0}^{i}j + \sum_{j=0}^{i} 1  \right) \\
    &= \sum_{i=0}^{n} \left( i^2 - \frac{i(i+1)}{2} + (i + 1) \right) \\
    &= \sum_{i=0}^{n} \frac{1}{2}i^2 + \frac{1}{2}i + 1 \\
    &= \frac{1}{2}\sum_{i=0}^{n} i^2 + \frac{1}{2} \sum_{i=0}^{n}i + \sum_{i=0}^{n}1 \\
    &= \frac{1}{12}n(n+1)(2n+1) + \frac{1}{2}n(n+1) + n+1 \\
    &= \frac{1}{6}n^3 + \frac{3}{4}n^2 + \frac{19}{12}n + 1 \\
    &\Rightarrow M(n) \in \Theta(n^3)
\end{align}


\pagebreak
\section{تمرین}
الگوریتم را به سه بخش تقسیم می‌کنیم. ابتدا لیست
$m$
تایی ورودی را به لیست مجاورت تبدیل کرده و سپس به جستجوی عمیق در گراف پرداخته، در نهایت با اجرای پیمایش عمیق به یافتن جواب می‌پردازیم.

برای ساخت لیست مجاورت، یک آرایه‌ی
$n$
عضوی می‌سازیم. 
در اندیس
$0$
ام آرایه، آرایه هایی به شکل
$[v, e]$
قرار می‌دهیم که در آن 
$v$
نشان دهنده کامپیوتری است که کامپیوتر صفرم به آن متصل است و
$e$
زمانی است که در آن، کامپیوتر صفرم به کامپیوتر
$v$ ام 
متصل می‌شود.

برای انجام این کار، در لیست ورودی پیمایش می‌کنیم.
\begin{latin}
    \begin{algorithm}[H]
    \caption{graph\_adjacency(connections)}
        \begin{algorithmic}
            \Require a connection list, containing $m$ lists of [$C_i$, $C_j$, $t_k$]
            \Ensure An Adjacency list of the graph
            \State m = len(connections)
            \State adjacency = [[\, \,] for \_ in range(m)]
            \For {$C_i$, $C_j$, $t_k$ in graph}
                \State adjacency[$C_i$ - 1].append([$C_j - 1$, $t_k$])
                \State adjacency[$C_j$ - 1].append([$C_i - 1$, $t_k$])
                
            \EndFor
            \State return $adjacency$
        \end{algorithmic}
    \end{algorithm}
\end{latin}



الگوریتم زیر، نحوه‌ی انجام پیمایش عمیق در لیست مجاورت را نشان می‌دهد.
لازم به ذکر است که در هر مرحله از انجام پیمایش، لازم است تا سه شرط مورد بررسی قرار گیرند.
\begin{enumerate}
    \item 
    راسی که قصد مشاهده آن را داریم در لیست راس های آلوده نباشد.
    \item 
    زمان ارتباط با راس بعدی، بیشترمساوی زمان شیوع ویروس و کمترمساوی زمان پایان بررسی باشد.
    \item 
    زمان ارتباط با راس بعدی، حتما بیشترمساوی زمان ارتباط راس قبلی با راس فعلی باشد.‌
\end{enumerate}

\begin{latin}
    \begin{algorithm}[H]
    \caption{dfs(graph, start\_node, last\_traveled\_time, infected\_devices)}
        \begin{algorithmic}
            \Require $graph$: The adjacency list of the graph, $start\_node$: The start of search, $last\_traveled\_time$: The time at which the $start\_node$ has been infected, $infected\_computers$: The list of infected computers by the time of the search
            \Ensure Returns None, but modifies the $infected\_computers$ list
            
            \For{$neighbor$ in $graph$[$start\_node$]}
                \State $neighbor\_id$, $neighbor\_time$ = $neighbor$
                \State $con\_1$ = $neighbor\_id$ not in $infected\_computers$
                \State con\_2 = $x \leq neighbor\_time \leq y$
                \State $con\_3$ = $neighbor\_time \geq last\_traveled\_time$
                    \If{con\_1 and con\_2 and con\_3}
                        \State infected\_computers.add(neighbor\_id)
                        \State dfs(graph, neighbor\_id, neighbor\_time)
                    \EndIf
            \EndFor
        \end{algorithmic}
    \end{algorithm}
\end{latin}

                    


در نهایت با ایجاد یک لیست از کامپیوتر های آلوده، بررسی می‌کنیم که آیا کامپیوتر 
$j$
ام در لیست وجود دارد یا خیر.

\begin{latin}
    \begin{algorithm}[H]
    \caption{is\_virus\_there$graph, C_i, C_j, x, y$)}
        \begin{algorithmic}
            \Require $graph$: The adjacency list of the graph, $C_i$: The first computer infected, $C_j$: The target computer, $x$: The time of first infection, $y$: The end of check time
            \Ensure True if $C_j$ is infected, False otherwise
            
            \State infected\_computers = [\, \,]
            
            \State dfs(graph, $C_i$, 0, infected\_computers)
                
            \State $C_j$ in infected\_computers
        \end{algorithmic}
    \end{algorithm}
\end{latin}

\end{document}