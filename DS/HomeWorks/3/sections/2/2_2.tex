\documentclass{article}
\usepackage{caption}
\usepackage{caption}
\usepackage{algpseudocodex}
\usepackage{algorithm}
\usepackage{amsmath}
\usepackage{xepersian}
\settextfont{XB Niloofar}

\begin{document}
    قسمت ب را نیز مشابه با قسمت الف می توانیم حل کنیم، با این تفاوت که پیمایش
    گراف را از راس نشان دهنده شهرداری شروع کرده و مولفه همبندی گراف را که شامل این راس است را پیدا کرده و سپس
    با پیمایش گراف ترانهاده آن، قویا همبند بودن آن را بررسی می کنیم.
    \begin{latin}
        \begin{algorithm}[H]
            \caption{GetConnectedComponent}
            \begin{algorithmic}
                \Require Graph G = <V, E> which is traversed by dfs algorithm, Vertex v
                \Ensure Connected Component of G which includes v
                \State connectedComponent = $<V_2, E_2>$
                \For{each vertex v in V}
                    \If{v is marked with 0}
                        \State add $v$ to $V_2$ and its edges to $E_2$
                    \EndIf
                \EndFor
                \State \Return connectedComponent
                
            \end{algorithmic}
        \end{algorithm}
    \end{latin}

    \begin{latin}
        \begin{algorithm}[H]
            \caption{IsStronglyConnected}
            \begin{algorithmic}
                \Require Graph G = <V, E>, Vertex v
                \Ensure Boolean representing if the connected component of G which includes v is strongly connected or not
                \State $G_1$ = dfs($G, v$)
                \State $G_2$ = GetConnectedComponent($G_1, v$)
                \State $G'$ is transposed graph of $G_2$
                \State $G_3$ = dfs($G'$, $v$)
                \For{each vertex w in $V_3$}
                \If{v is marked with 0}
                    \State return false
                \EndIf
                \EndFor
                \\
                \State \Return true
            \end{algorithmic}
        \end{algorithm}
    \end{latin}

    کارایی زمانی این الگوریتم نیز مانند قسمت قبل خطی خواهد بود.



\end{document}