\documentclass{article}
\usepackage{caption}
\usepackage{caption}
\usepackage{algpseudocodex}
\usepackage{algorithm}
\usepackage{amsmath}
\usepackage{xepersian}
\settextfont{XB Niloofar}

\begin{document}

برای حل این مسئله می توانیم خیابان ها و تقاطع های شهر را مانند رئوس و یال های گراف در نظر بگیریم.
با توجه به یکطرفه بودن خیابان ها، گراف مورد نظر جهت دار خواهد بود و برای اینکه نشان دهیم بین هر دو راس،
مسیری وجود دارد باید نشان دهیم که این گراف قویا همبند است.
\begin{latin}
    \begin{algorithm}[H]
        \caption*{dfs}
        \begin{algorithmic}
            \Require Graph G = <V, E> and v, a vertex of G
            \Ensure Marks all the vertices connected to v with 1
            \State mark v with 1
            \For{each vertex w in V adjacent to v}
                \If{w is marked with 0}
                    \State dfs(w)
                \EndIf
            \EndFor
        \end{algorithmic}
    \end{algorithm}
\end{latin}

\begin{latin}
    \begin{algorithm}[H]
        \caption*{IsStronglyConnected}
        \begin{algorithmic}
            \Require Graph G = <V, E>
            \Ensure Boolean representing if the graph is strongly connected or not
            \State mark each vertex in V with 0 as a mark of being "unvisited"
            \State $G_1$ = dfs(G, $V_0$)
            \For{each vertex v in $V_1$}
                \If{v is marked with 0}
                    \State return false
                \EndIf
            \EndFor
            \\
            \State $G'$ is transposed graph of G
            \State $G_2$ = dfs($G'$, $V_0$)
            \For{each vertex w in $V_2$}
            \If{v is marked with 0}
                \State return false
            \EndIf
            \EndFor
            \\
            \State return true
        \end{algorithmic}
    \end{algorithm}
\end{latin}

الگوریتم اول، تمام راس های متصل به ورودی را پیمایش می کند.
در الگوریتم دوم ابتدا از یکی از راس های گراف شروع کرده و تمام راس های متصل به آن را
علامت گذاری می کنیم. اگر تمام رئوس علامت داشته باشند، آن گاه حداقل یک مسیر بین راس انتخاب شده و بقیه رئوس وجود دارد،
در غیر اینصورت این گراف قویا همبند نیست.

حال باید نشان دهیم که از هر راس دیگر گراف به راس انتخاب شده نیز یک مسیر وجود دارد.
این کار را با بررسی وجود مسیر بین راس انتخاب شده و رئوس گراف ترانهاده انجام می دهیم.
به عبارتی اگر در گراف ترانهاده بین راس انتخاب شده و هر راس دیگر یک مسیر وجود داشته باشد پس در گراف اصلی،
یک مسیر از هر راس به راس انتخاب شده وجود خواهد داشت.

مانند قبل گراف ترانهاده را از راس انتخاب شده پیمایش می کنیم و تمام رئوس مجاور آن را به صورت بازگشتی علامت گذاری می کنیم،
اگر همه گراف علامت گذاری شده باشد پس می توانیم نتیجه بگیریم که گراف قویا همبند است.

این الگوریتم از دو جستجوی عمقی و دو حلقه استفاده می کند، پس کارایی زمانی آن
$$O(4|V| + 2|E|) = O(|V| + |E|)$$
خواهد بود.