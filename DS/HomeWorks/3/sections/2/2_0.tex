\documentclass{article}
\usepackage{caption}
\usepackage{algpseudocodex}
\usepackage{algorithm}
\usepackage{amsmath}
\usepackage{xepersian}
\settextfont{XB Niloofar}

\begin{document}
پلیس شهر رایانستان، همه خیابانهای شهر را یک طرفه کرده است. با وجود این، 
شهردار رایانستان ادعا می‌کند که هنوز راهی برای رانندگی قانونی از هر تقاطعی در شهر به هر تقاطع دیگر در آن، وجود دارد.
 چون مخالفان شهردار قانع نمیشوند و شهر هم بسیار بزرگ است،
 برای تعیین درستی یا نادرستی ادعای شهردار، برنامه‌ای رایانه ای لازم است. از آنجا که انتخابات شورای شهر به زودی برگزار خواهد شد،
 شهردار خدمتگزار، که نگران از دست دادن فرصت عظیم خدمت به شهروندان است، برای حل مسأله‌اش به سراغ شما آمده است.
 شما میدانید که با توجه به بزرگی شهر و زمان اندکی که شهردار دارد، تنها باید به دنبال الگوریتمی خطی برای اثبات درستی ادعای شهردار بود.

\textbf{الف)}
 مسأله شهردار را به شکل یک مسأله گراف بیان کنید
 و آنگاه الگوریتمی برای حل آن ارائه کنید که زمان اجرای آن (برحسب تعداد رأس‌ها و تعداد یال‌های گراف ورودی) 
 خطی باشد. الگوریتم خود را با شبه کد توصیف کنید.

 \textbf{ب)} 
شهردار باهوش رایانستان، به این هم فکر کرده است که در صورتی که با الگوریتم شما،
 نادرستی ادعای او مشخص شد، ادعای ضعیفتری را طرح کند:
 اینکه اگر شما از ساختمان شهرداری رانندگی را شروع کنید، خیابان‌های یک طرفه را بپیمایید و به هر جایی که ممکن بود برسید، 
 باز همیشه راهی برای آنکه به طور قانونی رانندگی کنید تا به ساختمان شهرداری برگردید، خواهید داشت. \\
 این ادعای ضعیفتر شهردار را هم به شکل یک مسأله گراف بیان کنید و باز الگوریتمی خطی ارائه کنید
 که با آن بتوان درستی یا نادرستی ادعای او را تحقیق کرد.
 الگوریتم خود را با شبه کد توصیف کنید.
 

\end{document}