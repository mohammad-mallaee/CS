\documentclass{article}
\usepackage{caption}
\usepackage{algpseudocodex}
\usepackage{algorithm}
\usepackage{amsmath}
\usepackage{xepersian}
\settextfont{XB Niloofar}

\begin{document}

آرایه 
$A[0\cdots n-1]$
 شامل «عنصر غالب» است، اگر بیشتر از 
$\lfloor\frac{n}{2}\rfloor$
عنصر در آن، یکسان باشند. الگوریتمی
کارا طراحی کنید که آرایه ای را به عنوان ورودی بگیرد و مشخص کند که آیا آرایه، عنصر غالب دارد یا خیر، و در 
صورت وجود عنصر غالب، آن را تعیین کند. در حالت کلی، عناصر آرایه، لزوماً اشیاء ترتیب پذیر مانند اعداد 
صحیح نیستند و به همین دلیل، نمیتوانید برای حل مسأله، از مقایسه هایی به شکل
 «آیا 
 $A_i>A_j$
  است؟» 
استفاده کنید. (مثلاً تصور کنید که عناصر آرایه، فایلهای تصویری GIF باشند.) اما شما میتوانید در الگوریتم از 
پرسشهایی به شکل «آیا 
$A_i=A_j$
 است؟» استفاده کنید.
\vspace{2.5pt}

\textbf{الف)}
الگوریتمی ساده اندیشانه
برای این مسأله طراحی کنید و آن را با شبه کد توصیف کنید. کارایی زمانی الگوریتم‌تان باید 
$\Theta(n^2)$
باشد و کارایی فضایی آن
$\Theta(1)$.

\textbf{ب)}
الگوریتمی تقلیل و حل برای این مسأله طراحی کنید
و آن را با شبه کد توصیف کنید.
کارایی زمانی الگوریتم‌تان باید 
$\Theta(n)$
باشد و کارایی فضایی آن
$\Theta(1)$.
\end{document}