\documentclass[]{article}
\usepackage{algpseudocodex}
\usepackage{algorithm}
\usepackage{amsmath}
\usepackage{listings}
\usepackage{xepersian}
\usepackage{graphicx}
\usepackage{setspace}
\settextfont{XB Niloofar}

\begin{document}

ابتدا رابطه ترتیب را برای دو جایگشت P و $P'$
تعریف می کنیم و داریم :
$$P_i = (A_{i0}, A_{i1}, \ldots, A_{in-1})$$
$$P_j = (A_{j0}, A_{j1}, \ldots, A_{jn-1})$$

فرض کنیم $A_{im}$
اولین عنصری باشد که این دو جایگشت در آن باهم متفاوتند، گوییم $P_i > P_j$ اگر و تنها اگر $A_{im} > A_{jm}$ باشد.
بنابرین اولین و آخرین جایگشت های این n تایی که کوچکترین و بزرگترین جایگشت ها نیز هستند را به صوت زیر داریم :

$$P_{first} = (p_0, p_1, \ldots, p_{n-1})  \hspace{2em} \forall i > j : p_i > p_j $$
$$P_{last} = (p_0, p_1, \ldots, p_{n-1})  \hspace{2em} \forall i > j : p_i < p_j $$

فرض کنیم در قسمتی از برنامه جایگشتی مانند $P = (a_0, a_1, \ldots, a_{n-1})$ ایجاد شده است،
حال رابطه بین $P$ و $P'$ را که جایگشت بعدی $P$ است را پیدا می کنیم.
فرض می کنیم اولین عنصری که $P$ و $P'$ که در آن باهم تفاوت دارند 
$m$ امین عنصر آرایه باشد که آن را $b$ می نامیم.
بنابرین $a_{m-1} \not = b$ و از آنجایی که $P$ < $P'$ نتیجه می گیریم $a_{m-1} < b$.

از آنجایی که $P$ بزرگترین جایگشت بین جایگشت هایی است که $m$ عنصر اول آنها با $P$ برابر است
می توانیم درباره $n - m$ عنصر باقیمانده نتیجه بگیریم که :

    \begin{align}
        a_m > a_{m+1} > \ldots > a_{n-1}
    \end{align}

از طرفی $ m - 1$ عنصر اول $P'$ با $P$ یکسان است و 
$ b \not = a_{m-1}$ پس می توانیم نتیجه بگیریم که $b$ یکی از $n-m$ عنصر باقیمانده خواهد بود
که مجموعه آنها را S می نامیم :
$$ S = \{a_m, a_{m+1}, a_{m+2}, ..., a_{n-1}\}$$

میدانیم که $P'$ کوچکترین جایگشت از بین جایگشت های بزرگتر از $P$ است
پس $b$ باید کوچکترین عضو $S$ باشد به طوری که $b > a_{m-1}$ و $a_k = b$.
چنین $b$ زمانی وجود دارد که $a_m > a_{m-1}$.

حال با پیدا کردن $b$ ، آن را با عنصر $m$ ام جابجا می کنیم و $n - m$
و مجموعه جدید را $S'$ می نامیم و داریم :
$$S' = (S \setminus \{a_k\}) \cup \{a_{m-1}\}$$

از آنجایی که $P'$ از تمام جایگشت های بعد از خود کوچکتر است پس باید از اعضای آن از $m$ به بعد به صورت نزولی مرتب شده باشند
و از $(1)$ داریم که اعضای $S'$ به صورت صعودی مرتب شده اند پس کافی است ترتیب آن ها را برعکس کنیم تا به $P'$ برسیم.

الگوریتم از $P_{first}$ که اولین جایگشت است شروع می شود

حال بنابر استقرا فرض می کنیم که الگوریتم تا جایگشت $P$ را محاسبه کرده است، نشان دادیم که الگوریتم با انجام مراحل بالا 
می تواند جایگشت بعدی $P$ را تا هنگام برقرار بودن شرط
محاسبه کند پس الگوریتم تمام جایگشت های این $n$ تایی را از
$P_{first}$ تا $P_{last}$ محاسبه می کند. بنابراین حکم ثابت شد و الگوریتم درست است.

\end{document}