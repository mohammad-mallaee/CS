\documentclass{article}
\usepackage{caption}
\usepackage{algpseudocodex}
\usepackage{algorithm}
\usepackage{amsmath}
\usepackage{xepersian}
\settextfont{XB Niloofar}

\begin{document}

الگوریتم زیر را در نظر بگیرید:

\begin{latin}
    \begin{algorithm}[H]
        \caption*{find\_k(A, k)}
        \begin{algorithmic}
            \Require a list $A[0, \ldots, m-1]$ 
            \Ensure returns index of $k$ in $A$
            \State $left$ = 1
            \State $right$ = len($A$) - 1
            \While {$A[left] < k$}
                \State $left \times = 2$                
            \EndWhile
            \State $right = left$
            \State $left //= 2$

            \While{$left \leq right$}
                $middle$ = ($left + right$) // 2
                \If {$A[middle] == k$}
                    \State return $middle$
                \ElsIf {$A[middle] < k$}
                    \State $left = middle + 1$
                \Else
                    \State $right = middle -1$
                \EndIf
            \EndWhile
            \State return -1
        \end{algorithmic}
    \end{algorithm}
\end{latin}

به طور خلاصه، آنقدر نشانه گر سمت چپ را دوبرابر می کنیم
تا به مقداری برسیم که از کلید
$k$
بیشتر است (خواه این مقدار عدد باشد، خواه بینهایت).
آنگاه، اشاره گر راست را برابر با اشاره گر چپ قرار داده و سپس اشاره گر چپ را نصف
می کنیم (ما هربار اشاره گر چپ را دوبرابر کرده بودیم، پس بعد از یافتن اشاره گر راست، مطمئن
هستیم که قبل از 
$left // 2$
عنصری نبوده که از کلید
$k$
بیشتر باشد، چون در این صورت حلقه همان جا متوقف می شد.
)

اگر دقت کنیم، تا اینجا ما به اندازه 
$\lceil \log k \rceil$
یا همان
$\lfloor \log k \rfloor + 1$
عمل مقایسه انجام داده ایم، پس کارایی الگوریتم تا اینجا برابر
$O(\log n)$
خواهد بود چون در بدترین حالت
$k$
همان $n$ خواهد بود.

اکنون که چپ و راستِ بازه ای که
$k$
در آن قرار دارد را یافته ایم، با اعمال الگوریتم جستجوی دودویی
به یافتن محل دقیق کلید
$k$
می پردازیم. 

در بدترین حالت،
$right = n$
و
$left = n//2$
است که باعث می شود طول
بازه ی جستجو
$n - \frac{n}{2} = \frac{n}{2}$
باشد و کارایی زمانی الگوریتم جستجوی دودویی
$O(\log \frac{n}{2})$
شود که همان
$O(\log n)$
است.
\end{document}