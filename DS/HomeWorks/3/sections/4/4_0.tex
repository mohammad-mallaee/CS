\documentclass{article}
\usepackage{caption}
\usepackage{algpseudocodex}
\usepackage{algorithm}
\usepackage{amsmath}
\usepackage{xepersian}
\settextfont{XB Niloofar}

\begin{document}

تصور کنید که آرایه‌ای داریم به نام 
$A$
 و به طول 
$n$.
و اینکه ما می‌دانیم که از ابتدای آرایه تا جایی
(که
ما نمی‌دانیم کجاست!) اعداد صحیح به ترتیب صعودی قرار گرفته‌اند و از آنجا به بعد تا انتهای آرایه، اعداد
صحیح به ترتیب نزولی قرار گرفته‌اند. (در حالات خاص، ممکن است آرایه قسمت صعودی یا قسمت نزولی 
نداشته باشد.)

\textbf{الف)} 
الگوریتمی با کارایی 
$O(\log n)$
برای یافتن بزرگ‌ترین عدد صحیح در آرایه $A$ بیابید.  


\textbf{ب)} تصور کنید که آرایه‌ای بسیار طولانی داریم به نام 
$A$
 و به طول 
 $m$ 
 . و اینکه ما می‌دانیم که از ابتدای آرایه تا جایی
(که ما نمی‌دانیم کجاست!) اعداد صحیح به ترتیب صعودی قرار گرفته‌اند و 
از آنجا به بعد تا انتهای آرایه، که قسمت اعظم آرایه است، نمادهای 
$\infty$
در آرایه ذخیره شده‌اند. 

فرض کنید که می‌خواهیم کلید 
$K$
 را که عددی صحیح است در آرایه 
 $A$
  جستجو کنیم. اگر 
  $n$
   (که مقدار آن برای
 ما معلوم نیست) طول قسمت کوچکی از ابتدای آرایه باشد که اعداد صحیح در آن قسمت ذخیره شده باشند، 
الگوریتمی با کارایی 
$O(\log n)$
برای جستجوی کلید 
$K$
 طراحی کنید. (توجه کنید که حتی اگر آرایه را بی‌انتها
 تصور کنیم باز می‌توان الگوریتمی با کارایی 
$O(\log n)$
 طراحی کرد.)

 
\end{document}