\documentclass[]{exam}
\usepackage{hyperref}
\usepackage[a4paper, total={6in, 10in}]{geometry}
\usepackage{caption}
\usepackage{algpseudocodex}
\usepackage{algorithm}
\usepackage{amsmath}
\usepackage{listings}
\usepackage{xepersian}
\usepackage{graphicx}
\usepackage{setspace}
\usepackage{subfiles}
\usepackage{xcolor}
\usepackage{mathtools}
\usepackage{multirow}
\usepackage{mathrsfs}
\usepackage{amssymb}
\usepackage{tikz}
\usepackage[tableaux]{prooftrees}
\renewcommand*\linenumberstyle[1]{(#1)}
\newcommand*{\lif}{\ensuremath{\mathbin{\rightarrow}}}
\settextfont{XB Niloofar}
\algrenewcommand{\algorithmicrequire}{\textbf{Input:}}
\algrenewcommand{\algorithmicensure}{\textbf{Output:}}

\definecolor{codegreen}{rgb}{0,0.6,0}
\definecolor{codegray}{rgb}{0.5,0.5,0.5}
\definecolor{codepurple}{rgb}{0.58,0,0.82}
\definecolor{backcolour}{rgb}{0.95,0.95,0.92}

\lstdefinestyle{mystyle}{
    backgroundcolor=\color{backcolour},   
    commentstyle=\color{codegreen},
    keywordstyle=\color{magenta},
    numberstyle=\tiny\color{codegray},
    stringstyle=\color{codepurple},
    basicstyle=\ttfamily\footnotesize,
    breakatwhitespace=false,         
    breaklines=true,                 
    captionpos=b,                    
    keepspaces=true,                 
    numbers=left,                    
    numbersep=5pt,                  
    showspaces=false,                
    showstringspaces=false,
    showtabs=false,                  
    tabsize=2
}

\lstset{style=mystyle}

\begin{document}
\pagestyle{head}
\firstpageheader{}{}{}
\runningheader{صفحه \thepage\ از \numpages}{}{}
\runningheadrule
\begin{tabular}{p{.75\textwidth} l}
\multicolumn{2}{c}{\textbf{به نام خدا}}\\
\multirow{2}{*}{\includegraphics[scale=0.2] {UILOGO.png}} & \\ \\
&  \textbf{علوم کامپیوتر}\\
&  \textbf{نیم‌سال دوم ۰۲-۰۳}\\
&  \textbf{مبانی منطق}\\ \\
 \textbf{دانشکده ریاضی و آمار} &  \\
\end{tabular}\\

\rule[1ex]{\textwidth}{.1pt}
\textbf{
    اعضای گروه: 
    محمد ملائی - داوود نصرتی امیرآبادی - 
    حسنا سلطان‌الکتابی - فرزانه سلیمی - یگانه رستگاری
}


\rule[1ex]{\textwidth}{.1pt}
\vspace{0pt}

\section*{تمرینات سری ۶}
\subfile{sections/1/1_0.tex}
\subsection*{جواب}
% \subsection*{\color{blue}{جواب}}
\subfile{sections/1/1_1.tex}


\subfile{sections/2/2_0.tex}
\\
\subsection*{جواب}
\subfile{sections/2/2_1.tex}

\subfile{sections/3/3_0.tex}

\subsection*{جواب}
\subfile{sections/3/3_1.tex}

\subfile{sections/4/4_0.tex}
\subsection*{جواب}
\subfile{sections/4/4_1.tex}

\subfile{sections/5/5_0.tex}
\subsection*{جواب}
\subfile{sections/5/5_1.tex}

% \pagebreak
\subfile{sections/6/6_0.tex}
\subsection*{جواب}
\subfile{sections/6/6_1.tex}

\subfile{sections/7/7_0.tex}
\subsection*{جواب}
\subfile{sections/7/7_1.tex}

\subfile{sections/8/8_0.tex}
\subsection*{جواب}
\subfile{sections/8/8_1.tex}

\subfile{sections/9/9_0.tex}
\subsection*{جواب}
\subfile{sections/9/9_1.tex}

\pagebreak
\subfile{sections/10/10_0.tex}
\subsection*{جواب}
\subfile{sections/10/10_1.tex}

% \pagebreak
\subfile{sections/11/11_0.tex}
\subsection*{جواب}
\subfile{sections/11/11_1.tex}

\pagebreak
\subfile{sections/12/12_0.tex}
\subsection*{جواب}
\subfile{sections/12/12_1.tex}
\end{document}