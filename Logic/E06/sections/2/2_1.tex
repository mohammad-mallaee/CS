\documentclass[]{exam}
\usepackage{hyperref}
\usepackage[a4paper, total={6in, 10in}]{geometry}
\usepackage{caption}
\usepackage{algpseudocodex}
\usepackage{algorithm}
\usepackage{amsmath}
\usepackage{amssymb}
\usepackage{listings}
\usepackage{xepersian}
\usepackage{graphicx}
\usepackage{setspace}
\usepackage{subfiles}
\usepackage{xcolor}
\usepackage{mathtools}
\usepackage{multirow}
\usepackage{mathrsfs}
\usepackage{tikz}

\usepackage[tableaux]{prooftrees}
\renewcommand*\linenumberstyle[1]{(#1)}
\newcommand*{\lif}{\ensuremath{\mathbin{\rightarrow}}}
\settextfont{XB Niloofar}
\begin{document}

برای اثبات این  قسمت، از مدل نقض استفاده می‌کنیم. به همین منظور ابتدا با روش تابلو اطلاعاتی به دست آورده و سپس
از آنها در ساخت مدل نقض بهره می‌جوییم.

\forestset{%
    vertical/.style={ %define style for phantom node
            %with vertical edge drawn from it
            before drawing tree={not ignore edge, edge=draw},
        },
}
\begin{tableau}
    {line no sep= 1.5cm,
        just sep= 1.5cm,
        for tree={s sep'=10mm},
    }
    [{\forall x \: (p(x) \lor q(x))}, just={ریشه}
    [{\neg (\forall x \: p(x) \lor q(x))}, just={ریشه}
    [{\neg p(a_1)}, just={2 F $\forall$ و جدید $a_1 \in C$}
    [{\neg q(a_2)}, just={2 F $\forall$ و جدید $a_2 \in C$}
    [{p(a_1) \lor p(a_1)}, just={1 T $\forall$ $a_1 \in C$}
        [{p(a_1)}, just={}, close={۳ و ۶}
        ]
        [{q(a_1)}, just={}
        [{p(a_2) \lor q(a_2)}, just={1 T $\forall$ و $a_2 \in C$}
            [{p(a_2)}, just={},
            ]
            [{q(a_2)}, just={}, close={۴ و ۸}
            ]
        ]
        ]
    ]
    ]
    ]
    ]
    ]
\end{tableau}

به نظر می‌رسد که تابلو تا همین مرحله اطلاعات خوبی را در اختیار ما قرار داده باشد. حال به تشکیل مدل نقض برای تنها شاخه‌ی باز تابلو می‌پردازیم.
\begin{align*}
    \mathscr{I} &= (D, \{p^\mathscr{I}, q^\mathscr{I}\})\\
    D &= \{\alpha_1, \alpha_2\}\\
    p^\mathscr{I} &= \{\alpha_2\}\\
    q^\mathscr{I} &= \{\alpha_1\}
\end{align*}

حال باید ثابت کنیم هنگامی که مقدم صادق است
($\mathscr{I} \models \forall x \: (p(x) \lor q(x))$)
تالی کاذب است
($\mathscr{I} \not \models \forall x \: p(x) \lor \forall x \: q(x)$).
ابتدا با استفاده از صدق تارسکی بررسی می‌کنیم 
$\mathscr{I} \overset{?}{\models} \forall x \: (p(x) \lor q(x))$.
صدق تارسکی بیان می‌کند که برای درست بودن این عبارت، لازم است تا نشان دهیم برای تک تک اعضای دامنه صادق است.
\\
میدانیم که 
$\alpha_1 \in q^{\mathscr{I}}$
پس
$\mathscr{I} \underset{\sigma [x \leftarrow \alpha_1]}{\models} p(x) \lor q(x)$.
\\
میدانیم که 
$\alpha_2 \in p^{\mathscr{I}}$
پس
$\mathscr{I} \underset{\sigma [x \leftarrow \alpha_2]}{\models} p(x) \lor q(x)$.
\\
در نتیجه برای هر 
$\alpha_i \in D$
داریم
$\mathscr{I} \underset{\sigma [x \leftarrow \alpha_i]}{\models} p(x) \lor q(x)$.
پس بنابر صدق تارسکی 
$\mathscr{I} \underset{\sigma}{\models} \forall x \: (p(x) \lor q(x))$.

حال باید ثابت کنیم
$\mathscr{I} \not \models \forall x \: p(x) \lor \forall x \: q(x)$
\\
میدانیم
$\alpha_1 \notin p^\mathscr{I}$
پس 
$\mathscr{I} \underset{\sigma [x \leftarrow \alpha_1]}{\not \models} p(x)$.
\\
میدانیم
$\alpha_2 \notin q^\mathscr{I}$
پس
$\mathscr{I} \underset{\sigma [x \leftarrow \alpha_2]}{\not \models} q(x)$.
\\
پس نتیجه می‌گیریم که عطف بالا صادق نیست و 
$\mathscr{I} \not \models \forall x \: p(x) \lor \forall x \: q(x)$
در نتیجه:
$$\forall x \: ( p(x) \lor q(x) ) \not \models \forall x \: p(x) \lor \forall x  \: q(x)$$

\end{document}