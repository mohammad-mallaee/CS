\documentclass[]{exam}
\usepackage{hyperref}
\usepackage[a4paper, total={6in, 8in}]{geometry}
\usepackage{caption}
\usepackage{algpseudocodex}
\usepackage{algorithm}
\usepackage{amsmath}
\usepackage{amssymb}
\usepackage{listings}
\usepackage{xepersian}
\usepackage{graphicx}
\usepackage{setspace}
\usepackage{subfiles}
\usepackage{xcolor}
\usepackage{mathtools}
\usepackage{multirow}
\usepackage{mathrsfs}
\settextfont{XB Niloofar}
\algrenewcommand{\algorithmicrequire}{\textbf{Input:}}
\algrenewcommand{\algorithmicensure}{\textbf{Output:}}

\definecolor{codegreen}{rgb}{0,0.6,0}
\definecolor{codegray}{rgb}{0.5,0.5,0.5}
\definecolor{codepurple}{rgb}{0.58,0,0.82}
\definecolor{backcolour}{rgb}{0.95,0.95,0.92}

\lstdefinestyle{mystyle}{
    backgroundcolor=\color{backcolour},   
    commentstyle=\color{codegreen},
    keywordstyle=\color{magenta},
    numberstyle=\tiny\color{codegray},
    stringstyle=\color{codepurple},
    basicstyle=\ttfamily\footnotesize,
    breakatwhitespace=false,         
    breaklines=true,                 
    captionpos=b,                    
    keepspaces=true,                 
    numbers=left,                    
    numbersep=5pt,                  
    showspaces=false,                
    showstringspaces=false,
    showtabs=false,                  
    tabsize=2
}

\lstset{style=mystyle}

\begin{document}

همانند تمرین قبل، نشان می‌‌دهیم که همه‌ی گزاره‌های اتمی دارای خاصیت فوق هستند. 
در واقع در هر گزاره‌ی اتمی‌ای مثل 
$p \in \mathscr{P}$
خود
$p$
یک اتم است. پس حداقل یک اتم در گزاره‌های اتمی وجود دارد و حکم اول ثابت است.

حال اثبات می‌کنیم که اگر در فرمولی چون
$A \in \mathscr{F}$
حداقل یک اتم رخ داده‌است، در
$\neg A$
نیز همین‌گونه است.
چگونه ثابت می‌کنیم؟ متاسفانه در زمان طرح سوال هنوز با مفهوم
$Sub(A)$
آشنا نشده بودیم، ولیکن ابزاری است بس کارا در جهت اثبات گزاره‌ی فوق. امیدواریم تقلب محسوب نشود :)
\\
فرض کنیم حداقل یک اتم در 
$A$
وجود دارد. یعنی
$p \in Sub(A)$.
از طرفی داریم:
$$Sub(\neg A) = Sub(A) \cup \{\neg A\}$$
چون 
$p \in Sub(A)$
پس
$p$
در اجتماع 
$Sub(A)$
با مجموعه‌ای دیگر نیز حضور دارد، یعنی 
$p \in Sub(\neg A)$.
در نتیجه در نقیض
$A$
نیز حداقل یک اتم وجود دارد.

صورت گزاره‌ی آخری که اثباتش حسن ختامی است بر اثبات ما، به شرح زیر است:
\\
اگر
$A$
و
$B$
دو فرمول باشند که در هر کدام حداقل یک اتم وجود داشته باشد،
آنگاه در فرمول
$A * B$، 
$* \in \{\land, \lor, \uparrow, \downarrow, \oplus\, \rightarrow, \leftrightarrow \}$
نیز حداقل یک اتم وجود دارد.
\\
خب، دست به کار شویم.
می‌دانیم که حداقل یک اتم مانند
$p \in \mathscr{P}$
وجود دارد به قسمی که 
$p \in Sub(A)$
و همچنین اتمی مانند
$q \in \mathscr{P}$
وجود دارد به صورتی که
$q \in Sub(B)$.
البته در بدترین شرایط
$p = q$
است و در هر دو فرمول، حداقل یک اتم مانند
$p$
وجود دارد.
با توجه به فرضیات و اینکه
$
\mathbf{
Sub(A * B) = Sub(A) \cup Sub(B) \cup \{A * B\}
}
$
داریم
$p \in Sub(A * B)$
چرا که اتم
$p$
در حداقل یکی از زیرفرمول ها حضور دارد،
پس در نتیجه در اجتماع آن زیرفرمول با دیگر مجموعه ها نیز حضور خواهد داشت.
پس گزاره سوم اثبات شده و حکم کلی ثابت است.
\end{document}