\documentclass[]{exam}
\usepackage{hyperref}
\usepackage[a4paper, total={6in, 8in}]{geometry}
\usepackage{caption}
\usepackage{algpseudocodex}
\usepackage{algorithm}
\usepackage{amsmath}
\usepackage{amssymb}
\usepackage{listings}
\usepackage{xepersian}
\usepackage{graphicx}
\usepackage{setspace}
\usepackage{subfiles}
\usepackage{xcolor}
\usepackage{mathtools}
\usepackage{multirow}
\usepackage{mathrsfs}
\settextfont{XB Niloofar}
\algrenewcommand{\algorithmicrequire}{\textbf{Input:}}
\algrenewcommand{\algorithmicensure}{\textbf{Output:}}

\definecolor{codegreen}{rgb}{0,0.6,0}
\definecolor{codegray}{rgb}{0.5,0.5,0.5}
\definecolor{codepurple}{rgb}{0.58,0,0.82}
\definecolor{backcolour}{rgb}{0.95,0.95,0.92}

\lstdefinestyle{mystyle}{
    backgroundcolor=\color{backcolour},   
    commentstyle=\color{codegreen},
    keywordstyle=\color{magenta},
    numberstyle=\tiny\color{codegray},
    stringstyle=\color{codepurple},
    basicstyle=\ttfamily\footnotesize,
    breakatwhitespace=false,         
    breaklines=true,                 
    captionpos=b,                    
    keepspaces=true,                 
    numbers=left,                    
    numbersep=5pt,                  
    showspaces=false,                
    showstringspaces=false,
    showtabs=false,                  
    tabsize=2
}

\lstset{style=mystyle}

\begin{document}

در ابتدا این خاصیت را برای گزاره‌های اتمی ثابت می‌کنیم.
به عبارت دیگر باید نشان دهیم هیچ گزاره‌ای چون
$p \in \mathscr{P}$
وجود ندارد که دنباله‌ای چون
$\land \land$
در آن ظاهر شود.

در واقع چون
$\land$
عملگر بولی است، باید در طرفین آن گزاره های اتمی
(یا فرمول ها)
قرار گیرند. به دیگر سخن، در نمایش گرافی این فرمول، عملگر
$\land$
ریشه درخت 
(یا زیردرختی)
است که فرزندان راست و چپ آن گزاره‌ها
(یا فرمول‌ها)
هستند.

مشخص است که به ازای هیچ گزاره‌ی اتمی‌ای چون
$p \in \mathscr{P}$
، 
عبارت
$\land \land p$
این شرط را دارا نیست و 
در دو طرف آن گزاره‌‌های اتمی
(یا فرمول‌ها)
قرار نگرفته اند. پس حکم درباره‌ی گزاره‌های اتمی ثابت است.
حال گزاره‌ی ذیل را ثابت می‌کنیم:

اگر
$A \in \mathscr{F}$
یک فرمول باشد که در آن
$\land \land$
ظاهر نشده است، 
در
$\neg A$
نیز ظاهر نخواهد شد.\\
در واقع با توجه به اینکه علامت نقیض صرفا در ریشه فرمول ظاهر می‌شود، تغییری در سلسله مراتب 
ادات ایجاد نکرده و منجر به ظاهر شدن
$\land \land$
در فرمول 
$\neg A$
نمی‌شود.

در نهایت گزاره‌‌ی
«
اگر
$\land \land$
در فرمول
$A$
و فرمول
$B$
ظاهر نشود، آنگاه در فرمول
$A * B$
که
$* \in \{\land, \lor, \uparrow, \downarrow, \oplus\, \rightarrow, \leftrightarrow \}$
نیز ظاهر نخواهد
»
را اثبات می‌کنیم.
\\
از آنجایی که این خاصیت در هیچ‌کدام از گزاره‌های 
$A$
و
$B$
ظاهر نشده، تنها حالت ممکن این است که فرمول ما به صورت 
$A \land B$
باشد و 
در گراف مربوط به هر فرمول، یا در سمت چپ ترین برگ
$B$
عملگر
$\land$
یا در سمت راست ترین برگ
$A$
عملگر 
$\land$
جای خوش کرده باشد.
که البته مثل روز روشن است که 
اولا 
$\land$
علمگری بولی است و محتاج دو فرزند،
و دوما در برگ های یک درخت فقط گزاره‌ها قرار می‌گیرند و نه ادات بولی. پس این فرض نیز باطل و حکم ثابت است.

در نهایت با اثبات سه گزاره‌ی فوق به این نتیجه می‌رسیم که 
$\land \land$
در هیچ فرمولی در منطق گزاره‌ای پدیدار نخواهد شد.
کسی چه می‌داند؟ شاید منطقی وجود دارد که
$\land \land$
در آن چشمک می‌زند. 


\end{document}
