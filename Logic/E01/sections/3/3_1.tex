\documentclass[]{exam}
\usepackage{hyperref}
\usepackage[a4paper, total={6in, 8in}]{geometry}
\usepackage{caption}
\usepackage{algpseudocodex}
\usepackage{algorithm}
\usepackage{amsmath}
\usepackage{amssymb}
\usepackage{listings}
\usepackage{xepersian}
\usepackage{graphicx}
\usepackage{setspace}
\usepackage{subfiles}
\usepackage{xcolor}
\usepackage{mathtools}
\usepackage{multirow}
\usepackage{mathrsfs}
\settextfont{XB Niloofar}
\algrenewcommand{\algorithmicrequire}{\textbf{Input:}}
\algrenewcommand{\algorithmicensure}{\textbf{Output:}}

\definecolor{codegreen}{rgb}{0,0.6,0}
\definecolor{codegray}{rgb}{0.5,0.5,0.5}
\definecolor{codepurple}{rgb}{0.58,0,0.82}
\definecolor{backcolour}{rgb}{0.95,0.95,0.92}

\lstdefinestyle{mystyle}{
    backgroundcolor=\color{backcolour},   
    commentstyle=\color{codegreen},
    keywordstyle=\color{magenta},
    numberstyle=\tiny\color{codegray},
    stringstyle=\color{codepurple},
    basicstyle=\ttfamily\footnotesize,
    breakatwhitespace=false,         
    breaklines=true,                 
    captionpos=b,                    
    keepspaces=true,                 
    numbers=left,                    
    numbersep=5pt,                  
    showspaces=false,                
    showstringspaces=false,
    showtabs=false,                  
    tabsize=2
}

\lstset{style=mystyle}

\begin{document}
در ابتدا ثابت می‌کنیم که خاصیت فوق برای هر گزاره‌ی اتمی صادق است.
اگر
$p$
گزاره‌ای اتمی باشد، می‌دانیم
$Sub(p) = \{p\}$.
پس در نتیجه تعداد اعضای این زیرفرمول متناهی است. یعنی
$|Sub(A)| = |\{p\}| = 1$

اکنون باید تحقیق کنیم که به ازای فرمولی چون
$A \in \mathscr{F}$
،
اگر
$Sub(A)$
متناهی باشد، آنگاه
$Sub(\neg A)$
نیز متناهی است.
\\
فرضا تعداد زیرفرمول های
$A$
متناهی است. یعنی
$|Sub(A)| = n, n \in \mathbb{N}$
.
حال داریم:
$$Sub(\neg A) = Sub(A) \cup \{\neg A\}$$
.
پس
$$|Sub(\neg A)| = n + 1$$
که عددی است متناهی.
در نتیجه حکم دوم نیز ثابت است.

حال به اثبات حکم سوم می‌پردازیم:
اگر 
$A, B \in \mathscr{F}$
فرمول هایی باشند که زیرفرمول‌های آنها متناهی باشد،
آنگاه
$$Sub(A * B) = Sub(A) \cup Sub(B) \cup \{A * B\}$$
نیز متناهی است.
\\
فرض کنیم
$$|Sub(A)| = n$$
$$|Sub(B)| = m$$
به صورتی که 
$m,n \in \mathbb{N}$.
در نتیجه 
$$|Sub(A * B)| \le |n| + |m| + |1|$$
چرا که در بدترین حالت،
$A$
و
$B$
هیچ اشتراکی ندارند و تعداد اعضای اجتماع آنها برابر با مجموع اعضای تک‌تک آنهاست. از آنجایی که 
$m + n + 1$
نیز متناهی است، پس
$Sub(A * B)$
نیز متناهی خواهد بود و حکم ثابت است.

پس صورت کلی صادق و به ازای هر فرمول
$A \in \mathscr{F}$
داریم: 
$Sub(A)$
متناهی است.
\end{document}
