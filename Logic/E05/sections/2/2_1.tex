\documentclass[]{exam}
\usepackage{hyperref}
\usepackage[a4paper, total={6in, 8in}]{geometry}
\usepackage{caption}
\usepackage{algpseudocodex}
\usepackage{algorithm}
\usepackage{amsmath}
\usepackage{amssymb}
\usepackage{listings}
\usepackage{xepersian}
\usepackage{graphicx}
\usepackage{setspace}
\usepackage{subfiles}
\usepackage{xcolor}
\usepackage{mathtools}
\usepackage{multirow}
\usepackage{mathrsfs}
\settextfont{XB Niloofar}
\algrenewcommand{\algorithmicrequire}{\textbf{Input:}}
\algrenewcommand{\algorithmicensure}{\textbf{Output:}}

\definecolor{codegreen}{rgb}{0,0.6,0}
\definecolor{codegray}{rgb}{0.5,0.5,0.5}
\definecolor{codepurple}{rgb}{0.58,0,0.82}
\definecolor{backcolour}{rgb}{0.95,0.95,0.92}

\lstdefinestyle{mystyle}{
    backgroundcolor=\color{backcolour},   
    commentstyle=\color{codegreen},
    keywordstyle=\color{magenta},
    numberstyle=\tiny\color{codegray},
    stringstyle=\color{codepurple},
    basicstyle=\ttfamily\footnotesize,
    breakatwhitespace=false,         
    breaklines=true,                 
    captionpos=b,                    
    keepspaces=true,                 
    numbers=left,                    
    numbersep=5pt,                  
    showspaces=false,                
    showstringspaces=false,
    showtabs=false,                  
    tabsize=2
}

\lstset{style=mystyle}

\begin{document}
از سمت چپ این هم ارزی شروع می‌کنیم و نشان می‌دهیم اگر سمت چپ درست باشد، سمت راست این هم‌ارزی نیز درست خواهد بود.
طبق صدق تارسکی داریم:
\begin{align}
    \mathscr{I} \underset{\sigma}{\models} \exists x_1 \dots \exists x_n A \Longleftrightarrow
    \mathscr{I} \underset{\sigma_1}{\models} \exists x_2 \dots \exists x_n A \; for \; some \; d_1, \sigma_1 := \sigma[x_1 \leftarrow d_1] \\
    \mathscr{I} \underset{\sigma_1}{\models} \exists x_2 \dots \exists x_n A \Longleftrightarrow
    \mathscr{I} \underset{\sigma_1}{\models} \exists x_3 \dots \exists x_n A \; for \; some \; d_2, \sigma_2 := \sigma_1[x_2 \leftarrow d_2]
\end{align}
با ادامه این روند می‌توانیم $\sigma'$ را به صورت $\sigma'(x_i) = d_i$ تعریف کنیم
و چون چنین $d_i$ هایی وجود دارند می‌توانیم بگوییم:
\begin{align}
    \mathscr{I} \models \exists x_1 \dots \exists x_n A \longleftrightarrow \mathscr{I} \underset{\sigma'}{\models} A
\end{align}
حالا با برعکس کردن این روند می‌توانیم طرف دیگر این هم‌ارزی را اثبات کنیم:
\begin{align}
    \mathscr{I} \underset{\sigma}{\models} A, d_n = \sigma(x_n) \longleftrightarrow
    \mathscr{I} \underset{\sigma_n}{\models} A \; for \; d_n, \sigma_n := \sigma[x_n \leftarrow d_n] \longleftrightarrow
    \mathscr{I} \underset{\sigma_n}{\models} \exists x_n A
\end{align}
با ادامه این روند نیز می‌تواینم نشان دهیم:
\begin{align}
    \mathscr{I} \underset{\sigma}{\models} A \longleftrightarrow \mathscr{I} \models \exists x_1 \dots \exists x_n A
\end{align}

برای قسمت دوم تمرین نیز داریم:
\begin{align}
    \mathscr{I} \models \forall x_1 \dots \forall x_n A \longleftrightarrow \mathscr{I} \underset{\sigma}{\models} A
\end{align}

با توجه به صدق تارسکی برای هر تخصیصی چون 
$\sigma_1$
داریم:
\begin{align}
    \mathscr{I} \models \forall x_1 \dots \forall x_n A \longleftrightarrow
    \mathscr{I} \underset{\sigma_1}{\models} \forall x_2 \dots \forall x_n A, \; \sigma_1 := \sigma[x_1 \leftarrow d] \; for \; all \; d \in D
\end{align}
پس عضوی از 
$D$
 مانند 
$d_1$
را
به دلخواه انتخاب می‌کنیم 
و داریم:
\begin{align}
    d_1 \in D: \mathscr{I} \underset{\sigma_1}{\models} \forall x_2 \dots \forall x_n A
\end{align}
در مرحله بعدی خواهیم داشت:
\begin{align}
    d_1, d_2 \in D: \mathscr{I} \underset{\sigma_2}{\models} \forall x_3 \dots \forall x_n A
\end{align}
این عملیات را تا آخر ادامه می‌دهیم تا به
$\mathscr{I} \underset{\sigma}{\models} A$
برسیم. یعنی:
\begin{align}
    d_1, \dots, d_n \in D: \mathscr{I} \underset{\sigma_n}{\models} A
\end{align}
چون 
$d_1, \dots, d_n$
را به دلخواه انتخاب کردیم، این حکم برای هر 
$\sigma$
ای برقرار است.

برای حل قسمت بازگشت این هم‌ارزی نیز، با توجه به اینکه تعبیر فوق برای هر تخصیصی برقرار است،
در هر مرحله می‌توانیم به دلخواه یک عضو مانند
$d_i \in D$
انتخاب کنیم.
همچون قسمت قبل می‌توانیم این تخصیص را با صور عمومی نمایش دهیم. این کار را برای تمام متغیرهای موجود در 
فرمول 
$A$
انجام می‌دهیم تا سرانجام همچون قسمت قبل به 
\begin{align}
    \mathscr{I} \models \forall x_1 \dots \forall x_n A 
\end{align}
دست یابیم.

\end{document}