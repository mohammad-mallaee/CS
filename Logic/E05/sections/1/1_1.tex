\documentclass[]{exam}
\usepackage{hyperref}
\usepackage[a4paper, total={6in, 8in}]{geometry}
\usepackage{caption}
\usepackage{algpseudocodex}
\usepackage{algorithm}
\usepackage{amsmath}
\usepackage{amssymb}
\usepackage{listings}
\usepackage{xepersian}
\usepackage{graphicx}
\usepackage{setspace}
\usepackage{subfiles}
\usepackage{xcolor}
\usepackage{mathtools}
\usepackage{multirow}
\usepackage{mathrsfs}
\settextfont{XB Niloofar}
\algrenewcommand{\algorithmicrequire}{\textbf{Input:}}
\algrenewcommand{\algorithmicensure}{\textbf{Output:}}

\definecolor{codegreen}{rgb}{0,0.6,0}
\definecolor{codegray}{rgb}{0.5,0.5,0.5}
\definecolor{codepurple}{rgb}{0.58,0,0.82}
\definecolor{backcolour}{rgb}{0.95,0.95,0.92}

\lstdefinestyle{mystyle}{
    backgroundcolor=\color{backcolour},   
    commentstyle=\color{codegreen},
    keywordstyle=\color{magenta},
    numberstyle=\tiny\color{codegray},
    stringstyle=\color{codepurple},
    basicstyle=\ttfamily\footnotesize,
    breakatwhitespace=false,         
    breaklines=true,                 
    captionpos=b,                    
    keepspaces=true,                 
    numbers=left,                    
    numbersep=5pt,                  
    showspaces=false,                
    showstringspaces=false,
    showtabs=false,                  
    tabsize=2
}

\lstset{style=mystyle}

\begin{document}
ابتدا پایه‌های استقرا را در نظر می‌گیریم:
\begin{itemize}
    \item فرض کنید $A$ یک فرمول اتمی بسته باشد و $A'$ تنها متغیر آزادش $x$ باشد به طوری که
    $A = \forall x \: A'(x)$.
    طبق تعریف صدق تارسکی :
    \begin{align}
        \mathscr{I} \underset{\sigma_1}{\models} \forall x \: A'(x) \Longleftrightarrow
        \mathscr{I} \underset{\sigma'_1}{\models} A' \; for \; all \; d \in D \;, \sigma'_1 := \sigma_1[x \leftarrow d] \\
        \mathscr{I} \underset{\sigma_2}{\models} \forall x \: A'(x) \Longleftrightarrow
        \mathscr{I} \underset{\sigma'_2}{\models} A' \; for \; all \; d \in D \;, \sigma'_2 := \sigma_2[x \leftarrow d]
    \end{align}
    از آنجایی که $x$ تنها متغیر آزاد $A'$ است و توسط $\sigma'$ با $d$ جایگزین می‌شود،
    مقدار این دو تابع در $x$ یکسان است
    و از آنجا که میدانستیم $\sigma_1$ و $\sigma_2$ تنها در $x$ می‌توانند تفاوت داشته باشند
    و چون طبق $(1)$ و $(2)$ نشان دادیم در $x$ نیز مقدارشان یکسان است
    ، نتیجه می‌گیریم $\sigma_1 = \sigma_2$ و :
    $$\mathscr{I} \underset{\sigma_1}{\models} A \longleftrightarrow \mathscr{I} \underset{\sigma_2}{\models} A$$
    \item اگر $A = \exists x A'(x)$ باشد، همانند قسمت قیل می‌توانیم درستی حکم را نشان دهیم.
\end{itemize}
حالا طبق استقرای ساختاری فرض می‌کنیم حکم برای $A$ و $B$
برقرار باشد نشان می‌دهیم برای $\neg A$ و $A * B$ نیز برقرار است:
\begin{align}
    \mathscr{I} \underset{\sigma_1}{\models} \neg A \leftrightarrow 
    \mathscr{I} \not \underset{\sigma_1}{\models} A \leftrightarrow 
    \mathscr{I} \not \underset{\sigma_2}{\models} A \leftrightarrow 
    \mathscr{I} \underset{\sigma_2}{\models} \neg A \\
    \mathscr{I} \underset{\sigma_1}{\models} A * B \leftrightarrow
    \mathscr{I} \underset{\sigma_1}{\models} A * \mathscr{I} \underset{\sigma_1}{\models}  B \leftrightarrow
    \mathscr{I} \underset{\sigma_2}{\models} A * \mathscr{I} \underset{\sigma_2}{\models}  B \leftrightarrow
    \mathscr{I} \underset{\sigma_2}{\models} A * B
\end{align}
حالا کافیست با استفاده از استقرا روی تعداد سورها حکم را ثابت کنیم تا اثبات به پایان برسد:

فرض می کنیم $A$ یک فرمول بسته باشد و $\lbrace x_1, \dots x_n \rbrace$
متغیرهای آزاد $A'$ باشند به طوری که : $A = \forall x_n \dots \forall x_1 A'(x_1, \dots, x_n)$
با فرض اینکه حکم برای $n-1$ سور برقرار است اثبات را انجام می‌دهیم:
\begin{align}
    \mathscr{I} \underset{\sigma_1}{\models} \forall x_n \dots \forall x_1 A' \Longleftrightarrow \mathscr{I} \underset{\sigma'_1}{\models} 
    \forall x_{n-1} \dots \forall x_1 A' \; for \; all \; d \in D \;, \sigma'_1 := \sigma_1[x_n \leftarrow d] \\
    \mathscr{I} \underset{\sigma_2}{\models} \forall x_n \dots \forall x_1 A' \Longleftrightarrow \mathscr{I} \underset{\sigma'_2}{\models} 
    \forall x_{n-1} \dots \forall x_1 A' \; for \; all \; d \in D \;, \sigma'_2 := \sigma_2[x_n \leftarrow d]
\end{align}
این قسمت نیز به طور مشابه اثبات می‌شود چرا که اگر $\forall x_{n-1} \dots \forall x_1 A'$ را $A''$ بنامیم، داریم:
\begin{align}
    \mathscr{I} \underset{\sigma_1}{\models} \forall x_n A'' \Longleftrightarrow \mathscr{I} \underset{\sigma'_1}{\models} 
    A'' \; for \; all \; d \in D \;, \sigma'_1 := \sigma_1[x_n \leftarrow d] \\
    \mathscr{I} \underset{\sigma_2}{\models} \forall x_n A'' \Longleftrightarrow \mathscr{I} \underset{\sigma'_2}{\models} 
    A'' \; for \; all \; d \in D \;, \sigma'_2 := \sigma_2[x_n \leftarrow d]
\end{align}
این بخش قبلا اثبات شده است بنابراین اثبات کامل و حکم همیشه برقرار است.
\end{document}