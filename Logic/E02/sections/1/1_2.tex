\documentclass[]{exam}
\usepackage{hyperref}
\usepackage[a4paper, total={6in, 8in}]{geometry}
\usepackage{caption}
\usepackage{algpseudocodex}
\usepackage{algorithm}
\usepackage{amsmath}
\usepackage{amssymb}
\usepackage{listings}
\usepackage{xepersian}
\usepackage{graphicx}
\usepackage{setspace}
\usepackage{subfiles}
\usepackage{xcolor}
\usepackage{mathtools}
\usepackage{multirow}
\usepackage{mathrsfs}
\settextfont{XB Niloofar}
\algrenewcommand{\algorithmicrequire}{\textbf{Input:}}
\algrenewcommand{\algorithmicensure}{\textbf{Output:}}

\definecolor{codegreen}{rgb}{0,0.6,0}
\definecolor{codegray}{rgb}{0.5,0.5,0.5}
\definecolor{codepurple}{rgb}{0.58,0,0.82}
\definecolor{backcolour}{rgb}{0.95,0.95,0.92}

\lstdefinestyle{mystyle}{
    backgroundcolor=\color{backcolour},   
    commentstyle=\color{codegreen},
    keywordstyle=\color{magenta},
    numberstyle=\tiny\color{codegray},
    stringstyle=\color{codepurple},
    basicstyle=\ttfamily\footnotesize,
    breakatwhitespace=false,         
    breaklines=true,                 
    captionpos=b,                    
    keepspaces=true,                 
    numbers=left,                    
    numbersep=5pt,                  
    showspaces=false,                
    showstringspaces=false,
    showtabs=false,                  
    tabsize=2
}

\lstset{style=mystyle}

\begin{document}

طبق فرض مسئله داریم 
$A \equiv A'$
و
$B \equiv B'$
که یعنی:
\begin{align*}
    v_\mathscr{I} (A) = v_\mathscr{I} (A')\\
    v_\mathscr{I} (B) = v_\mathscr{I} (B')
\end{align*}

\begin{itemize}
    \item برای
    $\land$
    داریم:
    
    \begin{align*}
        v_\mathscr{I} (A \land B) &= v_\mathscr{I} (A) \cdot v_\mathscr{I} (B)\\
        &= v_\mathscr{I} (A') \cdot v_\mathscr{I} (B')\\
        &= v_\mathscr{I} (A' \land B')\\
    \end{align*}
    در نتیجه ثابت شد که
    $A \land B \equiv A' \land B'$
    و حکم ثابت است. 

    \item برای
    $\lor$
    داریم:

    \begin{align*}
        v_\mathscr{I} (A \lor B) &= v_\mathscr{I} (A) + v_\mathscr{I} (B) - v_\mathscr{I} (A) \cdot v_\mathscr{I} (B)\\
        &= v_\mathscr{I} (A') + v_\mathscr{I} (B') - v_\mathscr{I} (A') \cdot v_\mathscr{I} (B')\\
        &= v_\mathscr{I} (A' \lor B')\\
    \end{align*}
    که یعنی 
    $A \lor B \equiv A' \lor B'$
    و حکم ثابت است.

    \item برای
    $\rightarrow$
    داریم:

    \begin{align*}
        v_\mathscr{I} (A \rightarrow B) &= v_\mathscr{I} (\neg A \lor B)\\
        &= 1 - v_\mathscr{I} (A) + v_\mathscr{I} (B) - (1 - v_\mathscr{I} (A)) \cdot v_\mathscr{I} (B)\\
        &= 1 - v_\mathscr{I} (A') + v_\mathscr{I} (B') - (1 - v_\mathscr{I} (A')) \cdot v_\mathscr{I} (B')\\
        &= v_\mathscr{I} (\neg A' \lor B')\\
        &= v_\mathscr{I} (A' \rightarrow B')
    \end{align*}

    پس در نتیجه
    $A \rightarrow B \equiv A' \rightarrow B'$
    و حکم ثابت است.
\end{itemize}
پس در نتیجه حکم به صورت کلی برای سه عملگر فوق ثابت میباشد.




\end{document}
