\documentclass[]{exam}
\usepackage{hyperref}
\usepackage[a4paper, total={6in, 8in}]{geometry}
\usepackage{caption}
\usepackage{algpseudocodex}
\usepackage{algorithm}
\usepackage{amsmath}
\usepackage{amssymb}
\usepackage{listings}
\usepackage{xepersian}
\usepackage{graphicx}
\usepackage{setspace}
\usepackage{subfiles}
\usepackage{xcolor}
\usepackage{mathtools}
\usepackage{multirow}
\usepackage{mathrsfs}
\settextfont{XB Niloofar}
\algrenewcommand{\algorithmicrequire}{\textbf{Input:}}
\algrenewcommand{\algorithmicensure}{\textbf{Output:}}

\definecolor{codegreen}{rgb}{0,0.6,0}
\definecolor{codegray}{rgb}{0.5,0.5,0.5}
\definecolor{codepurple}{rgb}{0.58,0,0.82}
\definecolor{backcolour}{rgb}{0.95,0.95,0.92}

\lstdefinestyle{mystyle}{
    backgroundcolor=\color{backcolour},   
    commentstyle=\color{codegreen},
    keywordstyle=\color{magenta},
    numberstyle=\tiny\color{codegray},
    stringstyle=\color{codepurple},
    basicstyle=\ttfamily\footnotesize,
    breakatwhitespace=false,         
    breaklines=true,                 
    captionpos=b,                    
    keepspaces=true,                 
    numbers=left,                    
    numbersep=5pt,                  
    showspaces=false,                
    showstringspaces=false,
    showtabs=false,                  
    tabsize=2
}

\lstset{style=mystyle}

\begin{document}
    پایه استقرا:
    فرمول
    $B$
    گزاره اتمی است، پس
    $A$
    همان
    $B$
    است. در این صورت
    $B'$
    همان
    $A'$
    است. از آنجایی که
    $A \equiv A'$
    پس
    $B \equiv B'$.
    \\
    گام استقرا:
    فرض کنید
    $B = \neg C$.
    از اینکه
    $A \in sub(B)$
    نتیجه میشود یا
    $A$
    همان
    $B$
    است و یا
    $A \in sub(C)$.
    در حالت اول 
    $B'$
    همان
    $A'$
    است و در نتیجه به وضوح
    $B \equiv B'$.
    در حالت دوم بنا بر فرض استقرا
    $C' \equiv C \{A \leftarrow A'\}$
    پس بنابر تمرین ۱ که در بالا حل شد:
    $$\neg C' \equiv \neg C \{A \leftarrow A'\}$$
    در نتیجه 
    $B' \equiv B \{A \leftarrow A'\}$
    
    \vspace*{0.5cm}
    فرض کنید
    $B \equiv C * D$.
    مسئله به سه دسته تقسیم میشود:
    \begin{itemize}
        \item $A \in sub(C)$ \\
        دیدیم که 
        $C' \equiv C \{A \leftarrow A'\}$
        و میدانیم
        $D \equiv D$
        پس بنابر تمرین ۲ که اثبات شد، داریم:
        $C' * D \equiv C * D$
        پس 
        $B' \equiv B$

        \item $A \in sub(D)$\\
        دیدیم که 
        $D' \equiv D \{A \leftarrow A'\}$
        و میدانیم
        $C \equiv C$
        پس بنابر تمرین ۲ که اثبات شد، داریم:
        $C * D' \equiv C * D$
        پس 
        $B' \equiv B$

        \item $A \equiv C * D$ \\
        در این حالت
        $B'$
        همان 
        $A'$
        است و در نتیجه اثبات میشود
        $B \equiv B'$.
    \end{itemize}
\end{document}