\documentclass[]{exam}
\usepackage{hyperref}
\usepackage[a4paper, total={6in, 8in}]{geometry}
\usepackage{caption}
\usepackage{algpseudocodex}
\usepackage{algorithm}
\usepackage{amsmath}
\usepackage{amssymb}
\usepackage{listings}
\usepackage{xepersian}
\usepackage{graphicx}
\usepackage{setspace}
\usepackage{subfiles}
\usepackage{xcolor}
\usepackage{mathtools}
\usepackage{multirow}
\usepackage{mathrsfs}
\settextfont{XB Niloofar}
\algrenewcommand{\algorithmicrequire}{\textbf{Input:}}
\algrenewcommand{\algorithmicensure}{\textbf{Output:}}

\definecolor{codegreen}{rgb}{0,0.6,0}
\definecolor{codegray}{rgb}{0.5,0.5,0.5}
\definecolor{codepurple}{rgb}{0.58,0,0.82}
\definecolor{backcolour}{rgb}{0.95,0.95,0.92}

\lstdefinestyle{mystyle}{
    backgroundcolor=\color{backcolour},   
    commentstyle=\color{codegreen},
    keywordstyle=\color{magenta},
    numberstyle=\tiny\color{codegray},
    stringstyle=\color{codepurple},
    basicstyle=\ttfamily\footnotesize,
    breakatwhitespace=false,         
    breaklines=true,                 
    captionpos=b,                    
    keepspaces=true,                 
    numbers=left,                    
    numbersep=5pt,                  
    showspaces=false,                
    showstringspaces=false,
    showtabs=false,                  
    tabsize=2
}

\lstset{style=mystyle}

\begin{document}
ابتدا فرض می‌کنیم که 
$U,A \models B$ و سپس نتیجه می‌گیریم که $U \models A \to B$.
طبق فرض اگر تفسیر دلخواه $\mathscr{I}$ یک مدل برای $U$ و $A$ باشد،
یک مدل برای $B$ نیز خواهد بود و این بدین معناست که:
\begin{align*}
    V_{\mathscr{I}}(U) = V_{\mathscr{I}}(A) = V_{\mathscr{I}}(B) = 1 \\
    \rightarrow V_{\mathscr{I}}(A \to B) = 1
\end{align*}
که حکم را ثابت می‌کند زیرا $\mathscr{I}$ یک مدل برای $A \to B$ است. \\
حالا برعکس آن را اثبات می‌کنیم و فرض می‌کنیم $U \models A \to B$.
اگر $\mathscr{I}$ یک مدل برای $U$ باشد طبق فرض یک مدل برای $A \to B$ نیز خواهد بود
و داریم :
$$V_{\mathscr{I}}(U) = V_{\mathscr{I}}(A \to B) = 1$$
و حال باید نشان دهیم اگر $V_{\mathscr{I}}(U) = V_{\mathscr{I}}(A) = 1$، 
آنگاه $V_{\mathscr{I}}(B) = 1$ و برعکس.
از فرض مسئله داریم که $V_{\mathscr{I}}(A \to B) = 1$ که یعنی
یا $V_{\mathscr{I}}(\neg A) = 1$ که با فرض مسئله در تناقض است پس نمی‌تواند برقرار باشد
یا $V_{\mathscr{I}}(B) = 1$ که حکم را ثابت می‌کند.
برعکس این موضوع نیز به راحتی اثبات خواهد شد زیرا اگر ارزش $B$ همیشه $1$ باشد
تفاوتی ندارد که ارزش $A$ چند است
و در هر صورت $V_{\mathscr{I}}(A \to B) = 1$.
\end{document}